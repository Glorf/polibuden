\documentclass[a4paper]{article}
\usepackage[polish]{babel}
\usepackage[utf8x]{inputenc}
\usepackage[T1]{fontenc}

\usepackage[a4paper,top=3cm,bottom=2cm,left=3cm,right=3cm,marginparwidth=1.75cm]{geometry}
\usepackage{amsmath}
\usepackage{graphicx}
\usepackage{float}

\author{Michał Bień, Michał Gilski, Martyna Maciejewska, Wojciech Taisner}
\title{Żółty Skrypt}
\begin{document}
\maketitle
\section{Arytmetyka przedziałowa}
\subsection{Teoria}
Działania na przedziałach:
\begin{itemize}
\item Dodawanie: $[\underline{x}, \overline{x}] + [\underline{y}, \overline{y}] = [\underline{x}+\underline{y}, \overline{x}+\overline{y}]$
\item Odejmowanie: $[\underline{x}, \overline{x}] - [\underline{y}, \overline{y}] = [\underline{x}-\overline{y}, \overline{x}-\underline{y}]$
\item Mnożenie: $[\underline{x}, \overline{x}] \cdot [\underline{y}, \overline{y}] = [\min(\underline{x}\underline{y}, \underline{x}\overline{y}, \overline{x}\underline{y}, \overline{x}\overline{y}), \max(\underline{x}\underline{y}, \underline{x}\overline{y}, \overline{x}\underline{y}, \overline{x}\overline{y})] $
\item Dzielenie: $[\underline{x}, \overline{x}]/[\underline{y}, \overline{y}] = [\underline{x}, \overline{x}]\cdot[\frac{1}{\overline{y}},\frac{1}{\underline{y}}]$ \\ \textbf{nie można dzielić przez przedział zawierający zero}
\item Średnica przedziału: $d([\underline{x}, \overline{x}]) = \overline{x} - \underline{x}$
\item Promień przedziału: $r([\underline{x}, \overline{x}]) = \frac{\overline{x} - \underline{x}}{2}$
\item Punkt środkowy: $m([\underline{x}, \overline{x}]) = \frac{\underline{x} - \overline{x}}{2}$
\item Odległość między przedziałami: $q([\underline{x}, \overline{x}], [\underline{y}, \overline{y}]) = \max(|\underline{x}-\underline{y}|, |\overline{x}-\overline{y}|)$
\item Najmniejsza wartość bezwzględna: $\forall_{x \in [\underline{x}, \overline{x}]}: \langle[\underline{x}, \overline{x}]\rangle = \min(|x|)$
\item Największa wartość bezwzględna: $\forall_{x \in [\underline{x}, \overline{x}]}: |[\underline{x}, \overline{x}]| = \max(|x|)$
\item Wartość bezwzględna przedziału: $\text{abs}([\underline{x}, \overline{x}]) = [\langle[\underline{x}, \overline{x}]\rangle, |[\underline{x}, \overline{x}]|]$
\item Kwadrat przedziału: $[\underline{x}, \overline{x}]^2 = [\langle[\underline{x}, \overline{x}]\rangle^2, |[\underline{x}, \overline{x}]|^2]$
\item Pierwiastek przedziału: $\sqrt{[\underline{x}, \overline{x}]} = [\sqrt{\underline{x}}, \sqrt{\overline{x}}]$
\item $\mathbf{e}^{[\underline{x}, \overline{x}]} = [\mathbf{e}^{\underline{x}}, \mathbf{e}^{\overline{x}}]$
\end{itemize}
\subsection{Zadania}
\subsubsection*{Zadanie 1}
Oblicz:
\begin{itemize}
\item $[-1,0]+[0,\pi]$
\item $[1,4]-[1,4]$
\item $[\frac{1}{2}, 1]-[0,\frac{1}{6}]$
\item $[2,4] - 3$
\item $-1\cdot[2,5]$
\item $[-2,3]\cdot[-2,3]$
\item $[1,\sqrt{2}]\cdot[-1,1]$
\item $[1,2]/[-2,-1]$
\item $[1,2]/[-2,1]$
\end{itemize}

\subsubsection*{Zadanie 2}
Znajdź przedziały $[x]$, $[y]$, $[z]$ dla których zachodzi:
$$ [x]\cdot([y]+[z]) \subseteq [x]\cdot[y] + [x]\cdot[z] $$

\subsubsection*{Zadanie 3}
Podaj przykład przedziału, dla którego: 
$$[x]^2 \subset [x]\cdot[x]$$

\subsubsection*{Zadanie 4}
Na przykładzie trzech, matematycznie równoważnych, zapisów funkcji $f$ zmiennej rzeczywistej $x$:
$$f(x) = \frac{1}{2-x}+\frac{1}{2+x}$$
$$f(x) = \frac{4}{4-x\cdot x}$$
$$f(x) = \frac{4}{4-x^2}$$
takiej że $|x|<2$, oraz przedziału
$$[x]=\left[-\frac{1}{2},\frac{3}{2}\right]$$
pokazać, że jej rozszerzenia przedziałowe mogą być różne. Jaka jest wartość funkcji przedziałowej odpowiadającej danej funkcji rzeczywistej dla podanego przedziału?

\subsection{Rozwiązania}
\subsubsection*{Zadanie 1}
\begin{itemize}
\item $[-1,\pi]$
\item $[-3,3]$
\item $[\frac{1}{3}, 1]$
\item $[-1,1]$
\item $[-5, -2]$
\item $[-6, 9]$
\item $[-\sqrt{2}, \sqrt{2}]$
\item $[-2, -\frac{1}{2}]$
\item \textbf{dzielenie przez przedział zawierający zero}
\end{itemize}

\subsubsection*{Zadanie 2}
Szukamy na oko, poprawne rozwiązanie to np.\\
$[x] = [-3,3]$, $[y] = [-1,2]$, $[z] = [-2,1]$ \\
Wtedy: \\
$L = [x]\cdot([y] + [z]) = [-3,3]\cdot[-3,3] = [-9,9] $\\
$P = [-6,6]+[-6,6] = [-12, 12] $ \\
Stąd $L\subseteq P$

\subsubsection*{Zadanie 3}
Wystarczy wybrać przedział z wartością ujemną po lewej stronie, np.\\
$[x] = [-5,2]$ \\
Wtedy:\\
$L = [x]^2 = [0, 4]$ \\
$[x]\cdot[x] = [-10, 4]$ \\
Stąd $L\subset P$

\subsubsection*{Zadanie 4}
Należy wyliczyć wartości przedziałów dla każdego z równań:\\
To odpowiednio:
$$[x]=\left[\frac{24}{35}, \frac{8}{3}\right]$$
$$[x]=\left[\frac{16}{19}, \frac{16}{7}\right]$$
$$[x]=\left[1,\frac{16}{7}\right]$$
Jak widać rozszerzenia przedziałowe różnią się
Aby znaleźć wartość funkcji przedziałowej należy narysować sobie wykres funkcji rzeczywistej, zaznaczyć interesujący nas przedział i znaleźć minimalną i maksymalną wartość funkcji w tym przedziale.

\section{Algorytm Hornera}
\subsection{Teoria}
Algorytm Hornera można przedstawić w postaci rekurencyjnej:\\
$w_n = a_n$ \\
$w_k = w_{k+1}x+a_k$, $k\in(n-1,0)$\\
Algorytm można wykorzystać do:
\begin{itemize}
\item Obliczenia wartości wielomianu\\
Liczymy $w(x)$ gdzie za $x$ podstawiamy liczbę, dla której wartość wielomianu chcemy policzyć.
Wynik otrzymany w $w_0$ to wartość wielomianu dla zadanego $x$.\\
\textbf{Algorytm Hornera jest optymalny - jest jedynym algorytmem, który minimalizuje liczbę dodawań i mnożeń przy obliczaniu wartości wielomianu danego w postaci naturalnej.}

\item Dzielenia wielomianu przez dwumian (y-x)\\
Za pomocą algorytmu Hornera można przedstawić wielomian jako iloczyn dwumianu $(y-x)$ oraz wielomianu $W(y)$ stopnia $n-1$ o współczynnikach wartości $a_k = w_{k+1}\cdot x$ (gdzie $k \in (n-1, 0)$ Wartość rozwiązania jest z kolei resztą z dzielenia tych wielomianów.

\item Znajdywania pochodnej znormalizowanej wielomianu\\
Stosując podejście klasyczne możemy dla wielomianu znaleźć pochodną znormalizowaną stopnia $j$ w punkcie $x$ korzystając ze wzoru:
$$\nu_j(x) = \frac{w^{(j)}(x)}{j!}$$
W przypadku algorytmu Hornera sprawa jest dużo prostsza i wystarczy wielokrotnie podzielić wielomian zgodnie z powyższym algorytmem.
\textbf{Znalezienie $m \leq n$ początkowych znormalizowanych pochodnych wielomianu wymaga $(m+1)\left(n-\frac{m}{2}\right)$ dodawań i mnożeń.}
\end{itemize}
\subsection{Zadania}
\subsubsection*{Zadanie 1}
Stosując algorytm Hornera znajdź $w(1)$ dla wielomianu
$$w(x)=x^4+x^3-4x^2-3x+3$$

\subsubsection*{Zadanie 2}
Jaki wielomian jest ilorazem z dzielenia przez dwumian $(x-1)$ wielomianu
$$w(x)=x^4+x^3-4x^2-3x+3$$

\subsubsection*{Zadanie 3}
Za pomocą algorytmu Hornera znajdź wartości wszystkich znormalizowanych pochodnych w punkcie $x = 1$ wielomianu
$$w(x)=x^4+x^3-4x^2-3x+3$$

\subsubsection*{Zadanie 4}
Stosując algorytm Hornera znajdź wartości wszystkich znormalizowanych pochodnych w punkcie $x = -1$ wielomianu
$$w(x)=4x^4-2x^2+x-3$$

\subsection{Rozwiązania}
\subsubsection*{Zadanie 1}
Dla $x=1$:\\
$w_4 = a_4 = 1$ \\
$w_3 = w_4\cdot x + a_3 = 1 \cdot 1 + 1 = 2 $ \\ 
$w_2 = w_3\cdot x + a_2 = 2 \cdot 1 - 4 = -2 $ \\
$w_1 = w_2\cdot x + a_1 = -2 \cdot 1 - 3 = -5$ \\
$w_0 = w_1\cdot x + a_0 = -5+3 = -2$ \\
Czyli $w(1) = -2$
\subsubsection*{Zadanie 2}
Rozwiązujemy zadanie ($x=1$ czyli analogicznie jak w zad 1) i bierzemy:\\
$a_3 = w_4 \cdot x = 1$\\
$a_2 = w_3 \cdot x = 2$\\
$a_1 = w_2 \cdot x = -2$\\
$a_0 = w_1 \cdot x = -5$\\
Czyli wielomian ma postać:\\
$y^3+2y^2-2y-5$ \\
A reszta wynosi: \\
$r = w_0 = -2 $
\subsubsection*{Zadanie 3}
Bierzemy wynik z poprzedniego równania i liczymy schematem Hornera wartość wielomianu w punkcie $x=1$\\
$w_3 = a_3 = 1$ \\
$w_2 = w_3\cdot x + a_2 = 1\cdot 1 + 2 = 3$ \\
$w_1 = w_2\cdot x + a_1 = 3\cdot 1 - 2 = 1$ \\
$w_0 = w_1\cdot x + a_0 = 1\cdot 1 - 5 = -4$ \\
Odczytujemy wartość pochodnej $w'(1) = -4$ i liczymy kolejną dla wielomianu: \\
$y^2 + 3y + 1$ \\
$w_2 = a_2 = 1$ \\
$w_1 = w_2\cdot x + a_1 = 1 \cdot 1 + 3 = 4$ \\
$w_0 = w_1\cdot x + a_0 = 4 \cdot 1 + 1 = 5$ \\
Odczytujemy wartość pochodnej $w''(1) = 5$ i liczymy kolejną dla wielomianu: \\
$y + 4$\\
$w_1 = a_1 = 1$ \\
$w_0 = w_1 \cdot x + a_0 = 1\cdot 1+4 = 5$ \\
Odczytujemy wartość pochodnej $w'''(1) = 5$ i liczymy dalej: \\
$w_0 = 1$\\
Tak więc $w''''(1) = 1$\\
Obliczyliśmy wszystkie pochodne znormalizowane wielomianu.

\subsubsection*{Zadanie 4}
Postępujemy jak w poprzednim zadaniu ale dla $x=-1$ i wielomianu:\\
$w(x)=x^4-2x^2+x-3$\\
$w_4 = a_4 = 1$\\
$w_3 = w_4 \cdot x + a_3 = 1 \cdot -1 + 0 = -1$\\
$w_2 = w_3 \cdot x + a_2 = -1 \cdot -1 - 2 = -1$\\
$w_1 = w_2 \cdot x + a_1 = -1 \cdot -1 + 1 = 2$\\
$w_0 = w_1 \cdot x + a_0 = 2 \cdot -1 - 3 = -5$\\
$w(y) = -y^3 + y^2 + y - 2$\\
$w_3 = a_3 = -1$\\
$w_2 = w_3 \cdot x + a_2 = -1 \cdot -1 + 1 = 2$\\
$w_1 = w_2 \cdot x + a_1 = 2 \cdot -1 + 1 = -1$\\
$w_0 = w_1 \cdot x + a_0 = -1 \cdot -1 - 2 = -1$\\
$w'(-1) = -1$ \\
$w(y) = y^2 - 2y + 1$\\
$w_2 = a_2 = 1$\\
$w_1 = w_2 \cdot x + a_1 = 1 \cdot -1 - 2 = -3$\\
$w_0 = w_1 \cdot x + a_0 = -2 \cdot -1 + 1 = 3$\\
$w''(-1) = 3$\\
$w(y) = -y + 2$\\
$w_1 = a_1 = -1$\\
$w_0 = w_1 \cdot x + a_0 = -1 \cdot -1 + 2 = 3$\\
$w'''(-1) = 3$\\
$w(y) = 1$\\
$w_0 = a_0 = 1$\\
$w''''(-1) = 1$

\section{Algorytm Shaw-Trauba}
\subsection{Teoria}
Algorytm lepszy od Hornera, gdy chcemy policzyć wiele, lub wszystkie, pochodne wielomianu\\
Określamy pewne p i q, gdzie: \\
Jeśli $q = 1$, to mamy algorytm Hornera.\\
Jeśli $q = n+1$ to mamy optymalną liczbę działań (algorytm jest optymalny ze względu na liczbę mnożeń).\\
$p=\frac{n+1}{q}$\\
Określamy funkcje pomocnicze:
$$ s(j) = (n-j)\mod q $$
$$ r(j) = \begin{cases} 0\text{,\ \ } j\mod q \neq 0 \\ q\text{,\ \ } j\mod q = 0 \end{cases} $$\\

Wyznaczamy tablicę $T^j_i$ za pomocą funkcji rekurencyjnej, gdzie:
$$ T^{-1}_i = a_{n-i-1}x^{s(i+1)} $$
$$ T^j_j = a_nx^{s(0)} $$
$$ T^j_i = T^{j-1}_{i-1}+T^j_{i-1}x^{r(i-j)} $$
Wyniki pobieramy z:
$$ T^j_n = \frac{w^{(j)}(x)}{j!}x^{j \mod q} $$


\subsection{Zadania}
\subsubsection*{Zadanie 1}
Za pomocą algorytmu Shaw-Trauba o odpowiednio dobranych wartościach $p$ i $q$ znaleźć wartości wszystkich znormalizowanych pochodnych w punkcie $x=1$ dla wielomianu:
$$w(x)=x^4+x^3-4x^2-3x+3$$

\subsection{Rozwiązania}
\subsubsection*{Zadanie 1}
Wybieramy $q = n+1 = 5$\\
Wybieramy $p = 1$\\
$s(0)=4$\\
$s(1)=3$\\
$s(2)=2$\\
$s(3)=1$\\
$s(4)=0$\\
$r(j) = \begin{cases} 0\text{,\ \ } j=1,2,3,4\\5\text{,\ \ }j=0\\\end{cases}$\\
Teraz możemy zacząć wypełniać pola tablicy:\\
$T^{-1}_0 = 1\cdot 1^3 = 1 $ \\
$T^{-1}_1 = -4\cdot 1^2 = -4 $\\
$T^{-1}_2 = -3\cdot 1^1 = -3 $\\
$T^{-1}_3 = 3 \cdot 1^0 = 3 $\\
$T^j_j = 1$\\ 
$T^0_1 = T^{-1}_0+T^0_0 = 1+1 = 2$\\
$T^0_2 = T^{-1}_1+T^0_1 = -4+2 = -2$\\
$T^0_3 = T^{-1}_2+T^0_2 = -3-2 = -5$\\
$T^0_4 = T^{-1}_3+T^0_3 = 3-5 = -2$\\
$T^1_2 = T^0_1+T^1_1 = 2 + 1 = 3$\\
$T^1_3 = T^0_2+T^1_2 = -2 + 3 = 1$\\
$T^1_4 = T^0_4+T^1_4 = -5 + 1 = -4$\\
$T^2_3 = T^1_2+T^2_2 = 3 + 1 = 4$\\
$T^2_4 = T^1_3+T^2_3 = 1 + 4 = 5$\\
$T^3_4 = T^2_3+T^3_3 = 4 + 1 = 5$\\

\section{Wzór interpolacyjny Lagrange'a}
\subsection{Teoria}
\textbf{Zadanie interpolacyjne Lagrange'a} polega na znalezieniu dla danej funkcji $f$ wielomianu $L_n$ stopnia nie wyższego niż $n$, którego wartości w $n+1$ punktach (wybranych) są takie same, jak wartości interpolowanej funkcji $f$. \\
Wielomian interpolacyjny dla tego zadania (wyprowadzenie w skrypcie Kapłana) przyjmuje postać:
$$L_n(x) = \sum^n_{i=0}f(x_i)\prod^n_{j=0, j\neq i}\frac{x-x_j}{x_i-x_j} $$
\textbf{Nie korzystać z tego wzoru! Jest niepraktyczny. Wątpiących odsyłam do rozwiązań}

\subsection{Zadania}
\subsubsection*{Zadanie 1}
Znajdź wielomian interpolacyjny Lagrange'a, który w punktach -2, 1, 2, 4 przyjmuje wartości odpowiednio 3, 1, -3, 8. Jaka jest wartość tego wielomianu w punkcie $x=0$?
\subsubsection*{Zadanie 2}
Znajdź wielomian interpolacyjny Lagrange'a, który w punktach 0, 1, 2 przyjmuje wartości odpowiednio 1, 1, 3. Jaka jest wartość tego wielomianu w punkcie $x=\frac{1}{2}$?

\subsection{Rozwiązania}

\subsubsection*{Zadanie 1}
$$L_3(x) = 3\cdot\frac{x-1}{-2-1}\cdot\frac{x-2}{-2-2}\cdot\frac{x-4}{-2-4}+1\cdot\frac{x+1}{1+2}\cdot\frac{x-2}{1-2}\cdot\frac{x-4}{1-4}+(-3)\cdot\frac{x+2}{2+2}\cdot\frac{x-1}{2-1}\cdot\frac{x-4}{2-4}+8\cdot\frac{x+2}{4+2}\cdot\frac{x-1}{4-1}\cdot\frac{x-2}{4-2}$$
WolframAlpha twierdzi, iż równa się to:
$$L_3(x) = \frac{12x^3-29x^2-63x+92}{18}$$
A więc $L_3(0) = \frac{46}{9}$

\subsubsection*{Zadanie 2}

$$L_2(x) = 1\cdot \frac{x-1}{0-1}\cdot\frac{x-2}{0-2} + 1\cdot\frac{x-0}{1-0}\cdot\frac{x-2}{1-2} + 3\cdot\frac{x-0}{2-0}\cdot\frac{x-1}{2-1}$$
$$L_2(x) = \frac{(x-1)(x-2)}{2}+\frac{x(x-2)}{-1}+\frac{3x(x-1)}{2}$$
I dalej przekształcamy\\
Nasz wielomian interpolacyjny Lagrange'a przyjmuje postać:
$$L_2(x) = x^2-x+1$$
A jego wartość w punkcie $x=\frac{1}{2}$ wynosi:
$$L_2(\frac{1}{2}) = \frac{1}{4}-\frac{1}{2} + 1 = \frac{3}{4}$$

\section{Algorytm Neville'a}
\subsection{Teoria}
Powód dla którego nie używamy poprzedniego algorytmu\\
Wyznaczamy go z wzorów rekurencyjnych:
$$P_i(x) = f(x_i)$$
$$P_{i_0,i_1\cdots i_k}(x) = \frac{(x-x_{i_0})P_{i_1i_2\cdots i_k}(x) - (x-x_{i_k})P_{i_0i_1\cdots i_{k-1}}(x)}{x_{i_k}-x_{i_0}}$$
Oczywiście jak przystało na metody Kapłana, powyższy wzór nie jest do niczego potrzebny, więc nie ma się czego bać.\\
Zadanie rozwiązujemy inaczej, przyjmijmy że szukamy wielomianu $L_3$\\
Rysujemy tabelkę:\\
\begin{tabular}{|ccccc|}
\hline
&$k=0$&$k=1$&$k=2$&$k=3$\\
\hline
$x_0$&$f(x_0)=P_{00}$&&&\\
&&$\Big>P_{11}$&&\\
$x_1$&$f(x_1)=P_{10}$&&$\Big>P_{22}$&\\
&&$\Big>P_{21}$&&$\Big>P_{33}$\\
$x_2$&$f(x_2)=P_{20}$&&$\Big>P_{32}$&\\
&&$\Big>P_{31}$&&\\
$x_3$&$f(x_3)=P_{30}$&&&\\
\hline
\end{tabular}\\
Do kolumny $k=0$ wpisujemy początkowe wartości i zaczynamy:\\
Kolejną wartość wyznacza się jako: \\
((x - indeks góry)*wartość dołu - (x - indeks dołu)*wartość góry) przez różnicę między indeksami w całej rozpiętości przedziału z którego czerpiemy.\\
Wiem, i tak brzmi okropnie, ale jak się zrobi z tego zadanie to można ogarnąć o co chodzi.

\subsection{Zadania}
\subsubsection*{Zadanie 1}
Dla wielomianu interpolacyjnego, który w punktach 0,1,2 przyjmuje wartości odpowiednio 1,1,3, znajdź wartość w punkcie $x=\frac{1}{2}$ stosując algorytm Neville'a.

\subsection{Rozwiązania}
\subsubsection*{Zadanie 1}
\begin{tabular}{|cccc|}
\hline
&$k=0$&$k=1$&$k=2$\\
\hline
$0$&$P_{00}=1$&&\\
&&$\Big>P_{11}=\frac{(\frac{1}{2}-0)\cdot 1 - (\frac{1}{2}-1)\cdot 1}{1}=1$&\\
$1$&$P_{10}=1$&&$\Big>P_{22}=\frac{(\frac{1}{2}-0)\cdot 0 - (\frac{1}{2}-2)\cdot 1}{2} = \frac{3}{4}$\\
&&$\Big>P_{21}=\frac{(\frac{1}{2}-1)\cdot 3 - (\frac{1}{2}-2)\cdot 1}{1}=0$&\\
$2$&$P_{20}=3$&&\\
\hline
\end{tabular}\\
Rozwiązaniem jest $L_3(\frac{1}{2}) = P_{22}(\frac{1}{2}) = \frac{3}{4}$

\section{Wzór interpolacyjny Newtona}
\subsection{Teoria}
Algorytm Neville'a jest przydatny gdy chcemy znaleźć wartość wielomianu interpolacyjnego w punkcie, ale co jeśli intersuje nas sam wielomian, a nie mamy ochoty na zabawę z Lagrange'em i przeklejanie wzorów do WolframAlpha?\\
Z pomocą przychodzi wzór Newtona.\\
I tu niestety sprawa się komplikuje bo musimy wprowadzić sobie kilka pojęć \\
\textbf{Iloraz różnicowy} rzędu $k$ funkcji $f$ oparty na węzłach od $x_l$ do $x_{l+k}$ to taka funkcja, że:
$$[x_l, x_{l+1},\cdots,x_{l+k};f] = \sum_{i=l}^{l+k}\frac{f(x_i)}{\prod_{j=l, j\neq i}^{l+k}(x_i - x_j)}$$
TLDR, jest to przerobiony wzór Lagrange'a którego nie liczy się łatwiej ale jest tego liczenia wymiernie mniej. Po wielu przekształceniach Kapłan przestawia wielomian interpolacyjny Newtona (równoważny z Lagrange'a):
$$N_n(x) = f(x_0) + [x_0,x_1;f](x-x_0) + [x_0,x_1,x_2;f](x-x_0)(x-x_1)+\cdots + [x_0,x_1,\cdots,x_n;f](x-x_0)(x-x_1)\cdots(x-x_{n-1})$$
Ale oczywiście nie korzystamy z tego wzoru wprost, znów robimy tabelkę i wyznaczamy to krokowo \\
\begin{tabular}{|cccc|}
\hline
&$k=0$&$k=1$&$k=2$\\
\hline
$x_0$&$f(x_0) = [x_0;f]$&&\\
&&$\Big>[x_0,x_1;f]$&\\
$x_1$&$f(x_1) = [x_1;f]$&&$\Big>[x_0,x_1,x_2;f]$\\
&&$\Big>[x_1,x_2;f]$&\\
$x_2$&$f(x_2) = [x_2;f]$&&\\
\hline
\end{tabular}\\
Korzystamy z zależności:
$$\frac{[x_{l+1},\cdots,x_{l+k};f] - [x_l,\cdots,x_{l+k-1};f]}{x_{l+k}-x_{l}} = [x_l,\cdots,x_{l+k};f]$$
Czyli odejmujemy wyższe od niższego i dzielimy przez różnicę między największym i najmniejszym x w otrzymywanym wzorze

\subsection{Zadania}
\subsubsection*{Zadanie 1}
Znajdź wielomian interpolacyjny Newtona, który w punktach 0, 1, 2 przyjmuje wartości odpowiednio: 1,1,3.

\subsubsection*{Zadanie 2}
Napisz wzór interpolacyjny Newtona dla funkcji $f(x)$ i następujących danych: $f(0)=1$, $f(2)=3$, $f(3)=2$, $f(4)=5$, $f(6)=7$

\subsubsection*{Zadanie 3}
Udowodnij, że jeżeli:
$$f(x) = (x-x_0)(x-x_1)\cdots (x-x_p)$$
i $n\leq p$ to:
$$[x_0,x_1,\cdots,x_n;f] = 0$$
Jaka jest wartość tego ilorazu, gdy $n=p+1$?

\subsection{Rozwiązania}
\subsubsection*{Zadanie 1}
\begin{tabular}{|cccc|}
\hline
&$k=0$&$k=1$&$k=2$\\
\hline
$0$&$1$&&\\
&&$\Big>\frac{1-1}{1-0}=0$&\\
$1$&$1$&&$\Big>\frac{2-0}{2-0}=1$\\
&&$\Big>\frac{3-1}{2-1}=2$&\\
$2$&$3$&&\\
\hline
\end{tabular}\\
$$N_2(x) = f(x_0)+[x_0,x_1;f](x-x_0)+[x_0,x_1,x_2;f](x-x_0)(x-x_1)$$
$$N_2(x) = 1+0\cdot(x-0)+1\cdot(x-0)(x-1)$$
$$N_2(x) = x^2 - x + 1$$
\subsubsection*{Zadanie 2}
\begin{tabular}{|cccccc|}
\hline
&$k=0$&$k=1$&$k=2$&$k=3$&$k=4$\\
\hline
$0$&$1$&&&&\\
&&$\Big>\frac{3-1}{2-0}=1$&&&\\
$2$&$3$&&$\Big>\frac{-1-1}{3-0}=-\frac{2}{3}$&&\\
&&$\Big>\frac{2-3}{3-2}=-1$&&$\Big>\frac{2+\frac{2}{3}}{4-0}=\frac{2}{3}$&\\
$3$&$2$&&$\Big>\frac{3+1}{4-2}=2$&&$\Big>\frac{-\frac{2}{3}-\frac{2}{3}}{6-0}=-\frac{2}{9}$\\
&&$\Big>\frac{5-2}{4-3}=3$&&$\Big>\frac{-\frac{2}{3}-2}{6-2}=-\frac{2}{3}$&\\
4&5&&$\Big>\frac{1-3}{6-3}=-\frac{2}{3}$&&\\
&&$\Big>\frac{7-5}{6-4}=1$&&&\\
6&7&&&&\\
\hline
\end{tabular}\\
$$N_4(x) = f(x_0)+[x_0,x_1;f](x-x_0)+[x_0,x_1,x_2;f](x-x_0)(x-x_1)+[x_0,x_1,x_2,x_3;f](x-x_0)(x-x_1)(x-x_2)+$$
$$+[x_0,x_1,x_2,x_3,x_4;f](x-x_0)(x-x_1)(x-x_2)(x-x_3)$$
Podstawiamy i otrzymujemy wynik - wzór intepolacyjny dla wielomianu:\\
$$N_4(x) = 1+1(x-0)-\frac{2}{3}(x-0)(x-2)+\frac{2}{3}(x-0)(x-2)(x-3)-\frac{2}{9}(x-0)(x-2)(x-3)(x-4)$$

\subsubsection*{Zadanie 3}
Zadanie na wkucie formułki na iloraz różnicowy. Wyjdźmy od tezy: \\
$$L=[x_0,x_1,\cdots,x_n;f]$$
Wiemy że iloraz różnicowy definiuje się jako:
$$[x_l, x_{l+1},\cdots,x_{l+k};f] = \sum_{i=l}^{l+k}\frac{f(x_i)}{\prod_{j=l, j\neq i}^{l+k}(x_i - x_j)}$$
Czyli:
$$L=[x_0,x_1,\cdots,x_n;f] = \sum_{i=0}^{n}\frac{f(x_i)}{\prod_{j=0, j\neq i}^n(x_i - x_j)}$$
Spójrzmy na licznik i przeanalizujmy wartości. Jako że wszystkie $x_i$ dla i od 0 do n są pierwiastkami równania, licznik zawsze będzie wynosił 0. Po matematycznemu:
$$\forall_{x_i: i\in (0,n)}: f(x_i) = 0$$
$$L=0$$
$$L=P$$
Teraz przypadek z $n=p+1$. Wszystkie mamy więc jeden x - $x_n$ który nie jest pierwiastkiem równania! A więc:
$$n = p+1 \rightarrow p = n-1$$

$$[x_0,x_1,\cdots,x_n;f] = 0 + 0 + \cdots + \frac{f(x_n)}{\prod_{j=0, j\neq n}^n(x_n - x_j)} = \frac{(x_n-x_0)(x_n-x_1)\cdots(x_n-x_{n-1})}{(x_n-x_0)(x_n-x_1)\cdots(x_n-x_{n-1})} = 1$$

A więc dla $n = p+1$ iloraz różnicowy przyjmuje wartość 1.

\section{Różnice progresywne i wsteczne}
\subsection{Teoria}
\textbf{Różnica zwykła (progresywna) funkcji rzędu n}, to operacja na funkcji $f$ którą zapisujemy jako $\Delta^n f$ i definiujemy wzorem:
$$\Delta^1 f(x) = f(x+h) - f(x)$$
$$\Delta^n f(x) = \Delta^{n-1}f(x+h) - \Delta^{n-1}f(x)$$
Gdzie h jest stałą różnicą pomiędzy węzłami.\\
\textbf{Różnica wsteczna rzędu n} jest analogiczna, ale do tyłu, tzn. zapisujemy ją jako:
$$\nabla^1 f(x) = f(x) - f(x-h)$$
$$\nabla^n f(x) = \nabla^{n-1}f(x) - \nabla^{n-1}f(x-h)$$
\textbf{Różnica centralna} to funkcja:
$$\delta f(x) = f(x+\frac{h}{2}) - f(x-\frac{h}{2})$$
Na końcu \textbf{operacja przesunięcia}:
$$Ef(x) = f(x+h)$$
Teraz tak.\\
Jeśli węzły funkcji są równoodległe, tzn. kolejne interpolowane $x_i$ maleją lub rosną za każdym razem o tę samą wartość stałą $h$ to możemy pokusić się o użycie wzoru interpolacyjnego dla węzłów równoodległych, co znacząco upraszcza obliczenia.\\
$$L_n(x_0+th) = L_n(t) = \sum_{i=0}^n f(x_i) \prod_{j=0, j\neq i}^n\frac{t-j}{i-j}$$
Gdzie $t$ jest stałą normalizującą wielomian:
$$t = \frac{x-x_0}{h}$$
Dla funkcji rosnących wykorzystujemy różnicę progresywną:
$$L_n(x_0+th) = L_n(t) = \sum_{i=0}^n\frac{\Delta^if(x_0)}{i!}q_i(t)$$
$$q_0(t) = 1\text{,\ \ }q_k(t)=\prod_{j=0}^{k-1}(t-j)\text{,\ \ }k=1,2,\cdots,n$$
Dla funkcji malejących wykorzystujemy różnicę wsteczną:
$$L_n(x_0+th) = L_n(t) = \sum_{i=0}^n\frac{\nabla^if(x_0)}{i!}q_i(t)$$
$$q_0(t)=1 \text{,\ \ }q_k(t) = \prod_{j=0}^{k-1}(t+j)\text{,\ \ }k=1,2,\cdots,n$$

\subsection{Zadania}
\subsubsection*{Zadanie 1}
Oblicz różnice progresywne wielomianu:
$$W_4(x)=x^4-x-1$$
przyjmując $h=1$

\subsubsection*{Zadanie 2}
Oblicz różnice wsteczne wielomianu:
$$W_4(x)=x^4-x-1$$
przyjmując $h=1$

\subsubsection*{Zadanie 3}
Udowodnij, że jeśli:
$$x_i = x_0 + ih, i=0,1,\cdots,k$$
to
$$[x_0,x_1,\cdots,x_k;f]=\frac{\Delta^kf(x_0)}{k!h^k}$$

\subsubsection*{Zadanie 4}
Pokaż, że:
$$\delta^2f(x)=(\Delta-\nabla)f(x)$$

\subsubsection*{Zadanie 5}
Udowodnij, że:
$$\Delta(\alpha f(x) + \beta g(x)) = \alpha\Delta f(x) + \beta\Delta g(x)$$

\subsubsection*{Zadanie 5}
Korzystając z przedstawienia wielomianu interpolacyjnego za pomocą różnic progresywnych znajdź ten wielomian, gdy $f(0) = 1$, $f(1) = 3$, $f(2) = 2$, $f(3) = 3$

\subsubsection*{Zadanie 6}
Korzystając z przedstawienia wielomianu interpolacyjnego za pomocą różnic wstecznych znajdź ten wielomian, gdy $f(0) = 1$, $f(1) = 3$, $f(2) = 2$, $f(3) = 3$

\subsection{Rozwiązania}
\subsubsection*{Zadanie 1}
$$\Delta^1 W_4(x) = (x+1)^4 - (x+1) - 1 - (x^4 - x - 1) = 4x^3 + 6x^2 + 4x$$
$$\Delta^2 W_4(x) = 4(x+1)^3 + 6(x+1)^2 + 4(x+1) - (4x^3 + 6x^2 + 4x) = 12x^2 + 24x + 14$$
$$\Delta^3 W_4(x) = 12(x+1)^2 + 24(x+1) + 14 - (12x^2 + 24x + 14) = 24x + 36$$
$$\Delta^4 W_4(x) = 24(x+1) + 36 - (24x + 36) = 24$$

\subsubsection*{Zadanie 2}
$$\nabla^1 W_4(x) = x^4 - x - 1 - ((x-1)^4 - (x-1) - 1) = 4x^3 - 6x^2 + 4x - 2$$
$$\nabla^2 W_4(x) = 4x^3 - 6x^2 + 4x - 2 - (4(x-1)^3 - 6(x-1)^2 + 4(x-1) -2) = 12x^2 - 24x + 14$$
$$\nabla^3 W_4(x) = 12x^2 - 24x + 14 - (12(x-1)^2 - 24(x-1) + 14) = 24x - 36$$
$$\nabla^4 W_4(x) = 24x - 36 - (24(x-1) - 36) = 24$$

\subsubsection*{Zadanie 3}
Wyjdźmy od definicji ilorazu różnicowego:
$$[x_0, x_1, \cdots, x_k; f] = \sum_{i=0}^k\frac{f(x_i)}{\prod_{j=0, j\neq i}^k(x_i - x_j)}$$
Mając dane $x_i = x_0 + ih$ możemy podstawić:
$$L = \sum_{i=0}^k\frac{f(x_0 + ih)}{\prod_{j=0, j\neq i}^k(x_0 + ih - x_0 - jh)}$$
$$L = \sum_{i=0}^k\frac{f(x_0 + ih)}{\prod_{j=0, j\neq i}^k((i - j)h)}$$
$$L = \sum_{i=0}^k\frac{f(x_0 + ih)}{h^k\prod_{j=0, j\neq i}^k(i - j)}$$
$$L = \frac{f(x_0)}{h^k\prod_{j=0, j\neq i}^k(0 - j)} + \frac{f(x_0 + h)}{h^k\prod_{j=0, j\neq i}^k(1 - j)} + \cdots + \frac{f(x_0 + kh)}{h^k\prod_{j=0, j\neq i}^k(k - j)}$$
$$L = \frac{1}{h^k}\left(\frac{f(x_0)}{(-1)(-2)\cdots(-k)} + \cdots + \frac{f(x_0 + (k-1)h)}{(k-1)(k-2)\cdots 1 \cdot (-1)} + \frac{f(x_0 + kh)}{k\cdot(k-1)\cdots 2 \cdot 1}\right)$$
Spróbujmy przeanalizować indukcyjnie dwa ostatnie wyrazy:
$$\frac{f(x_0 + (k-1)h)}{(k-1)(k-2)\cdots 1 \cdot (-1)} + \frac{f(x_0 + kh)}{k\cdot(k-1)\cdots 2 \cdot 1}$$
$$\frac{f(x_0 + ih)}{(i-k)! \cdot i!} + \frac{f(x_0 + (i+1)h)}{(i-k+1)!\cdot (i+1)!}$$
$$\frac{f(x_0 + ih)\cdot (i-k+1)(i+1) + f(x_0 + (i+1)h)}{(i-k+1)! \cdot (i+1)!}$$
\textbf{Dalej nie mam pomysłu}

\subsubsection*{Zadanie 4}
Z definicji:
$$\delta^1 f(x) = f(x+\frac{h}{2}) - f(x-\frac{h}{2})$$
$$L = \delta^2 f(x) = (f(x+h) - f(x)) - (f(x) - f(x-h)) = f(x+h) - 2f(x) + f(x-h)$$
$$P = (\Delta - \nabla) f(x) = \Delta f(x) - \nabla f(x) = f(x+h) - f(x) - (f(x) - f(x-h)) = f(x+h) - 2f(x) + f(x-h)$$
$$L = P$$

\subsubsection*{Zadanie 5}
$$L=\Delta (\alpha f(x) + \beta g(x)) = \alpha f(x+h) + \beta g(x+h) - (\alpha f(x) + \beta g(x))$$
$$L = \alpha(f(x+h)-f(x)) + \beta(g(x+h)-g(x)) = \alpha\Delta f(x) + \beta\Delta g(x)$$
$$L = P$$

\subsubsection*{Zadanie 6}
x różnią się zawsze o 1 więc $h=1$
Liczymy różnice:
$$\Delta^1 f(0) = f(1) - f(0) = 2 $$
$$\Delta^1 f(1) = f(2) - f(1) = -1 $$
$$\Delta^1 f(2) = f(3) - f(2) = 1 $$
$$\Delta^2 f(0) = \Delta^1 f(1) - \Delta^1 f(0) = -3$$
$$\Delta^2 f(1) = \Delta^1 f(2) - \Delta^1 f(1) = 2$$
$$\Delta^3 f(0) = \Delta^2 f(1) - \Delta^2 f(0) = -5$$
Korzystamy ze wzoru:
$$L_n(x_0+th) = L_n(t) = \sum_{i=0}^n\frac{\Delta^if(x_0)}{i!}q_i(t)$$
$$q_0(t) = 1\text{,\ \ }q_k(t)=\prod_{j=0}^{k-1}(t-j)\text{,\ \ }k=1,2,\cdots,n$$
Stąd:
$$L_3(t) = \frac{f(x_0)}{0!}q_0(t) + \frac{\Delta^1f(x_0)}{1!}q_1(t) + \frac{\Delta^2f(x_0)}{2!}q_2(t) + \frac{\Delta^3f(x_0)}{3!}q_3(t)$$
$$L_3(t) = 1 + 2t + \frac{-3}{2}t(t-1) + \frac{-5}{6}t(t-1)(t-2) = -\frac{5}{6}t^3+t^2+\frac{11}{6}t+1$$
$$t = \frac{x-1}{1}$$
$$L_3(x) = -\frac{5}{6}(x-1)^3+(x-1)^2+\frac{11}{6}(x-1)+1 = -\frac{5}{6}x^3+\frac{7}{2}x^2-\frac{8}{3}x+1$$

\subsubsection*{Zadanie 7}
Jak wyżej liczymy różnice, tylko $h=-1$, bo zepsuliśmy algorytm (różnice wsteczne liczymy kiedy wartość faktycznie maleje) i musimy za to zapłacić:
$$\nabla^1 f(0) = f(0) - f(1) = -2$$
$$\nabla^1 f(1) = f(1) - f(2) = 1$$
Etc, w efekcie wyjdzie to samo.

\section{Interpolacja Hermita}
\subsection{Teoria}
\emph{Zadanie interpolacyjne Hermite'a} polega na znalezieniu wielomianu stopnia mniejszego lub równego $n$, takiego że:
$$H_n^{(j)}(x_i) = f^{(j)}(x_i)$$
dla $i \in [0,k]$, $j \in [0, m_i - 1]$
przy czym:
$$\sum_{i=0}^km_i = n+1$$
\textbf{Po polsku: } możemy interpolować funkcję, która interpoluje nie tylko węzły podstawowej funkcji, ale też $j = m-1$ jej pochodnych. Stąd dla $m=1$ otrzymujemy ,,zwykłą" interpolację Lagrange'a.\\
Jak to liczymy? Po pierwsze obliczamy wartość funkcji pomocniczej $s$ dla każdego węzła $i$:
$$s(i) = \begin{cases}0,& i=0\\\sum_0^{n-1}m_n, & i>0 \end{cases}$$
Potem obliczamy wartości wielomianu pomocniczego $p_{s(i)}(x)$ dla każdego $s(i):$\\
$$p_{s(0)}(x) = 1$$
$$p_{s(i)+j}(x) = (x-x_0)^{m_0}(x-x_1)^{m_1}\cdots (x-x_i)^j$$
Teraz szukamy takich współczynników $l$, dla których kombinacja współczynników $i$ oraz $j$ jest równa kolejnym liczbom od 0 do n:\\
np. jeśli $s(1) = 1$, a $s(2) = 3$ to $l_2 = s(1) + 1 = 2$, ale już $l_3 = s(2) + 0$\\
Mając ustalone konfiguracje $l$ możemy policzyć współczynniki $b_l$ wielomianu:
$$b_l = \frac{f^{(j)}(x_i) - W_1^{(j)}(x_i)}{p_{s(i)}(x_i)\cdot j!}$$
gdzie $W_1(x_i) = \sum_{n=0}^ib_ip_i$ (gdzie $i$ jest współczynnikiem typu $l$)\\
Mając policzone te wszystkie funkcje pomocnicze, konstruujemy wielomian interpolacyjny Hermite'a, licząc po prostu:
$$H_n(x) = \sum_{l=0}^n b_lp_l(x)$$
Aby sprawdzić, czy dobrze policzyliśmy, podstawiamy wartości do warunku początkowego.\\
Wielomian Hermite'a możemy też uzyskać łatwiej, korzystając z pojęcia \emph{uogólnionych ilorazów różnicowych}.\\
W skrócie, istnieje dowód na to, że współczynnik $b_l$ wielomianu Hermite'a można zapisać jako:
$$b_l = [x_0,m_0;x_1,m_1;\cdots;x_{i-1}m_{i-1};x_i, j+1; f]$$
\subsection{Zadania}
\subsubsection*{Zadanie 1}
Dane są punkty $x_i$ o krotnościach $m_i$ oraz wartościach funkcji $f$ i jej pochodnych w tych punktach:
$$\begin{array}{c|ccc}
i&0&1&2\\
\hline
x_i&0&1&2\\
m_i&1&2&3\\
f(x_i)&0&1&0\\
f'(x_i)&-&2&1\\
f''(x_i)&-&-&2\\
\end{array}$$
Posługując się wzorami na współczynniki, skonstruuj wielomian interpolacyjny Hermite'a

\subsubsection*{Zadanie 2}
Dane są punkty $x_i$ o krotnościach $m_i$ oraz wartościach funkcji $f$ i jej pochodnych w tych punktach:
$$\begin{array}{c|ccc}
i&0&1&2\\
\hline
x_i&0&1&2\\
m_i&1&2&3\\
f(x_i)&0&1&0\\
f'(x_i)&-&2&1\\
f''(x_i)&-&-&2\\
\end{array}$$
Wyznacz współczynniki wielomianu interpolacyjnego Hermite'a korzystając z uogólnionych ilorazów różnicowych

\subsection{Rozwiązania}
\subsubsection*{Zadanie 1}
Liczymy stopień wielomianu:\\
$n = \sum m - 1 = 1+2+3 - 1 = 5$\\
Konstruujemy funkcję pomocniczą $s$ dla każdego $i$:\\
$s(0) = 0$\\
$s(1) = 1$\\
$s(2) = 1+2 = 3$\\
Liczymy $p_{s(i)}$ dla każdego $i$:\\
$p_{s(0)}(x) = 1$\\
$p_{s(1)}(x) = (x-0)^1 = x$\\
$p_{s(2)}(x) = (x-0)^1(x-1)^2 = x(x-1)^2$\\
Szukamy współczynników $l$:\\
$l = 0 = 0+0 = s(0) + 0$\\
$l = 1 = 1+0 = s(1) + 0$\\
$l = 2 = 1+1 = s(1) + 1$\\
$l = 3 = 3+0 = s(2) + 0$\\
$l = 4 = 3+1 = s(2) + 1$\\
$l = 5 = 3+2 = s(2) + 2$\\
Liczymy wyrazy $b_l$ (pominięto duże objętościowo operacje na wielomianach):\\
$b_0 = \frac{0-0}{1 \cdot 0!} = 0$\\\\
$b_1 = \frac{1-0}{1 \cdot 0!} = 1$\\\\
$b_2 = \frac{2-1}{1 \cdot 1!} = 1$\\\\
$b_3 = \frac{8-4}{2 \cdot 0!} = 2$\\\\
$b_4 = \frac{1+6}{2 \cdot 1!} = \frac{7}{2}$\\\\
$b_5 = \frac{2-21}{2 \cdot 2!} = -\frac{19}{4}$\\\\
Podstawiamy otrzymane współczynniki $b$ i wyrazy $p$ do wzoru:
$$H_5(x) = \sum_{i=0}^5 b_l p_l(x)$$
$$H_5(x) = x + x(x-1) -2x(x-2)^2 + \frac{7}{2}x(x-1)^2(x-2)-\frac{19}{4}x(x-1)^2(x-2)^2$$

\subsubsection*{Zadanie 2}
Z równania dot. uogólnionych ilorazów różnicowych wiemy, że:\\
$b_0 = [x_0,m_0; f] = f(x_0) = 0$\\
$b_1 = [x_0,m_0; x_1, m_1; f] = \frac{[x_1, 1] - [x_0, 1]}{x_1 - x_0} = \frac{1-0}{1-0} = 1$

\section{Interpolacja wymierna i trygonometryczna (Michał)}
\subsection{Teoria}

\subsection{Zadania}

\subsection{Rozwiązania}

\section{Interpolacja funkcjami sklejanymi (Michał)}
\subsection{Teoria}

\subsection{Zadania}

\subsection{Rozwiązania}



\section{Metoda Crouta (Glą)}
\subsection{Teoria}
\emph {Metoda Crouta} to jedna z metod $LU$ służąca do rozwiązywania układów równań liniowych poprzez rodzielenie macierzy współczynników $A$ na macierz dolnotrójkątną $L$ oraz górnotrójkątną $U$. 

\subsubsection*{Metoda LU}

\emph {Metoda LU} pozwala na rozwiązywanie równań postaci $\mathbf{A} \cdot \mathbf{x} = \mathbf{y}$, gdzie $\mathbf{A}$ - macierz współczynników, $\mathbf{x}$ - wektor niewiadomych, $\mathbf{y}$ - wektor danych.\newline
Macierz $A$ dzieli się na według wzoru : $$\mathbf{A} = \mathbf{L} \cdot \mathbf{U}$$ 
gdzie: 
\newline
        $$\mathbf{L} =
		\begin{bmatrix}
		l_{11} & 0      & \cdots & 0 \\
		l_{21} & l_{22} & \cdots & 0 \\
		\vdots & \vdots & \ddots & 0 \\
		l_{n1} & l_{n2} & \cdots & l_{nn}
		\end{bmatrix},\quad 		
	      \mathbf{U} = 
		\begin{bmatrix}
		u_{11} & u_{12} & \cdots & u_{1n} \\
		0      & u_{22} & \cdots & u_{2n} \\
		\vdots & \vdots & \ddots & \vdots \\
		0      & 0      & \cdots & u_{nn}
		\end{bmatrix}$$
Układ równań przyjmuje wówczas postać
$$\mathbf{L} \cdot \mathbf{U} \cdot \mathbf{x} = \mathbf{y},$$\newline
a jego rozwiązanie sprowadza się do rozwiązania dwóch układów równań z macierzami trójkątnymi:
$$\mathbf{L} \cdot \mathbf{z} = \mathbf{y},$$
$$\mathbf{U} \cdot \mathbf{x} = \mathbf{z}.$$

\textbf{Właściwości rozkładu LU:}
$$\mathbf{det(L)} = l_{11} \cdot l_{22} \cdot \cdots \cdot l_{nn}$$
$$\mathbf{det(U)} = u_{11} \cdot u_{22} \cdot \cdots \cdot u_{nn},$$
 czyli wyznaczniki są równe iloczynom przekątnych (jedno z nich jest zawsze równe 1, metodzie Croota jest macierz wyznacznik macierzy U).
 \newline
 Dodatkowo:
 $$\mathbf{det(A)}=\mathbf{det(L \cdot \mathbf U)}=\mathbf{det(L)} \cdot \mathbf{det(U)},$$
 co w przypadku metody Crouta sprowadza sie do:
 $$\mathbf{det(A)}= \mathbf{det(L)} \cdot 1 = l_{11} \cdot l_{22} \cdot \cdots \cdot l_{nn}$$

\subsubsection*{Metoda Crouta}
Metoda Crouta zakłada, że na diagonali macierzy $\mathbf{U}$ znajdują się 1. (w przeciwieństwie do bliźniaczej metody Doolittle'a w której macierz $\mathbf{L}$ ma taką diagonalę).

$$\begin{bmatrix}
a_{11} & a_{12} & \cdots & a_{1n} \\
a_{21} & a_{22} & \cdots & a_{2n} \\
\vdots & \vdots & \ddots & \vdots \\
a_{n1} & a_{n2} & \cdots & a_{nn}
\end{bmatrix} =
\begin{bmatrix}
l_{11}      & 0      & \cdots & 0 \\
l_{21} & l_{22}      & \cdots & 0 \\
\vdots & \vdots & \ddots & 0 \\
l_{n1} & l_{n2} & \cdots & l_{nn}
\end{bmatrix} \cdot
\begin{bmatrix}
1 & u_{12} & \cdots & u_{1n} \\
0      & 1 & \cdots & u_{2n} \\
\vdots & \vdots & \ddots & \vdots \\
0      & 0      & \cdots & 1
\end{bmatrix}$$
\textbf{Uwaga:} Metoda Crouta nie zadziała jeśli dowolna z liczb na przekątnej macierzy L jest równa 0, czyli nie może zajść $l_{ii} = 0$
\newline \newline
\textbf{Wzór na rozwiązanie:} 
\newline \newline
Dla macierzy 3x3 (większej raczej nie da, bo to dużo roboty, a dla większych wzór jest analogiczny)
\newline
Równanie $\mathbf{A} = \mathbf{L} \cdot \mathbf{U}$
$$ \mathbf{A} =
\begin{bmatrix}
a_{11} & a_{12} &  a_{13} \\
a_{21} & a_{22} &  a_{23} \\
a_{n1} & a_{n2} & a_{33}
\end{bmatrix} = 
\begin{bmatrix}
l_{11}      & 0      & 0 \\
l_{21} & l_{22}      & 0 \\
l_{31} & l_{32} & l_{33}
\end{bmatrix} 
\begin{bmatrix}
1 & u_{12} & u_{13} \\
0      & 1 &  u_{23} \\
0      & 0      &  1
\end{bmatrix} =
\begin{bmatrix}
l_{11} & l_{11}u_{12} & l_{11}u_{13} \\
l_{21} &l_{11}u_{12} + l_{22} & l_{21}u_{13} + l_{22}u_{23} \\
l_{31} & l_{31}u_{12} + l_{32} & l_{31}u_{13} + l_{32}u_{23} + l_{33} \\
\end{bmatrix}$$

Współczynniki $l_{ii}$ i $u_{ii}$ otrzymujemy poprzez koleje rozwiązywnie równań w kolumnach. 
\newline
Równanie $\mathbf{L} \cdot \mathbf{z} = \mathbf{y}:$

$$\begin{bmatrix}
l_{11}      & 0      & 0 \\
l_{21} & l_{22}      & 0 \\
l_{31} & l_{32} & l_{33}
\end{bmatrix} 
\begin{bmatrix}
z_{1} \\
z_{2} \\
z_{3}\\
\end{bmatrix} = 
\begin{bmatrix}
y_{1} \\
y_{2} \\
y_{3}\\
\end{bmatrix}$$
Wynik otrymujemy poprzez rozwiązanie po koleji równań rzędach macierzy, każdy wiersz ma tylko jedną niewiadomą, ponieważ poprzednie zmienne są wyzanczane z poprzednich równań.
\newline
Równanie $\mathbf{U} \cdot \mathbf{x} = \mathbf{z}:$
$$\begin{bmatrix}
1 & u_{12} & u_{13} \\
0      & 1 &  u_{23} \\
0      & 0      &  1
\end{bmatrix}
\begin{bmatrix}
x_{1} \\
x_{2} \\
x_{3}\\
\end{bmatrix} = 
\begin{bmatrix}
z_{1} \\
z_{2} \\
z_{3}\\
\end{bmatrix}$$
Analogicznie do poprzedniegu punktu, tylko że rozwiązujemy równania wierszy od dołu.

\subsection{Zadania}

\textbf{Zadanie 1:}
\newline
Rozwiąż ponizsze równanie liniowe metodą Crouta:
$$5x_{1}+4x_{2}+x_{3} = 3.4 $$
$$10x_{1}+9x_{2}+4x_{3} = 8.8 $$
$$10x_{1}+13x_{2}+15x_{3} = 19.2  $$

\subsection{Rozwiązania}

\textbf{Zadanie 1:}

$$ \mathbf{A} =
\begin{bmatrix}
5 & 4 & 1 \\
10 &9 & 4 \\
10 & 13 & 15 \\
\end{bmatrix}
\ \mathbf{x} =
\begin{bmatrix}
x_{1} \\
x_{2} \\
x_{3}\\
\end{bmatrix}
\ \mathbf{y} =
\begin{bmatrix}
3.4 \\
8.8 \\
19.2 \\
\end{bmatrix}$$

Równanie $\mathbf{A} = \mathbf{L} \cdot \mathbf{U}$ 
$$\begin{bmatrix}
5 & 4 & 1 \\
10 &9 & 4 \\
10 & 13 & 15 \\
\end{bmatrix} =
\begin{bmatrix}
l_{11}      & 0      & 0 \\
l_{21} & l_{22}      & 0 \\
l_{31} & l_{32} & l_{33}
\end{bmatrix}
\begin{bmatrix}
1 & u_{12} & u_{13} \\
0      & 1 &  u_{23} \\
0      & 0      &  1
\end{bmatrix}$$

$$\begin{bmatrix}
5 & 4 & 1 \\
10 &9 & 4 \\
10 & 13 & 15 \\
\end{bmatrix} =
\begin{bmatrix}
l_{11} & l_{11}u_{12} & l_{11}u_{13} \\
l_{21} &l_{11}u_{12} + l_{22} & l_{21}u_{13} + l_{22}u_{23} \\
l_{31} & l_{31}u_{12} + l_{32} & l_{31}u_{13} + l_{32}u_{23} + l_{33} \\
\end{bmatrix}$$

$$ l_{11}=5 \quad l_{21}=10 \quad l_{31}=10$$  
$$ u_{12}=\frac{4}{5} \quad l_{22}=1 \quad l_{32}=5$$
$$ u_{13}=\frac{1}{5} \quad u_{23}=1 \quad l_{33}=3$$

Równanie $\mathbf{L} \cdot \mathbf{z} = \mathbf{y}:$

$$\begin{bmatrix}
5 & 0 & 0 \\
10 &1 & 0 \\
10 & 5 & 3 \\
\end{bmatrix}
\begin{bmatrix}
z_{1} \\
z_{2} \\
z_{3}\\
\end{bmatrix} = 
\begin{bmatrix}
3.4 \\
8.8 \\
19.2\\
\end{bmatrix}$$

$$ 5z_{1} = 3.4 \quad z_{1} = 0.68 $$
$$ 10z_{1} + z_{2} = 8.8 \quad z_{2} = 2$$
$$ 10z_{1} + 5z_{2} + 3z_{3}  = 19.2 \quad z_{3} = 0.8$$

$$ \mathbf{z} =
\begin{bmatrix}
0.68 \\
2 \\
0.8\\
\end{bmatrix}$$


Równanie $\mathbf{U} \cdot \mathbf{x} = \mathbf{z}:$

$$\begin{bmatrix}
1 & \frac{4}{5} & \frac{1}{5} \\
0 &1 & 2 \\
0 & 0 & 1 \\
\end{bmatrix}
\begin{bmatrix}
x_{1} \\
x_{2} \\
x_{3}\\
\end{bmatrix} = 
\begin{bmatrix}
0.68 \\
2 \\
0.8\\
\end{bmatrix}$$
\newline
{\centering Zaczynamy od dołu macierzy \par}
$$ x_{3} = 0.8 $$
$$ x_{2}  + 2x_{3} = 2 \quad x_{2} = 0.4$$
$$ x_{1} + \frac{4}{5} x_{2} +  \frac{1}{5} x_{3}  = 0.68 \quad z_{1} = 0.2$$

$$ \mathbf{x} =
\begin{bmatrix}
0.2 \\
0.4 \\
0.8\\
\end{bmatrix}$$


\section{Metoda Choleskiego (Glą)}
\subsection{Teoria}
Rozkład Choleskiego jest procedurą rozkładu symetrycznej, dodatnio określonej  macierzy $A$ na iloczyn postaci:
 $$A=LL^T,$$

gdzie $L$ jest macierz trójkątna|dolną macierzą trójkątną, a $L^T$ jej macierzą transponowaną.

Macierz dowolnego typu można rozłożyć na iloczyn dolnej i górnej macierzy trójkątnej postaci $A=LU$ stosując metodę LU. Jedynie w przypadku macierzy symetrycznych i dodatnio określonych możliwy jest rozkład Choleskiego. Jeśli $A$ jest dodatnio określoną macierzą hermitowską to rozkład Choleskiego ma postać:
$$A=LL^*.$$

Metoda LU i jak stworzyć macierz dolnotrójkątną $L$ znajduje się w punkcie z metodą Crouta.

Rozpisując iloczyn $A=LL^T,$ otrzymujemy:
$$\begin{bmatrix}
a_{11} & a_{12} & \cdots & a_{1n} \\
a_{21} & a_{22} & \cdots & a_{2n} \\
\vdots & \vdots & \ddots & \vdots \\
a_{n1} & a_{n2} & \cdots & a_{nn}
\end{bmatrix}=
\begin{bmatrix}
l_{11} & 0 & \cdots & 0 \\
l_{21} & l_{22} & \cdots & 0 \\
\vdots & \vdots & \ddots & 0 \\
l_{n1} & l_{n2} & \cdots & l_{nn}
\end{bmatrix}
\begin{bmatrix}
l_{11} & l_{21} & \cdots & l_{n1} \\
0 & l_{22} & \cdots & l_{n2} \\
\vdots & \vdots & \ddots & \vdots \\
0 & 0 & \cdots & l_{nn}
\end{bmatrix}$$

Współczynniki macierzy $A$ są zatem równe:
\begin{align}
& a_{11} = l_{11}^2 & \rightarrow\ & l_{11} = \sqrt{a_{11}}\\
& a_{21} = l_{21}l_{11} & \rightarrow\ & l_{21} = \frac{a_{12}}{l_{11}}\\
& a_{22} = l_{21}^2+l_{22}^2 & \rightarrow\ & l_{22} = \sqrt{a_{22}-l_{21}^2}\\
& a_{32} = l_{31}l_{21} + l_{32}l_{22} & \rightarrow\ & l_{32} = \frac{a_{23} - l_{31}l_{21}}{l_{22}}\\
&\dots\end{align}



W ogólności (dla macierzy rzeczywistych):
$$l_{ii} = \sqrt{ a_{ii} - \sum_{k=1}^{i-1}l^2_{ik}}, \quad i=1,2,...,n.$$
$$l_{ji} = \frac{a_{ji} - \sum_{k=1}^{i-1}l_{jk}l_{ik}}{l_{ii}}, \quad j= i+1, i+2, ...,n$$
Ze względu na to, że $A$ jest dodatnio określona, wyrażenie pod pierwiastkiem jest zawsze dodatnie.
\newline \newline

Dla wszystkich macierzy:
$$l_{ii} = \sqrt{ a_{ii} - \sum_{k=1}^{i-1}|l_{ik}|^2}, \quad i=1,2,...,n.$$
$$l_{ji} = \frac{a_{ji} - \sum_{k=1}^{i-1}l_{jk}l_{ik}^-}{l_{ii}^-}, \quad j= i+1, i+2, ...,n$$



Powyższe wzory wynikają bezpośrednio z przedstawienia macierzy $A$ jako iloczyn $LL^T$ , a
następnie iterowania wierszami po kolejnych elementach macierzy $L$.
Żeby rozwiązać układ równań $Ax = LL^T x = b$ z taką macierzą wystarczy rozwiązać naj-
pierw układ równań z macierzą dolnotrójkątną Ly = b, a następnie układ równań z macierzą
górnotrójkątną $L^T x = y$ (liczby sprzężone zastępujemy odpowiednimi liczbami rzeczywisty-
mi).


\subsection{Zadania}

\subsection{Rozwiązania}

\section{Metoda Jacobiego (Wojtek)}
\subsection{Teoria}
\textbf{Wstęp}
[I - to macierz jednostkowa]\newline
Metoda Jacobiego należy do metod iterayjnych rozwiązywania układu równań liniowych $Ax = b$ Po pewnych przekształceniach i stosowaniu teorii (ponad strona) wychodzi nam $M = I - NA$ Otrzymujemy w ten sposób rodzinę metod iteracyjnych w postaci:
$$ x^{(i+1)} = (I-NA)x^{(i)} + Nb $$
która umozliwia wyznaczenie wektora $x'$ rozwiązań układu równań, o ile $p(I-NA)<1$
\newline Exclimer: $p$ jest promieniem spektralnym macierzy -> jest to największa z wartości własnych macierzy oznaczancyh jako $\lambda$. Wartości własne macierzy to pierwiastki równania charaktersytycznego macierzy. 
$$w(\lambda) = det(A-\lambda I) = 0$$
Rozkład macierzy A na L+D+U
\begin{itemize}
    \item L(lower) - macierz poddiagonalna
    \item D(diagonal) - macierz diagonalna
    \item U(upper) - macierz ponaddiagonalna
\end{itemize}
Przykład:
$$ A \quad = \quad 
\begin{bmatrix}
1 & 2 & 3 \\
4 & 5 & 6 \\
7 & 8 & 9
\end{bmatrix}
$$
$$ L \quad = \quad 
\begin{bmatrix}
0 & 0 & 0 \\
4 & 0 & 0 \\
7 & 8 & 0
\end{bmatrix} \quad
D \quad = \quad 
\begin{bmatrix}
1 & 0 & 0 \\
0 & 5 & 0 \\
0 & 0 & 9
\end{bmatrix} \quad
U \quad = \quad 
\begin{bmatrix}
0 & 2 & 3 \\
0 & 0 & 6 \\
0 & 0 & 0
\end{bmatrix}
$$
\textbf{Właściwa metoda Jacobiego}\newline
Jeżeli przyjmiemy, że: $N = D^{-1}$ tj. $M = -D^{-1}(L+U)$, otrzymamy metodę Jackobiego:
$$ Dx^{(i+1)} = -(L+U)x^{(i)} + b, \quad i = 0, 1, ...$$
Stosowanie tego wzoru wymaga, by elementy na głównej przekątnej macierzy A były niezerowe (jak są, to poprzestawiać wiersze - nie zmienia to zbieżności).
Algorytm:
\begin{itemize}
    \item spośród kolumn (gdzie w głownej przekątnej jest zero), wybieramy tę, gdzie jest najwieksza liczba zer
    \item w tej kolumnie wybieramy element o maksymalnym module i przestawiamy wiersze tak, by ten element był na głównej przekątnej, po czym "ustalamy go", czyli pomijamy w dalszych obliczeniach.
    \item kontynuować do skutku.
\end{itemize}
Warunek zbieżności metody Jackobiego:
$$ p(M_J)=p(-D^{-1}(L+U))<1 $$
\subsection{Zadania}
\subsubsection{Zadanie 1}
Wykazać, że dla macierzy A metoda Jackobiego nie gwarantuje zbieżności przy dowolnie wybranym przybliżeniu początkowym.

$$A \quad = \quad 
\begin{bmatrix}
1 & -\frac{1}{2} &  \frac{1}{2} \\
 1 & 1 &  1 \\
 -\frac{1}{2} &  -\frac{1}{2} & 1
\end{bmatrix}
$$


\subsubsection{Zadanie 2}
Wykazać, że dla macierzy A metoda Jackobiego jest zbieżna przy dowolnie wybranym elemencie początkowym.

$$A \quad = \quad 
\begin{bmatrix}
1 & \frac{1}{\sqrt{3}} &  \frac{1}{\sqrt{3}} \\
 \frac{1}{\sqrt{3}} & 1 &  \frac{1}{\sqrt{3}} \\
 \frac{1}{\sqrt{3}} &  \frac{1}{\sqrt{3}} & 1
\end{bmatrix}
$$


\subsection{Rozwiązania}

\subsubsection{Zadanie 1}
$$A \quad = \quad 
\begin{bmatrix}
1 & -\frac{1}{2} &  \frac{1}{2} \\
 1 & 1 &  1 \\
 -\frac{1}{2} &  -\frac{1}{2} & 1
\end{bmatrix}
$$

$$L \quad = \quad 
\begin{bmatrix}
0 & 0 &  0 \\
 1 & 0 &  0 \\
 -\frac{1}{2} &  -\frac{1}{2} & 0
\end{bmatrix}
\quad
D \quad = \quad 
\begin{bmatrix}
1 & 0 &  0 \\
 0 & 1 &  0 \\
 0 &  0 & 1
\end{bmatrix}
\quad
U \quad = \quad 
\begin{bmatrix}
0 & -\frac{1}{2} &  \frac{1}{2} \\
0 & 0 &  1 \\
 0 &  0 & 0
\end{bmatrix}
$$

$$M_J = -D^{-1}(L+U),\quad M_J\quad = \quad 
\begin{bmatrix}
0 & -\frac{1}{2} & \frac{1}{2} \\
 1 & 0 &  1 \\
 -\frac{1}{2} &  -\frac{1}{2} & 0
\end{bmatrix}
$$

$$ p(M_J) = det(M_J - \lambda I)<1$$

$$
\begin{vmatrix}
-\lambda & -\frac{1}{2} &  \frac{1}{2} \\
1 & -\lambda &  1 \\
-\frac{1}{2} &  -\frac{1}{2} & -\lambda
\end{vmatrix}
\quad = \quad -\lambda^3 - \frac{5}{4}\lambda
$$

$$ p(M_J) = max(\lambda) = max(0, \sqrt{\frac{5}{4}},-\sqrt{\frac{5}{4}}) =  \sqrt{\frac{5}{4}} > 1 $$
Dla tej macierzy metoda Jacobiego nie jest zbieżna przy dowolnym wyborze elementu początkowego.
\subsubsection{Zadanie 2}
$$A \quad = \quad 
\begin{bmatrix}
1 & \frac{1}{\sqrt{3}} &  \frac{1}{\sqrt{3}} \\
 \frac{1}{\sqrt{3}} & 1 &  \frac{1}{\sqrt{3}} \\
 \frac{1}{\sqrt{3}} &  \frac{1}{\sqrt{3}} & 1
\end{bmatrix}
$$

$$L \quad = \quad 
\begin{bmatrix}
0 & 0 &  0 \\
 \frac{1}{\sqrt{3}} & 0 &  0 \\
 \frac{1}{\sqrt{3}} &  \frac{1}{\sqrt{3}} & 0
\end{bmatrix}
\quad
D \quad = \quad 
\begin{bmatrix}
1 & 0 &  0 \\
 0 & 1 &  0 \\
 0 &  0 & 1
\end{bmatrix}
\quad
U \quad = \quad 
\begin{bmatrix}
0 & \frac{1}{\sqrt{3}} &  \frac{1}{\sqrt{3}} \\
0 & 0 &  \frac{1}{\sqrt{3}} \\
 0 &  0 & 0
\end{bmatrix}
$$

$$M_J = -D^{-1}(L+U),\quad M_J\quad = \quad 
\begin{bmatrix}
0 & \frac{1}{\sqrt{3}} &  \frac{1}{\sqrt{3}} \\
 \frac{1}{\sqrt{3}} & 0 &  \frac{1}{\sqrt{3}} \\
 \frac{1}{\sqrt{3}} &  \frac{1}{\sqrt{3}} & 0
\end{bmatrix}
$$

$$ p(M_J) = det(M_J - \lambda I)<1$$

$$
\begin{vmatrix}
-\lambda & \frac{1}{\sqrt{3}} &  \frac{1}{\sqrt{3}} \\
 \frac{1}{\sqrt{3}} & -\lambda &  \frac{1}{\sqrt{3}} \\
 \frac{1}{\sqrt{3}} &  \frac{1}{\sqrt{3}} & -\lambda
\end{vmatrix}
\quad = \quad -\lambda^3 + \lambda - \frac{2}{3\sqrt{3}}
$$

$$ p(M_J) = max(\lambda) = max( \frac{1}{\sqrt{3}},-\frac{2}{\sqrt{3}}) = \frac{1}{\sqrt{3}} <1 $$
Dla tej meacierzy metoda Jackobiego jest zbieżna.

\section{Metoda Gaussa-Seidla (Wojtek)}
\subsection{Teoria}
Jak ktoś nie czytał polecam wstęp teoretyczny z metody Jacobiego, bo nie będę przepisywał, chrońmy drzewa. \newline
\textbf{Metoda Gaussa-Seidla}\newline
Jeżeli przyjmiemy, że: $N = (D+L)^{-1}$, tj. $M = -(D+L)^{-1}U$, otrzymamy metodę Gaussa-Seidla:
$$ Dx^{(i+1)} = -Lx^{(i+1)} - Ux^{(i)} + b,\quad i=0,1,...$$
Warunek zbieżności metody Gaussa-Seidla:
$$ p(M_{GS}) = p(-(D+L)^{-1}U) <1 $$

\textbf{Która metoda jest szybciej zbieżna}\newline
Jeżeli: $$ 0<p(M_J)<1$$
to $$ p(M_{GS}) < p(M_J) $$
Innymi słowy: Jeżeli metoda Jacobiego jest zbieżna i promień spektralny jej macierzy M jest dodatni, to metoda Gaussa-Seidla jest asymptotycznie szybciej zbieżna.
Zauważmy, że powyższe twierdzenie nie rozstrzyga, która metoda jest lepsza (szybciej zbieżna, o ile w ogóle jest zbieżna) w przypadku, gdy promień spektralny macierzy M w metodzie Jacobiego jest równy 0.


\subsection{Zadania}

\subsubsection{Zadanie 1} 
Dla jakich wartości $\alpha$ układ równań można rozwiązać metodą Gaussa-Seidla?
$$x_1 + x_2 + x_3 = 1, $$
$$ \alpha x_3 = 0, $$
$$ \alpha x_2 + x_3 = 0 $$

\subsubsection{Zadanie 2}
Zbadać zbieżność macierzy A metoda Gaussa-Seidla.

$$A \quad = \quad 
\begin{bmatrix}
1 & \frac{1}{\sqrt{3}} &  \frac{1}{\sqrt{3}} \\
 \frac{1}{\sqrt{3}} & 1 &  \frac{1}{\sqrt{3}} \\
 \frac{1}{\sqrt{3}} &  \frac{1}{\sqrt{3}} & 1
\end{bmatrix}
$$
\subsection{Rozwiązania}

\subsubsection{Zadanie 1} 
Uwaga, obliczenia wymagają sprawdzenia, przez kogoś, kto sie zna na rachunku macierzowym //Wojtek
$$A \quad = \quad 
\begin{bmatrix}
1 & 1 &  1 \\
 0 & 0 &  \alpha \\
 0 &  \alpha & 1
\end{bmatrix}
\quad = \quad
\begin{bmatrix}
1 & 1 &  1 \\
0 &  \alpha & 1  \\
 0 & 0 &  \alpha
\end{bmatrix}
$$

$$L \quad = \quad 
\begin{bmatrix}
0 & 0 &  0 \\
 0 & 0 &  0 \\
 0 &  0 & 0
\end{bmatrix}
\quad
D \quad = \quad 
\begin{bmatrix}
1 & 0 &  0 \\
 0 & \alpha &  0 \\
 0 &  0 & \alpha
\end{bmatrix}
\quad
U \quad = \quad 
\begin{bmatrix}
0 & 1 & 1 \\
0 & 0 &  1 \\
 0 &  0 & 0
\end{bmatrix}
$$

$$M_{GS} = -(D+L)^{-1}U,\quad M_{GS}\quad = \quad -
\begin{bmatrix}
1 & 0 & 0 \\
 0 & \alpha & 0 \\
 0 &  0 & \alpha
\end{bmatrix}
^{-1} *
\begin{bmatrix}
0 & 1 & 1 \\
0 & 0 &  1 \\
 0 &  0 & 0
\end{bmatrix}
$$

$$ -
\begin{bmatrix}
1 & 0 & 0 \\
 0 & \frac{1}{\alpha} & 0 \\
 0 &  0 & \frac{1}{\alpha}
\end{bmatrix}
*
\begin{bmatrix}
0 & 1 & 1 \\
0 & 0 &  1 \\
 0 &  0 & 0
\end{bmatrix}
\quad = \quad -
\begin{bmatrix}
0 & 1 & 1 \\
0 & 0 &  \frac{1}{\alpha} \\
 0 &  0 & 0
\end{bmatrix}
$$


$$ p(M_{GS}) = det(M_{GS} - \lambda I)<1$$

$$ -
\begin{vmatrix}
-\lambda & 1 &  1 \\
 0 & -\lambda &  \frac{1}{\alpha} \\
 0 &  0 & -\lambda
\end{vmatrix}
\quad = \quad \lambda^3
$$

$$ p(M_J) = max(\lambda) =0  <1,
\quad \alpha \neq 0 $$
Podany układ równań można rozwiązać metodą Gaussa-Seidla gdy $\alpha \neq 0 $
\subsubsection{Zadanie 2} 
$$A \quad = \quad 
\begin{bmatrix}
1 & \frac{1}{\sqrt{3}} &  \frac{1}{\sqrt{3}} \\
 \frac{1}{\sqrt{3}} & 1 &  \frac{1}{\sqrt{3}} \\
 \frac{1}{\sqrt{3}} &  \frac{1}{\sqrt{3}} & 1
\end{bmatrix}
$$

$$L \quad = \quad 
\begin{bmatrix}
0 & 0 &  0 \\
 \frac{1}{\sqrt{3}} & 0 &  0 \\
 \frac{1}{\sqrt{3}} &  \frac{1}{\sqrt{3}} & 0
\end{bmatrix}
\quad
D \quad = \quad 
\begin{bmatrix}
1 & 0 &  0 \\
 0 & 1 &  0 \\
 0 &  0 & 1
\end{bmatrix}
\quad
U \quad = \quad 
\begin{bmatrix}
0 & \frac{1}{\sqrt{3}} &  \frac{1}{\sqrt{3}} \\
0 & 0 &  \frac{1}{\sqrt{3}} \\
 0 &  0 & 0
\end{bmatrix}
$$
No i tu koniec łatwych rzeczy xDDD (rak macierzowy)
$$M_{GS} = 
\begin{bmatrix}
1 & 0 &  0 \\
 \frac{1}{\sqrt{3}} & 1 &  0 \\
 \frac{1}{\sqrt{3}} &  \frac{1}{\sqrt{3}} & 1
\end{bmatrix}^{-1} * 
\begin{bmatrix}
0 & \frac{1}{\sqrt{3}} &  \frac{1}{\sqrt{3}} \\
0 & 0 &  \frac{1}{\sqrt{3}} \\
 0 &  0 & 0
\end{bmatrix}
$$
Na szczęście tu jest wolfram (policzyłem to ręcznie, ale umarłem na liczeniu wielomianu charakterystycznego, jak będzie taka macierz, to nie ma co czasu tracić, tylko Jakobim)
$$
M_{GS} = 
\begin{bmatrix}
0 & \frac{1}{\sqrt{3}} &  \frac{1}{\sqrt{3}} \\
 0 & -\frac{1}{3} & -\frac{1}{3}+\frac{1}{\sqrt{3}} \\
 0 &  -\frac{1}{3} + \frac{1}{3\sqrt{3}} & -\frac{2}{3} + \frac{1}{3\sqrt{3}}
\end{bmatrix}
$$
Wielomian charakterystyczny:
$$ \lambda(-\lambda^2 +\frac{\sqrt{3}-9}{9} \lambda - \frac{\sqrt{3}}{9}) $$
Jak widać wielomian ma tylko jedno miejsce zerowe (w dziedzinie liczb rzeczywistych) w $\lambda = 0$, i to jest nasz promień spektralny macierzy. Wiedząc, że 0<1 stwierdzamy że możemy użyć metody Gaussa-Seidla.
\section{Metoda połowienia}
\subsection{Teoria}
Metoda połowienie a.k.a. bisekcja służy do znajdywania miejsc zerowych wielomianu.
\newline
Załóżmy że mamy funkcję ciągłą w przedziale $[a,b]$ , wewnętrz którego znajduje się dokładnie jeden pierwiastek, a na jego końcach funkcja przyjmuje przeciwne znaki, tj. $f(a)*f(b)<0$.Wtedy, w takim przediale funkcja musi mieć miejsce zerowe.
W celu znalezienia przybliżonej wartości pierwiastka, dzielimy przedział $[a,b]$, na połowy, punktem 
$$ x^{(1)} = \frac{a+b}{2} $$

Jeśli $f(x^{(1)}) = 0$ to $x^{(1)}$ jest miejscem zerowym. W przeciwnym wypadku rozważamy dwa przedziały $[a,x^{(1)}]$ i $[x^{(1)},b]$. Wybieramy ten, na którego końcach funkcja przyjmuje przeciwne znaki.
Całość przetwarzamy rekurencyjnie.

Po pewnej liczbie kroków otrzymamy: dokłądny pierwiastek, albo ciąg przedziałów, takich że wartość funkcji z ich końców jest przeciwnych znaków.
$$f(x^{(i)}),f(x^{(i+1)})<0$$
No i to by było fajnie gdyby nie to: 
\begin{quote}
    Ponieważ lewe końce ciągu przedziałów tworzą ciąg niemalejący i ograniczony z góry, a prawe końce – ciąg nierosnący i ograniczony z dołu, więc z zależności (5.2) wynika, że istnieje ich wspólna granica, która jest szukanym pierwiastkiem.
\end{quote}
Wzór 5.2: $$|x^{(i+1)} - x^{(i)}| = \frac{1}{2^i}(b-a)$$

Wzór 5.2 ma jedno bardo ważne zastosowanie - jego lewa strona to szerokość przedziału. W zadaniach jest nam potrzebne, że szerokość przedziału można zamienić z dokładnością metody, dzięki czemu możemy policzyć liczbę przebiegów(kroków) algorytmu bisekcji.

\subsection{Zadania}

    \subsubsection*{Zadanie 1} Równanie $x^2 - 2 = 0$ ma pierwiastki. Ile iteracji należy wykonać w metodzie połowienia, aby startując z przedziału $[1, 2]$ obliczyć pierwiastek z dokładnością do czterech miejsc dziesiętnych? Jaki jest maksymalny błąd po tej liczbie iteracji?
    \subsubsection*{Zadanie 2} Pokazać że równanie ma $sin x + x -1 = 0$ pierwiastek w przedziale $[0, 1]$. Ile iteracji trzeba wykonć, aby metodą połowienia otrzymać przybliżoną wartość pierwiastka z błędem nie przekraczającym $0,5 *10^{-4}$ ?
    \subsubsection*{Zadanie 3} 3. Ile iteracji należy wykonać, aby metodą połowienia znaleźć pierwiastek równania $x^3-x-1 = 0$ z dokladnością $10^{-4}$ , który leży w przedziale [1, 2]?

\subsection{Rozwiązania}
\subsubsection*{Zadanie 1}
No to jedziemy: 
$$\epsilon = 10^{-4},\quad \frac{b-a}{2^i}<\epsilon ,\quad [a,b]=[1,2]$$
$$ \frac{2-1}{2^i} < 10^{-4}$$
$$ 2^{-i} < 10^{-4}, \quad 2^i>10^4$$
$$ log_2 10^4 < log_2 2^i , \quad 4log_2 10^4 < i$$
$$ 4\frac{log_{10}10}{log_{10}2}<i $$
$$ \frac{4}{log_{10}2}<i $$
Loguś który się ostał jest wartością tablicową i ostatecznie wychodzi nam: 
$$ i>13,29 ,\quad i = 14$$
Jeszcze tylko obliczamy maksymalny błąd ze wzoru:
$$ \frac{b-a}{2^i} = 2^{-14}$$

\subsubsection*{Zadanie 2}
1. Sprawdzamy wartości funkcji na końcach przedziału
$$f(0) = sin(0) + 0 - 1 = -1, \quad sin(1) +1 -1 = sin(1)>0 $$
Są róznych znaków więc w tym przedziale jest rozwiązanie \newline
2. Sprawdzamy czy funkcja jest ciągła -> tak jest ciągła
\newline Wszystko działa
\newline Ile iteracji wykonać: generalnie prawie taki sam przykład jak w zadaniu 1, tylko musimy uwzględnić dodatkowe 0,5
$$\epsilon = 0,5*10^{-4},\quad \frac{b-a}{2^i}<\epsilon ,\quad [a,b]=[0,1]$$
Nie będę bez sensu przepisywał reszty :)
\subsubsection*{Zadanie 3}
Wszystko podobnie jak w poprzednich zadaniach, sprowadza się do wznaczenia warunków:
$$\epsilon = 10^{-4},\quad \frac{b-a}{2^i}<\epsilon ,\quad [a,b]=[1,2]$$
(przez przypadek są takie same jak w zadaniu 1)

\section{Regula falsi}
\subsection{Teoria}
Algorytm rozwiązywanie równań nieliniowych jednej zmiennej.
\newline
Ograniczenia: \newline
1. W przedziale [a, b] znajduje się dokładnie jeden pierwiastek. \newline
2. Na końcach przedziałów funkcja ma przeciwne znaki f(a)f(b)<0. \newline
3. Istnieje pierwsza i druga pochodna funkcji. \newline
4. Pierwsza i druga pochodna mają w tym przedziale stałe znaki \newline
\newline
Przy tych założeniach wykres może mieć jedną z czterech postaci: \newline
a) malejąca wypukła \newline
b) rosnąca wypukła \newline
c) malejąca wklęsła \newline
d) rozsnąca wklęsła \newline

\begin{figure}[H]
\centering
\includegraphics[width=120mm]{falsi2.jpg}
\end{figure} 

\subsection{Zadania}
\subsubsection*{Zadanie 1}
Rozpatrzmy zadanie, gdy $f'(x), f''(x) > 0$ dla $x \in [a, b]$
\begin{figure}[H]
\centering
\includegraphics[width=100mm]{falsi.jpg}
\end{figure} 

\subsubsection*{Zadanie 2}
Stosując metodę reguła falsi znaleźć z dokładnością do dwóch miejsc dziesiętnych pierwiastek równania
$$ x^3+x^2-3x-3=0$$
leżący w przedziale [1, 2]
\subsection{Rozwiązania}
\subsubsection*{Zadanie 1}
1. Przez punkty  A(a, f(a)) i B(b, f(b)) prowadzimy cięciwę o równaniu: $y-f(a) = \frac{f(b)-f(a)}{b-a}(x-a)$ \newline
Stąd $$x^{(1)} = a - \frac{f(a)}{f(b)-f(a)}(b-a)$$ 
\newline
2. Jeśli $f(x^{(1)}) = 0$ to $x^{(1)}$ jest szukanym pierwiastkiem. \newline
3. Jeśli $f(x^{(1)}) \neq 0$ to przez punkt C($x^{(1)}$, $f(x^{(1)})$) oraz  ten z punktów A i B, którego rzędna ma przeciwny znak niż  $f(x^{(1)})$ prowadzimy następną cięciwę. 

\subsubsection*{Zadanie 2}
1. Funkcja spełnia ograniczenia: \newline
- pierwiastki funkcji to $-1$, $-\sqrt[2]{3}$, $\sqrt[2]{3}$, funkcja ma dokładnie jeden pierwiastek w przedziale [1, 2] \newline
- wartości funkcji na końcach przedziałów f(1)= -4; f(2)=3 z czego wynika f(1)f(2)<0 \newline
- istnieje pierwsza i druga pochodna funkcji
$f'(x) = 3x^2+2x-3$ oraz $f''(x)=6x+2$ \newline
- pierwsza i druga pochodna mają na tym przedziale dodatnie znaki \newline

2. Wyznaczamy punkty A(1, -4), B(2, 3), $$x^{(1)} = 1 - \frac{-4}{3-(-4)}(2-1)$$
$x^{(1)} = \frac{11}{7}$; $f(x^{(1)})=-\frac{468}{343}$
\newline
\newline
3. Wyznaczamy punkt C($\frac{11}{7}$, $-\frac{468}{343}$). Prowadzimy między punktami C i B.
$$x^{(2)} = \frac{11}{7} - \frac{-\frac{468}{343}}{3-(-\frac{468}{343})}(2-\frac{11}{7})$$ \newline
$x^{(2)} =1.705$, $f(x^{(2)})=-0.251$
\newline \newline
4. Wyznaczamy punkt D(1.705, -0.251). Prowadzimy styczną przez punkt D i B.
$$x^{(3)} = 1.705 - \frac{-0.251}{3-(-0.251)}(2-1.705)$$ \newline
$x^{(3)} =1.728$, $f(x^{(3)})=-0.038$
\newline \newline
5. Wyznaczamy punkt E(1.728, -0.038). Prowadzimy styczną przez punkt E i B.
$$x^{(4)} = 1.728 - \frac{-0.038}{3-(-0.038)}(2-1.728)$$ \newline
$x^{(4)} =1.731$, $f(x^{(4)})=0$
\newline \newline
6. Znaleźliśmy pierwiastek w przedziale [1, 2], w zaokrągleniu do drugiego miejsca dziesiętnego $x^{(0)} =1.73$
\section{Reguła falsi - rekurencyjnie}
$$x^{(0)} = a$$
$$x^{(i+1)} = x^{(i)} - \frac{f(x^{(i)})}{f(b)-f(x^{(i)})}(b-x^{(i)})$$
i=0,1,2,... \newline
Ciąg jest rosnący i ograniczony, a więc zbieżny, jego granicą jest szukany pierwiastek $\xi$. W rozpatrywanym przypadku kolejne wyrazy ciągu są mniejsze od dokładnego pierwiastka $\xi$  oraz $f(x^{(i)}) < 0$ dla każdego i. 
\newline
Wadą metody jest stosunkowa wolna zbieżność. 


\section{Metoda siecznych}
\subsection{Teoria}
Polepszona reguła falsi.\newline
Metodę reguła falsi można polepszyć rezygnująć z żądania by funkcja f miała w punktach wytyczających następną cięciwę róźne znaki.\newline
Wzór:
$$x^{(i+1)} = x^{(i)} - \frac{f(x^{(i)})}{f(x^{(i)})-f(x^{(i-1)})}(x^{(i)}-x^{(i-1)})  \hspace{2cm} i=1,2,...$$ 
Wymagane są dwa punkty początkowe: \newline
$x^{(0)} = a$ \newline
$x^{(1)} = b$ \newline
Zbieżność metody jest znacznie szybsza niż metody falsi. Ale może się zdażyć, że nie będzie ona zbieżna np. gdy początkowe przybliżenia nie leżą dostatecznie blisko rozwiązania.

\section{Metoda Newtona-Raphsona}
\subsection{Teoria}
Metoda służy do wyznaczenie przybliżonej wartości pierwiastka funkcji. \newline
Przyjmujemy następujące założenia dla funkcji: \newline
1. W przedziale [a, b] znajduje się dokładnie jeden pierwiastek. \newline
2. Na końcach przedziałów funkcja ma przeciwne znaki f(a)f(b)<0. \newline
3. Istnieje pierwsza i druga pochodna funkcji. \newline
4. Pierwsza i druga pochodna mają w tym przedziale stałe znaki \newline
\newline
Załóżmy, że $\xi$ oznacza pierwiastek funkcji f. Funkcja ta w pewnym otoczeniu $U(\xi)$ jest dostatecznie wiele razy róźniczkowalna. Ze wzoru Taylora otrzymujemy rozwinięcie funkcji f wokół punktu $x^{(0)} \in U(\xi)$. Pomijając wyrazy rzędu $(\xi - x^{(0)})^p $ dla wszystkich p>2, otrzymujemy równanie:
$$ 0= f(x^{(0)}) + (\overline{\xi}-x^{(0)})f'(x^{(0)})$$ \newline
Stąd 
$$ \overline{\xi}= x^{(0)} + \frac{f(x^{(0)})}{f'(x^{(0)})}$$ \newline
Co prowadzi do 
$$ x^{(i+1)}= x^{(i)} - \frac{f(x^{(i)})}{f'(x^{(i)})}$$
\subsubsection*{Graficzna interpretacja}
1. W danym punkcie $x^{(0)}$ wykreślamy styczną do krzywej f(x) (tanges kąta nachylenia $\alpha$ tej stycznej jest równy $f'(x^{(0)})$ ). \newline
2. Punkt przecięcai stycznej z osią OX wyznacza $x^{(1)}$.

\subsection{Zadania}
Stosując metodę reguła Newtona znaleźć z dokładnością do dwóch miejsc dziesiętnych pierwiastek równania
$$ x^3+x^2-3x-3=0$$
przyjąć $x^{(0)}=2$

\subsection{Rozwiązania}
1. Funcja spełnia założenia (sprawdziliśmy to już w regule falsi - bo to ta sama funkcja i te same założeia). \newline
2. $$ x^{(1)}= 2 - \frac{3}{13}$$ \newline
$x^{(1)}= 1.77$; $f(x^{(1)})=0.37$\newline
3. $$ x^{(2)}= 1.77 - \frac{0.37}{9.94}$$ 
$x^{(2)}=1.73$;  $f(x^{(2)})=-0.02$\newline
4. W przybliżeniu pierwiastek wielomianu wynosi 1.73.

\section{Twierdzenia Sturma (Martyna)}
\subsection{Teoria}

\subsection{Zadania}

\subsection{Rozwiązania}

\section{Operacje numeryczne na macierzach (Michał)}
\subsection{Teoria}
W obliczeniach na komputerze, normalne metody liczenia wyznaczników nie są użycteczne. Zamiast tego stosujemy tzw. rozkład $LU$ macierzy, w którym rozkładamy macierz $A = L\cdot U$ stosując np. metodę eliminacji Gaussa.\\
Macierz $L$ to macierz dolnotrójkątna o jednostkowej głównej przekątnej (czyli $det(L)=1$)\\
Macierz $U$ to macierz górnotrójkątna.\\


\subsection{Zadania}

\subsection{Rozwiązania}

\newpage
\tableofcontents
\end{document}

