\documentclass[a4paper]{article}
\usepackage[polish]{babel}
\usepackage[utf8x]{inputenc}
\usepackage[T1]{fontenc}

\usepackage[a4paper,top=3cm,bottom=2cm,left=3cm,right=3cm,marginparwidth=1.75cm]{geometry}
\usepackage{amsmath}
\usepackage{graphicx}
\usepackage{minted}

\author{Michał Bień}
\title{Obliczanie pierwiastków metodą Newtona}
\date{}
\begin{document}
\maketitle
\section{Zastosowanie}
Funkcja \emph{inewtonroots} oblicza przedział będący przybliżoną wartością pierwiastka rzeczywistego wielomianu
$$ p(x) = a_nx^n + a_{n-1}x^{n-1} + \cdots + a_1x + a_0 $$
o współczynnikach rzeczywistych.
\section{Opis metody}
Przybliżona wartość pierwiastka wielomianu jest obliczana metodą iteracyjną Newtona, na podstawie wzoru: 
$$ x_{k+1} = x_k - \frac{p(x_k)}{p'(x_k)} \text{,\ \ \ \ }k = 0, 1, 2, \cdots $$
gdzie przybliżenie początkowe $x_0$ jest przedziałem zadanym z góry. Wartości $p(x_k)$ i $p'(x_k)$ wyznacza się korzystając ze schematu Hornera:
$$ p(x) = ((\cdots(a_nx + a_{n-1})x + \cdots)x + a_1)x + a_0 $$
Proces iteracyjny kończy się, gdy:
$$ \frac{|x_{k+1} - x_k|}{\max(|x_{k+1}|,|x_k|)} < \epsilon\text{,\ \ \ \ } x_{k+1}\neq0 \vee x_k\neq0 $$
gdzie $\epsilon$ oznacza zadaną z góry dokładność, lub gdy:
$$ x_{k+1} = x_k = 0 $$

Ze względu na implementację interwałową, część operacji matematycznych jest wykonywana w sposób niestandardowy:
\begin{itemize}
\item Porównywanie do zera polega na sprawdzaniu, czy wartości bezwzględne obydwu stron przedziału są mniejsze od zadanej dokładności ($10^{-16}$)
\item Sprawdzanie nierówności jest koniunkcją wartości logicznych nierówności lewych i prawych stron sprawdzanych przedziałów
\item Wykorzystywana jest funkcja iabs służąca do wyznaczania wartości bezwzględnej przedziału
\item Warunek stopu jest sprawdzany osobno dla lewych i prawych stron odpowiednich przedziałów i jest koniunkcją tych dwóch wartości logicznych
\item Pozostałe operacje matematyczne są wykonywane w sposób naturalny, oprogramowane przez przeciążone operatory w bibliotece Intervals.h
\end{itemize}

\section{Wywołanie funkcji}
inewtonroots(n, a, x, mit, eps)

\section{Dane}
\begin{itemize}
\item n -- stopień wielomianu
\item a -- tablica przedziałów będących współczynnikami wielomianu (indeks w tablicy odpowiada stopniowi)
\item x -- przedział będący początkowym przybliżeniem pierwiastka
\item mit -- maksymalna dozwolona liczba operacji w procesie
\item eps -- względna dokładność rozwiązania
\end{itemize}

\section{Wyniki}
Funkcja zwraca strukturę wynikową \emph{ires} o następujących polach wynikowych:
\begin{itemize}
\item val -- przedział, w którym zawiera się obliczona wartość pierwiastka
\item w -- przedział, w którym zawiera się wartość wielomianu dla obliczonego pierwiastka
\end{itemize}

\section{Inne parametry}
Struktura wynikowa \emph{ires} zawiera także pola stanu, opisujące przebieg działania algorytmu
\begin{itemize}
\item it -- liczba wykonanych iteracji algorytmu do zatrzymania
\item st -- stan algorytmu:
\begin{itemize}
\item 0 -- algorytm zakończył się poprawnie
\item 1 -- wprowadzono niepoprawne dane
\item 2 -- wykryto dzielenie przez przedział zawierający zero podczas działania algorytmu
\item 3 -- po maksymalnej liczbie iteracji algorytm zakończył się, nie osiągając zadanej dokładności
\end{itemize}
\end{itemize}

\section{Typy parametrów}
\begin{itemize}
\item \textbf{int} -- n, mit, it
\item ld -- eps
\item ild --  x, val, w
\item vector -- a
\end{itemize}

\section{Identyfikatory nielokalne}
\begin{itemize}
\item ld -- alias long double
\item ild -- alias Interval<long double> z biblioteki interval\_arithmetic
\item vector -- nazwa typu tablicowego o elementach typu long double
\end{itemize}
\pagebreak
\section{Kod}

\begin{minted}[bgcolor=white]{C++}
ires inewtonroots (int n, ild *a, ild x, int mit, ld eps) {
    ires to_return;
    to_return.st = 0;
    to_return.it = 0;

    if(n<1 || mit<1) {
        to_return.st = 1;
        return to_return;
    }

    ild w = a[n];
    for(int i=n-1; i>=0; i--)
        w = w * x + a[i];

    if(abs(w.a) > 1e-16 || abs(w.b) > 1e-16) {
        to_return.st = 3;
        do {
            to_return.it++;
            w = a[n];
            ild dw = w;

            for(int i = n-1; i>0; i--) {
                w = w * x + a[i];
                dw = dw * x + w;
            }
            w = w * x + a[0];

            if(dw.a < 0 && dw.b >0) {
                to_return.st = 2;
                return to_return;
            }

            ild xh = x;
            ild u = iabs(xh);
            x = x-(w/dw);
            ild v = iabs(x);

            if(v.a<u.a && v.b<u.b) v = u;
            if((abs(v.a) < 1e-16 && abs(v.b) < 1e-16) || 
            (abs(x.a-xh.a)/abs(v.a) <= eps && abs(x.b-xh.b)/abs(v.b) <= eps))
                to_return.st = 0;

        } while(to_return.it!=mit && to_return.st==3);
    }

    to_return.val = x;
    w = a[n];
    for(int i = n-1; i>=0; i--)
        w = w * x + a[i];
    to_return.w = w;

    return to_return;
}
\end{minted}

\section{Przykłady}
\begin{itemize}
\item Wielomian: 
$$ p(x) = x^3 - 3x^2 - x + 3 = (x-1)(x+1)(x-3) $$
Dane:\\
n=3,a = \{3, -1, -3, 1\}, x = 2, mit = 10, eps = 1e-16 \\
Wyniki: \\
ires.val.l = -1.00000000000000E0\\
ires.val.r = -1.00000000000000E0\\
ires.w.l = -0.00000000000000E-1\\
ires.w.r = 0.00000000000000E-1\\
ires.it = 2\\
ires.st = 0\\

\item Wielomian:
$$ p(x) = x^9 - 4x^7 + x^5 = x^5(x-\sqrt{2-\sqrt{3}})(x+\sqrt{2-\sqrt{3}})(x-\sqrt{2+\sqrt{3}})(x+\sqrt{2+\sqrt{3}))} $$
Dane: \\
n=9, a= \{0, 0, 0, 0, 0, 1, 0, -4, 0, 1\}, x=3, mit = 20, eps = 1e-12 \\
Wyniki: \\
ires.val.l = 1.9318516525781361E0\\
ires.val.r = 1.9318516525781370E0\\
ires.w.l = -1.6327471328539744E-13\\
ires.w.r = 1.63303886154400927E-13\\
ires.it = 11\\
ires.st = 0\\

\item Wielomian:
$$ p(x) = x^9 - 4x^7 + x^5 = x^5(x-\sqrt{2-\sqrt{3}})(x+\sqrt{2-\sqrt{3}})(x-\sqrt{2+\sqrt{3}})(x+\sqrt{2+\sqrt{3}))} $$
Dane: \\
n=9, a= \{0, 0, 0, 0, 0, 1, 0, -4, 0, 1\}, x=3, mit = 20, eps = 1e-16 \\
Wyniki: \\
ires.val.l = 1.9318516525777434E0\\
ires.val.r = 1.9318516525785298E0\\
ires.w.l = -1.6348970566103320E-10\\
ires.w.r = 1.6348973337525876E-10\\
ires.it = 20\\
ires.st =3\\

\item Wielomian:
$$ p(x) = x^9 - 4x^7 + x^5 = x^5(x-\sqrt{2-\sqrt{3}})(x+\sqrt{2-\sqrt{3}})(x-\sqrt{2+\sqrt{3}})(x+\sqrt{2+\sqrt{3}))} $$
Dane: \\
n=9, a= \{0, 0, 0, 0, 0, 1, 0, -4, 0, 1\}, x.l=3.0, x.r=3.01, mit = 20, eps = 1e-16 \\
Wyniki: \\
ires.val.l = 0 \\
ires.val.r = 0 \\
ires.w.l = 0 \\
ires.w.r = 0 \\
ires.it = 6 \\
ires.st = 2 \\

\item Wielomian:
$$ p(x) = 2 $$
Dane: \\
n = 0, a=\{2\}, x=-1, mit=10, eps=1e-16 \\
ires.val.l = 0 \\
ires.val.r = 0 \\
ires.w.l = 0 \\
ires.w.r = 0 \\
ires.it = 0 \\
ires.st = 1 \\
\end{itemize}


\end{document}

