\documentclass[polish,a4paper]{article}
\usepackage{amsmath}
\usepackage{amssymb,amsfonts,amsthm}
\usepackage[english,main=polish]{babel}
\usepackage{polski}
\usepackage[utf8]{inputenc}
\usepackage[T1]{fontenc}
\usepackage{graphicx}
\usepackage{geometry}
\usepackage{tikz}
\usepackage{circuitikz}
\usepackage{float}
\usepackage{etoolbox}
\usepackage{pgfplots}
\patchcmd{\thebibliography}{\section*}{\section}{}{}

\selectlanguage{polish}
\title{Lab4}
\newgeometry{tmargin=3cm, bmargin=3cm, lmargin=2cm, rmargin=2cm}

\newcommand{\PRzFieldDsc}[1]{\sffamily\bfseries\scriptsize #1}
\newcommand{\PRzFieldCnt}[1]{\textit{#1}}
\newcommand{\PRzHeading}[8]{
\begin{center}
\begin{tabular}{ p{0.32\textwidth} p{0.15\textwidth} p{0.15\textwidth} p{0.12\textwidth} p{0.12\textwidth} }

  &   &   &   &   \\
\hline
\multicolumn{5}{|c|}{}\\[-1ex]
\multicolumn{5}{|c|}{{\LARGE #1}}\\
\multicolumn{5}{|c|}{}\\[-1ex]

\hline
\multicolumn{1}{|l|}{\PRzFieldDsc{Kierunek}}	& \multicolumn{1}{|l|}{\PRzFieldDsc{Specjalność}}	& \multicolumn{1}{|l|}{\PRzFieldDsc{Rok studiów}}	& \multicolumn{2}{|l|}{\PRzFieldDsc{Symbol grupy lab.}} \\
\multicolumn{1}{|c|}{\PRzFieldCnt{#2}}		& \multicolumn{1}{|c|}{\PRzFieldCnt{#3}}		& \multicolumn{1}{|c|}{\PRzFieldCnt{#4}}		& \multicolumn{2}{|c|}{\PRzFieldCnt{#5}} \\

\hline
\multicolumn{4}{|l|}{\PRzFieldDsc{Temat Laboratorium}}		& \multicolumn{1}{|l|}{\PRzFieldDsc{Numer lab.}} \\
\multicolumn{4}{|c|}{\PRzFieldCnt{#6}}				& \multicolumn{1}{|c|}{\PRzFieldCnt{#7}} \\

\hline
\multicolumn{5}{|l|}{\PRzFieldDsc{Skład grupy ćwiczeniowej oraz numery indeksów}}\\
\multicolumn{5}{|c|}{\PRzFieldCnt{#8}}\\

\hline
\multicolumn{3}{|l|}{\PRzFieldDsc{Uwagi}}	& \multicolumn{2}{|l|}{\PRzFieldDsc{Ocena}} \\
\multicolumn{3}{|c|}{\PRzFieldCnt{\ }}		& \multicolumn{2}{|c|}{\PRzFieldCnt{\ }} \\

\hline
\end{tabular}
\end{center}
}
\pgfplotsset{compat=1.14}
\begin{document}
\PRzHeading{Laboratorium Podstaw Elektroniki}{Informatyka}{--}{I}{I2}{Tranzystor MOS}{5}{Martyna Maciejewska(132273), Michał Bień(132191), Ziemowit Sokołowski(132318)}{}

{\large Wszystkie odnotowane w zadaniu różnice między wartością znamionową elementu a jego wartością rzeczywistą wynikają z niedoskonałości elementu i mieszczą się w granicy tolerowanego błędu.}


\section{Charakterystyka bramkowa nMOS}
\subsection{Cel} 
Celem ćwiczenia jest wyznaczenie empirycznej zależności pomiędzy sygnałem sterującym a sterowanym dla tranzystora nMOS
\subsection{Schemat}

\begin{figure}[H]
\centering
\begin{circuitikz}[american voltages]
\draw[green]
(3,-0.5) to (4,-0.5)
(5.5,-0.5) to (6.5,-0.5)
(1,-2.5) to (7,-2.5)
(4,-0.5) to (6,-0.5)
(1,-0.5) to (1,1)
(3,1) to (3.25,1)
(3.75,1) to (4,1)
(7,-2.5) to (7,-1)
(6,1) to (7,1)
(7,1) to (7,0)
(3,-2.5) to (3,-3);

\draw[red]
(5,-0.5) -- (5,-1.5)
circle [radius = 12pt]node[circle,fill=white,minimum size=10pt]{$V$} 
(5,-1.5) -- (5,-2.5)
(1,1) -- (2,1)
circle [radius = 13pt]node[circle,fill=white,minimum size=10pt]{$mA$} 
(2,1) -- (3,1)
(4,1) to [R, l=$R_1 \ 1k$ , *-*, color=red] (6,1)
(2.75,-3) to (3.25,-3);
\node at (3,-3.25) {GND};

\draw[red]
(1,-2.5) -- (1,-1.5)
circle [radius = 10pt]node[circle,fill=white,minimum size=16pt]{}
(1,-1.5) -- (1,-0.5);
\node at (1,-2.5) [red]{o};
\node at (1,-0.5) [red]{o};
\draw[thick, red]
(1.5,-1.1) to (1.5,-1.2)
(1.5,-1.1) to (1.4,-1.1)
(1.3,-1.3) to (1.5,-1.1)
(0.8,-1.8) to (0.6,-2);
\draw[-latex][red] (1,-1.7) -- (1,-1.3);

\draw[red]
(3,-2.5) -- (3,-1.5)
circle [radius = 10pt]node[circle,fill=white,minimum size=16pt]{}
(3,-1.5) -- (3,-0.5);
\node at (3,-2.5) [red]{o};
\node at (3,-0.5) [red]{o};
\draw[thick, red]
(3.5,-1.1) to (3.5,-1.2)
(3.5,-1.1) to (3.4,-1.1)
(3.3,-1.3) to (3.5,-1.1)
(2.8,-1.8) to (2.6,-2);
\draw[-latex][red] (3,-1.7) -- (3,-1.3);

\draw[color=red]
(7,-0.25) node[nigfete] (nmos5) {}
(7,-0.75) to (7.3,-0.75)
(7.25,-0.25) to (7.35,-0.25)
(7.3,-0.25) to (7.3,0.25)
(7,0.25) to (7.3,0.25);
\draw[-latex][red] (7.3,-0.75) -- (7.3,-0.25);

\node at (8.4,0) [gray]{Q1 BS170};
\node at (7.2,0.8) [pink]{D};
\node at (5.95,-0.2) [pink]{G};
\node at (7.2,-1.8) [pink]{S};
\node at (3.5,0.8) [pink]{Id};
\node at (6.5,-1.5) [pink]{U GS};
\node at (0.5,-0.75) [pink]{5V};
\node at (2.5,-2.25) [pink]{0..5V};
\draw[-latex][pink] (6,-2.25) -- (6,-0.75);
\draw[-latex][pink] (3.25,1) -- (3.75,1);

\end{circuitikz}
\caption{Układ do badania charakterystyki bramkowej tranzystora nMOS}
\end{figure}

\subsection{Tabela pomiarów}
\begin{table}[H]
\centering
\begin{tabular}{|c|c|c|}
\hline
$U_{GS}$ napięcie ustawione & $I_{D}  $ & $U_{GS}$ napięcie zmierzone \\
\hline 
0.5V & 0.00008mA &0.673V \\
\hline
1V & 0.00009mA & 1.133V \\
\hline
1.5V & 0.00093mA & 1.6498V\\
\hline
1.8V & 0.10107mA & 1.9362V\\
\hline
2.0V & 0.8124mA & 2.1562V\\
\hline
2.1V & 1.7457mA & 2.2704V\\
\hline
2.2V & 4.031mA & 2.3372V\\
\hline
2.4V & 5.0607mA & 2.5321V\\
\hline
2.8V & 5.0848mA & 2.8821V\\
\hline
3.0V & 5.0813mA & 3.237V\\
\hline
3.5V & 5.098mA & 3.580V\\
\hline
4.1V & 5.0976mA & 4.129V\\
\hline
4.5V & 5.105mA & 4.512V\\
\hline
5V & 5.1052mA & 5.061V\\
\hline

\end{tabular}
\caption{Tabela pomiarów dla schematu 1.6}
\end{table}

\subsection{Wykres}


\begin{figure}[H]
\centering
\begin{tikzpicture}
\begin{axis}[
xlabel={$U_{GS} [V]$ },
ylabel={$I_{D} [mA]$},
xmin=0,xmax=6,
ymin=0,ymax=6,
legend pos=north west,
ymajorgrids=true,grid style=dashed
]

\addplot[color=red,mark=*]
coordinates {
(0.673, 0.00008)
(1.133, 0.00009)
(1.6498, 0.00093)
(1.9362, 0.10107)
(2.1562, 0.8124)
(2.2704, 1.7457)
(2.3372, 4.031)
(2.5321, 5.0607)
(2.8821, 5.0848)
(3.237, 5.0813)
(3.580, 5.098)
(4.129, 5.0976)
(4.512, 5.105)
(5.061, 5.1052)
};

\legend{$U_{th}=2.0V$}
\end{axis}
\end{tikzpicture}
\caption{Przebieg charakterystyki $I_{D}=f(U_{GS})$ dla rezystora nMOS; $U_{th}$ wartość odczytana z noty katalogowej tranzystora}
\end{figure}

\subsection{Wnioski}
Wartość napięcia bramki, przy której lawinowo wzrasta przepływ prądu wynosi  2.1V (znamionowo 2.0V). Aby tranzystor nMOS zaczął przewodzić prąd, napięcie $U_{GS}$ musi być co do wartości większe od napięcia progowego $U_{th}$ Napięcie progowe układu BS170 wynosi 2.0V\cite{BS170} Odnotowaliśmy doświadczalnie największy wzrost natężenia prądu, gdy napięcie $U_{GS}$ przekraczało 2.1V


\section{Charakterystyka bramkowa pMOS}
\subsection{Cel} 
Wykazanie dualizmu między tranzystorem nMOS i pMOS.
\subsection{Schemat}

\begin{figure}[H]
\centering
\begin{circuitikz}[american voltages]
\draw[green]
(2,-0.5) to (4,-0.5)
(5.5,-0.5) to (6.5,-0.5)
(0,-2.5) to (4.5,-2.5)
(4.5,-2.5) to (4.5,-4)
(6.5,-4) to (7,-4)
(4,-0.5) to (6,-0.5)
(0,-0.5) to (0,1)
(3,1) to (4,1)
(7,-2.5) to (7,-1)
(6,1) to (7,1)
(7,1) to (7,0)
(0,1) to (7,1)
(2,-2.5) to (2,-3);

\draw[red]
(4,-0.5) -- (4,-1.5)
circle [radius = 12pt]node[circle,fill=white,minimum size=10pt]{$V$} 
(4,-1.5) -- (4,-2.5)
(4.5,-4) -- (5.5,-4)
circle [radius = 13pt]node[circle,fill=white,minimum size=10pt]{$mA$} 
(5.5,-4) -- (6.5,-4)
(7,-4) to [R, l_=$R_7 \ 1k$ , *-*, color=red] (7,-2)
(1.75,-3) to (2.25,-3);
\node at (2,-3.25) {GND};

\draw[red]
(0,-2.5) -- (0,-1.5)
circle [radius = 10pt]node[circle,fill=white,minimum size=16pt]{}
(0,-1.5) -- (0,-0.5);
\node at (0,-2.5) [red]{o};
\node at (0,-0.5) [red]{o};
\draw[thick, red]
(0.5,-1.1) to (0.5,-1.2)
(0.5,-1.1) to (0.4,-1.1)
(0.3,-1.3) to (0.5,-1.1)
(-0.2,-1.8) to (-0.4,-2);
\draw[-latex][red] (0,-1.7) -- (0,-1.3);

\draw[red]
(2,-2.5) -- (2,-1.5)
circle [radius = 10pt]node[circle,fill=white,minimum size=16pt]{}
(2,-1.5) -- (2,-0.5);
\node at (2,-2.5) [red]{o};
\node at (2,-0.5) [red]{o};
\draw[thick, red]
(2.5,-1.1) to (2.5,-1.2)
(2.5,-1.1) to (2.4,-1.1)
(2.3,-1.3) to (2.5,-1.1)
(1.8,-1.8) to (1.6,-2);
\draw[-latex][red] (2,-1.7) -- (2,-1.3);

\draw[color=red]
(7,-0.75) node[pigfete] (pmos5) {}
(7,-1.25) to (7.3,-1.25)
(7.25,-0.75) to (7.35,-0.75)
(7.3,-0.75) to (7.3,-0.25)
(7,-0.25) to (7.3,-0.25);
\draw[-latex][red] (7.3,-1.25) -- (7.3,-0.75);

\node at (8,0) [gray]{Q4 BS250};
\node at (7.2,0.8) [pink]{S};
\node at (5.95,-0.2) [pink]{G};
\node at (7.2,-1.8) [pink]{D};
\node at (7.4,-3.8) [pink]{Id};
\node at (-1.5,-1) [pink]{Uss};
\node at (6,0.5) [pink]{U GS};
\node at (5.5,-1.5) [pink]{U1};
\node at (0.5,-0.75) [pink]{5V};
\node at (2.5,-2.25) [pink]{0..5V};
\draw[-latex][pink] (-1,-2) -- (-1,0.5);
\draw[-latex][pink] (6.75,0.75) -- (6.75,-0.25);
\draw[-latex][pink] (5,-2.25) -- (5,-0.75);

\end{circuitikz}
\caption{Układ do badania charakterystyki bramkowej tranzystora pMOS}
\end{figure}

\subsection{Tabela pomiarów}
\begin{table}[H]
\centering
\begin{tabular}{|c|c|c|c|}
\hline
$U_{1}$ napięcie ustawione & $I_{D}  $ & $U_{1}$ napięcie zmierzone & $U_{GS}=-(U_{SS}-U_{1})$\\
\hline 
0.5V & 5.0623mA &0.614V & -4.386V \\
\hline
1V & 5.0455mA & 1.130V &  -3.870V\\
\hline
1.4V & 4.967mA & 1.6075V &  -3.3925V\\
\hline
1.5V & 4.715mA & 1.644V &  -3.356V\\
\hline
1.6V & 2.5735mA & 1.736V &  -3.264V\\
\hline
1.7V & 1.189mA & 1.898V &  -3.102V\\
\hline
1.8V & 0.4357mA & 2.008V &  -2.992V\\
\hline
2.0V & 0.0406mA & 2.190V &  -2.810V\\
\hline
2.5V & 0.00023mA & 2.641V &  -2.359V\\
\hline
3.0V & 0.00008mA & 3.135V &  -1.865V\\
\hline
4.0V & 0,00006mA & 4.202V &  -0,795V\\
\hline
5.0V & 0,00006mA & 5.169V &  0,169V\\
\hline

\end{tabular}
\caption{Tabela pomiarów dla schematu 1.7}
\end{table}

\subsection{Wykres}


\begin{figure}[H]
\centering
\begin{tikzpicture}
\begin{axis}[
xlabel={$U_{GS} [V]$ },
ylabel={$I_{D} [mA]$},
xmin=-5,xmax=0.2,
ymin=0,ymax=6,
legend pos=north east,
ymajorgrids=true,grid style=dashed
]

\addplot[color=red,mark=*]
coordinates {
(-4.386, 5.0623)
(-3.870, 5.0455)
(-3.3925, 4.967)
(-3.356, 4.715)
(-3.264, 2.574)
(-3.102, 1.189)
(-2.992, 0.4357)
(-2.810, 0.0406)
(-2.359, 0.00023)
(-1.865, 0.00008)
(-0.795, 0.00006)
(0.169, 0.00006)
};

\legend{$U_{th} \approx -3V$}
\end{axis}
\end{tikzpicture}
\caption{Przebieg charakterystyki $I_{D}=f(U_{GS})$ dla rezystora pMOS; wartość $U_{th}$ zapisana w nocie katalogowej wynosi od -1 do -3.5V }
\end{figure}
\subsection{Wnioski}

Działanie tranzystora pMOS jest dualne względem tranzystora nMOS  wynika to wprost z odwrotnych własności polaryzowania złącz
półprzewodnikowych PN. 
Wartość napięcia bramki, przy której lawinowo spada przepływ prądu wynosi  -3.1V (znamionowo między -1V a -3.5V).


\section{Charakterystyka drenowa nMOS}
\subsection{Cel} 
Przeanalizować zależność pomiędzy wartością  prądu  Drenu
$I_{D}$ tranzystora a wartością napięcia  $U_{DS}$ przyłożonego pomiędzy Dren a Źródło. 
\subsection{Schemat}

\begin{figure}[H]
\centering
\begin{circuitikz}[american voltages]
\draw[green]
(3,-0.5) to (4,-0.5)
(5.5,-0.5) to (6.5,-0.5)
(1,-2.5) to (7,-2.5)
(4,-0.5) to (6,-0.5)
(1,-0.5) to (1,1)
(3.75,1) to (4,1)
(3,1) to (3.25,1)
(7,-2.5) to (7,-1)
(6,1) to (7,1)
(7,1) to (7,0)
(7,1) to (9,1)
(7,-2.5) to (9,-2.5)
(9,-0.5) to (9,1);

\draw[red]
(9,-0.5) -- (9,-1.5)
circle [radius = 12pt]node[circle,fill=white,minimum size=10pt]{$V$} 
(9,-1.5) -- (9,-2.5)
(1,1) -- (2,1)
circle [radius = 13pt]node[circle,fill=white,minimum size=10pt]{$mA$} 
(2,1) -- (3,1)
(4,1) to [R, l=$R_2 \ 1k$ , *-*, color=red] (6,1);

\draw[red]
(1,-2.5) -- (1,-1.5)
circle [radius = 10pt]node[circle,fill=white,minimum size=16pt]{}
(1,-1.5) -- (1,-0.5);
\node at (1,-2.5) [red]{o};
\node at (1,-0.5) [red]{o};
\draw[thick, red]
(1.5,-1.1) to (1.5,-1.2)
(1.5,-1.1) to (1.4,-1.1)
(1.3,-1.3) to (1.5,-1.1)
(0.8,-1.8) to (0.6,-2);
\draw[-latex][red] (1,-1.7) -- (1,-1.3);

\draw[red]
(3,-2.5) -- (3,-1.5)
circle [radius = 10pt]node[circle,fill=white,minimum size=16pt]{}
(3,-1.5) -- (3,-0.5);
\node at (3,-2.5) [red]{o};
\node at (3,-0.5) [red]{o};
\draw[thick, red]
(3.5,-1.1) to (3.5,-1.2)
(3.5,-1.1) to (3.4,-1.1)
(3.3,-1.3) to (3.5,-1.1)
(2.8,-1.8) to (2.6,-2);
\draw[-latex][red] (3,-1.7) -- (3,-1.3);

\draw[color=red]
(7,-0.25) node[nigfete] (nmos5) {}
(7,-0.75) to (7.3,-0.75)
(7.25,-0.25) to (7.35,-0.25)
(7.3,-0.25) to (7.3,0.25)
(7,0.25) to (7.3,0.25);
\draw[-latex][red] (7.3,-0.75) -- (7.3,-0.25);

\node at (5.75,0.4) [gray]{Q2 BS170};
\node at (7.2,0.8) [pink]{D};
\node at (5.95,-0.2) [pink]{G};
\node at (7.2,-1.8) [pink]{S};
\node at (3.5,0.75) [pink]{Id};
\node at (8.1,-1) [pink]{U DS};
\node at (1.75,-0.75) [pink]{0..10V};
\node at (2.5,-2.25) [pink]{5V};
\draw[-latex][pink] (7.5,-2) -- (7.5,0.5);
\draw[-latex][pink] (3.25,1) -- (3.75,1);

\end{circuitikz}
\caption{Układ do badania charakterystyki drenowej tranzystora nMOS}
\end{figure}

\begin{figure}[H]
\centering
\begin{circuitikz}[american voltages]
\draw[green]
(3,-0.5) to (3.5,-0.5)
(5.5,-0.5) to (6.5,-0.5)
(1,-2.5) to (7,-2.5)
(1,-0.5) to (1,1)
(3.75,1) to (4,1)
(3,1) to (3.25,1)
(7,-2.5) to (7,-1)
(6,1) to (7,1)
(7,1) to (7,0)
(7,1) to (9,1)
(7,-2.5) to (9,-2.5)
(9,-0.5) to (9,1);

\draw[red]
(9,-0.5) -- (9,-1.5)
circle [radius = 12pt]node[circle,fill=white,minimum size=10pt]{$V$} 
(9,-1.5) -- (9,-2.5)
(1,1) -- (2,1)
circle [radius = 13pt]node[circle,fill=white,minimum size=10pt]{$mA$} 
(2,1) -- (3,1)
(4,1) to [R, l=$R_3 \ 1k$ , *-*, color=red] (6,1)
(3.5,-0.5) to [R, l=$R_4 \ 1k$ , *-*, color=red] (5.5,-0.5)
(5.75,-0.5) to [R, l_=$R_5 \ 1k$ , *-*, color=red] (5.75,-2.5);

\draw[red]
(1,-2.5) -- (1,-1.5)
circle [radius = 10pt]node[circle,fill=white,minimum size=16pt]{}
(1,-1.5) -- (1,-0.5);
\node at (1,-2.5) [red]{o};
\node at (1,-0.5) [red]{o};
\draw[thick, red]
(1.5,-1.1) to (1.5,-1.2)
(1.5,-1.1) to (1.4,-1.1)
(1.3,-1.3) to (1.5,-1.1)
(0.8,-1.8) to (0.6,-2);
\draw[-latex][red] (1,-1.7) -- (1,-1.3);

\draw[red]
(3,-2.5) -- (3,-1.5)
circle [radius = 10pt]node[circle,fill=white,minimum size=16pt]{}
(3,-1.5) -- (3,-0.5);
\node at (3,-2.5) [red]{o};
\node at (3,-0.5) [red]{o};
\draw[thick, red]
(3.5,-1.1) to (3.5,-1.2)
(3.5,-1.1) to (3.4,-1.1)
(3.3,-1.3) to (3.5,-1.1)
(2.8,-1.8) to (2.6,-2);
\draw[-latex][red] (3,-1.7) -- (3,-1.3);

\draw[color=red]
(7,-0.25) node[nigfete] (nmos5) {}
(7,-0.75) to (7.3,-0.75)
(7.25,-0.25) to (7.35,-0.25)
(7.3,-0.25) to (7.3,0.25)
(7,0.25) to (7.3,0.25);
\draw[-latex][red] (7.3,-0.75) -- (7.3,-0.25);

\node at (5.75,0.4) [gray]{Q3 BS170};
\node at (7.2,0.8) [pink]{D};
\node at (5.95,-0.2) [pink]{G};
\node at (7.2,-1.8) [pink]{S};
\node at (3.5,0.75) [pink]{Id};
\node at (8.1,-1) [pink]{U DS};
\node at (1.75,-0.75) [pink]{0..10V};
\node at (2.5,-2.25) [pink]{5V};
\draw[-latex][pink] (7.5,-2) -- (7.5,0.5);
\draw[-latex][pink] (3.25,1) -- (3.75,1);

\end{circuitikz}
\caption{Układ do badania charakterystyki drenowej nMOS dla obnizonego napiecia bramki}
\end{figure}

\subsection{Tabela pomiarów}
Zmierzone napięcie $U_{GS}$ wyniosło 5.048V

\begin{table}[H]
\centering
\begin{tabular}{|c|c|c|}
\hline
napięcie ustawione & $I_{D}  $ & $U_{DS}$ \\
\hline 
1V & 1.1302mA & 3.829mV \\
\hline
2V & 2.1841mA & 6.735mV \\
\hline
3V & 3.2588mA & 10.101mV\\
\hline
4V & 4.239mA & 13.103mV\\
\hline
5V & 5.2423mA & 15.987mV\\
\hline
6V & 6.303mA & 18.960mV\\
\hline
7V & 7.2873mA & 22.055mV\\
\hline
8V & 8.2957mA & 24.851mV\\
\hline
9V & 9.3264mA & 28.136mV\\
\hline
10V & 10.3095mA & 30.954mV\\
\hline


\end{tabular}
\caption{Tabela pomiarów dla schematu 1.8}
\end{table}

Zmierzone napięcie $U_{GS}$ wyniosło 5.0505V
\begin{table}[H]
\centering
\begin{tabular}{|c|c|c|}
\hline
napięcie ustawione & $I_{D}  $ & $U_{DS}$ \\
\hline 
1V & 1.12161mA & 2.665mV \\
\hline
2.1V & 2.3247mA & 5.325mV \\
\hline
3V & 3.1972mA & 7.382mV\\
\hline
4V & 4.2174mA & 9.771mV\\
\hline
5.1V & 5.3074mA & 12.015mV\\
\hline
6V & 6.2512mA & 14.263mV\\
\hline
6.9V & 7.2092mA & 16.833mV\\
\hline
8V & 8.306mA & 19.016mV\\
\hline
9V & 9.2912mA & 21.401mV\\
\hline
10V & 10.2999mA & 23.907mV\\
\hline


\end{tabular}
\caption{Tabela pomiarów dla schematu 1.9}
\end{table}

\subsection{Wykres}

\begin{figure}[H]
\centering
\begin{tikzpicture}
\begin{axis}[
xlabel={$U_{DS} [mV]$ },
ylabel={$I_{D} [mA]$},
xmin=0,xmax=31,
ymin=0,ymax=11,
legend pos=north west,
ymajorgrids=true,grid style=dashed
]

\addplot[color=red,mark=*]
coordinates {
(3.829, 1.1302)
(6.735, 2.1841)
(10.101, 3.2588)
(13.103, 4.239)
(15.987, 5.2423)
(18.96, 6.303)
(22.055, 7.2873)
(24.851, 8.2957)
(28.136, 9.3264)
(30.954, 10.3264)
};

\addplot[color=blue,mark=*]
coordinates {
(2.665, 1.12161)
(5.325, 2.3247)
(7.382, 3.1972)
(9.771, 4.2174)
(12.015, 5.3074)
(14.263, 6.2512)
(16.833, 7.2092)
(19.016, 8.306)
(21.401, 9.2912)
(23.907, 10.2999)
};

\legend{$U_{GS}=5.048V$,$U_{GS}=5.0505V$}
\end{axis}
\end{tikzpicture}
\caption{Przebieg charakterystyki $I_{D}=f(U_{DS})$ dla rezystora nMOS}
\end{figure}

\subsection{Wnioski}
W zależności od napięcia Bramki $U_{GS}$ zmienia kształt wykresu, dla większej wartości $U_{GS}$ wykres "sięga wyżej", dla tego samego $U_{DS}$ osiąga większą wartość $I_{D}$

\section{Charakterystyka drenowa pMOS}
\subsection{Cel} 
Konfrontacja charakterystyki drenowej tranzystora pMOS z charakterystyką drenową tranzystorem nMOS zmierzoną w poprzednim punkcie.
\subsection{Schemat}

\begin{figure}[H]
\centering
\begin{circuitikz}[american voltages]
\draw[green]
(3,-0.5) to (4,-0.5)
(5.5,-0.5) to (6.5,-0.5)
(1,-2.5) to (4,-2.5)
(6,-2.5) to (6,-4)
(6,-4) to (7,-4)
(4,-0.5) to (6,-0.5)
(1,-0.5) to (1,1)
(1,1) to (9,1)
(7,-2.5) to (7,-1)
(6,1) to (7,1)
(7,1) to (7,0)
(7,1) to (9,1)
(7,-4) to (9,-4)
(9,-4) to (9,-2)
(9,0) to (9,1)
(3,-2.5) to (3,-3);

\draw[red]
(9,0) -- (9,-1)
circle [radius = 12pt]node[circle,fill=white,minimum size=10pt]{$V$} 
(9,-1) -- (9,-2)
(4,-2.5) -- (5,-2.5)
circle [radius = 13pt]node[circle,fill=white,minimum size=10pt]{$mA$} 
(5,-2.5) -- (6,-2.5)
(7,-2) to [R, l=$R_6 \ 1k$ , *-*, color=red] (7,-4)
(2.75,-3) to (3.25,-3);
\node at (3,-3.25) {GND};

\draw[red]
(1,-2.5) -- (1,-1.5)
circle [radius = 10pt]node[circle,fill=white,minimum size=16pt]{}
(1,-1.5) -- (1,-0.5);
\node at (1,-2.5) [red]{o};
\node at (1,-0.5) [red]{o};
\draw[thick, red]
(1.5,-1.1) to (1.5,-1.2)
(1.5,-1.1) to (1.4,-1.1)
(1.3,-1.3) to (1.5,-1.1)
(0.8,-1.8) to (0.6,-2);
\draw[-latex][red] (1,-1.7) -- (1,-1.3);

\draw[red]
(3,-2.5) -- (3,-1.5)
circle [radius = 10pt]node[circle,fill=white,minimum size=16pt]{}
(3,-1.5) -- (3,-0.5);
\node at (3,-2.5) [red]{o};
\node at (3,-0.5) [red]{o};
\draw[thick, red]
(3.5,-1.1) to (3.5,-1.2)
(3.5,-1.1) to (3.4,-1.1)
(3.3,-1.3) to (3.5,-1.1)
(2.8,-1.8) to (2.6,-2);
\draw[-latex][red] (3,-1.7) -- (3,-1.3);

\draw[color=red]
(7,-0.75) node[pigfete] (pmos5) {}
(7,-1.25) to (7.3,-1.25)
(7.25,-0.75) to (7.35,-0.75)
(7.3,-0.75) to (7.3,-0.25)
(7,-0.25) to (7.3,-0.25);
\draw[-latex][red] (7.3,-1.25) -- (7.3,-0.75);

\node at (6,0.25) [gray]{Q5 BS250};
\node at (7.2,0.8) [pink]{S};
\node at (5.75,-0.3) [pink]{G};
\node at (7.2,-1.8) [pink]{D};
\node at (7.4,-3.8) [pink]{Id};
\node at (-0.5,-1) [pink]{Usd};
\node at (1.75,-0.75) [pink]{0..10V};
\node at (2.5,-2.25) [pink]{5V};
\draw[-latex][pink] (0,-2.5) -- (0,1);

\end{circuitikz}
\caption{Układ do badania charakterystyki drenowej tranzystora pMOS}
\end{figure}

\begin{figure}[H]
\centering
\begin{circuitikz}[american voltages]
\draw[green]
(1,-0.5) to (1.5,-0.5)
(3.5,-0.5) to (4,-0.5)
(5.5,-0.5) to (6.5,-0.5)
(-1,-2.5) to (4,-2.5)
(6,-2.5) to (6,-4)
(6,-4) to (7,-4)
(4,-0.5) to (6,-0.5)
(-1,-0.5) to (-1,1)
(-1,1) to (9,1)
(7,-2.5) to (7,-1)
(6,1) to (7,1)
(7,1) to (7,0)
(7,1) to (9,1)
(7,-4) to (9,-4)
(9,-4) to (9,-2)
(9,0) to (9,1)
(1,-2.5) to (1,-3);

\draw[red]
(9,0) -- (9,-1)
circle [radius = 12pt]node[circle,fill=white,minimum size=10pt]{$V$} 
(9,-1) -- (9,-2)
(4,-2.5) -- (5,-2.5)
circle [radius = 13pt]node[circle,fill=white,minimum size=10pt]{$mA$} 
(5,-2.5) -- (6,-2.5)
(7,-2) to [R, l=$R_8 \ 1k$ , *-*, color=red] (7,-4)
(1.5,-0.5) to [R, l=$R_9 \ 1k$ , *-*, color=red] (3.5,-0.5)
(3.75,-0.5) to [R, l=$R_{10} \ 1k$ , *-*, color=red] (3.75,-2.5)
(0.75,-3) to (1.25,-3);
\node at (1,-3.25) {GND};

\draw[red]
(-1,-2.5) -- (-1,-1.5)
circle [radius = 10pt]node[circle,fill=white,minimum size=16pt]{}
(-1,-1.5) -- (-1,-0.5);
\node at (-1,-2.5) [red]{o};
\node at (-1,-0.5) [red]{o};
\draw[thick, red]
(-0.5,-1.1) to (-0.5,-1.2)
(-0.5,-1.1) to (-0.6,-1.1)
(-0.7,-1.3) to (-0.5,-1.1)
(-1.2,-1.8) to (-1.4,-2);
\draw[-latex][red] (-1,-1.7) -- (-1,-1.3);

\draw[red]
(1,-2.5) -- (1,-1.5)
circle [radius = 10pt]node[circle,fill=white,minimum size=16pt]{}
(1,-1.5) -- (1,-0.5);
\node at (1,-2.5) [red]{o};
\node at (1,-0.5) [red]{o};
\draw[thick, red]
(1.5,-1.1) to (1.5,-1.2)
(1.5,-1.1) to (1.4,-1.1)
(1.3,-1.3) to (1.5,-1.1)
(0.8,-1.8) to (0.6,-2);
\draw[-latex][red] (1,-1.7) -- (1,-1.3);

\draw[color=red]
(7,-0.75) node[pigfete] (pmos5) {}
(7,-1.25) to (7.3,-1.25)
(7.25,-0.75) to (7.35,-0.75)
(7.3,-0.75) to (7.3,-0.25)
(7,-0.25) to (7.3,-0.25);
\draw[-latex][red] (7.3,-1.25) -- (7.3,-0.75);

\node at (6,0.25) [gray]{Q5 BS250};
\node at (7.2,0.8) [pink]{S};
\node at (5.75,-0.3) [pink]{G};
\node at (7.2,-1.8) [pink]{D};
\node at (7.4,-3.8) [pink]{Id};
\node at (-2.5,-1) [pink]{Usd};
\node at (-0.25,-0.75) [pink]{0..10V};
\node at (0.5,-2.25) [pink]{5V};
\draw[-latex][pink] (-2,-2.5) -- (-2,1);

\end{circuitikz}
\caption{Układ do badania charakterystyki drenowej dla obnizonego napiecia bramki pMOS}
\end{figure}
\subsection{Tabela pomiarów}
Zmierzone napięcie $U_{GS}$ wyniosło -5.049V

\begin{table}[H]
\centering
\begin{tabular}{|c|c|c|}
\hline
$U_{SD}$ napięcie ustawione & $I_{D}  $ & $U_{DS}$ napięcie zmierzone \\
\hline 
1V & 0.00001mA & -1.202V \\
\hline
2V & 0.00008mA & -2.271V \\
\hline
3.1V & 0.00008mA & -3.420V\\
\hline
4.1V & 0.00008mA & -4.172V\\
\hline
5V & 0.00008mA & -5.249V\\
\hline
6V & 0.00008mA & -6.207V\\
\hline
7V & 0.00011mA & -7.144V\\
\hline
8.1V & 1.803mA & -8.170V\\
\hline
9V & 9.132mA & -9.140V\\
\hline
10V & 10.4223mA & -10.103V\\
\hline


\end{tabular}
\caption{Tabela pomiarów dla schematu 1.10}
\end{table}

Zmierzone napięcie $U_{GS}$ wyniosło 2.539V

\begin{table}[H]
\centering
\begin{tabular}{|c|c|c|}
\hline
$U_{SD}$ napięcie ustawione & $I_{D}  $ & $U_{DS}$ napięcie zmierzone \\
\hline 
1V & 0.513mA & -1.241V \\
\hline
2V & 1.562mA & -2.184V \\
\hline
3V & 2.5127mA & -3.161V\\
\hline
4V & 3.674mA & -4.189V\\
\hline
5V & 5.060mA & -5.208V\\
\hline
6V & 6.148mA & -6.194V\\
\hline
7V & 7.162mA & -7.158V\\
\hline
8V & 8.233mA & -8.168V\\
\hline
9V & 9.231mA & -9.226V\\
\hline
10V & 10.330mA & -10.186V\\
\hline


\end{tabular}
\caption{Tabela pomiarów dla schematu 1.11}
\end{table}

\subsection{Wykres}

\begin{figure}[H]
\centering
\begin{tikzpicture}
\begin{axis}[
xlabel={$U_{DS} [V]$ },
ylabel={$I_{D} [mA]$},
xmin=-10.2,xmax=0,
ymin=0,ymax=10.5,
legend pos=north east,
ymajorgrids=true,grid style=dashed
]

\addplot[color=red,mark=*]
coordinates {
(-1.202, 0.00001)
(-2.271, 0.00008)
(-3.420, 0.00008)
(-4.172, 0.00008)
(-5.249, 0.00008)
(-6.207, 0.00008)
(-7.144, 0.00011)
(-8.170, 1.803)
(-9.140, 9.132)
(-10.103, 10.4223)
};

\addplot[color=blue,mark=*]
coordinates {
(-1.241, 0.513)
(-2.184, 1.562)
(-3.161, 2.5127)
(-4.189, 3.674)
(-5.208, 5.060)
(-6.194, 6.148)
(-7.158, 7.162)
(-8.168, 8.233)
(-9.226, 9.231)
(-10.186, 10.330)
};

\legend{$U_{GS}=-5.049V$,$U_{GS}=2.539V$}
\end{axis}
\end{tikzpicture}
\caption{Przebieg charakterystyki $I_{D}=f(U_{DS})$ dla rezystora pMOS}
\end{figure}

\subsection{Wnioski}
Nieprawidłowe pomiary mogą wynkać z uszkodzonego tranzystora.
Tranzystor pMOS zachowuje się dualnie względem tranzystora nMOS. Na powyższym wykresie zobrazowane są dwa tryby jego pracy: liniowy, oraz odcięcia. Tryb pracy tranzystora zależny jest od jego napięcia zasilania.

\section{Tranzystor nMOS jako przełącznik}
\subsection{Cel} 
Unaocznienie możliwości wykorzystania tranzystora nMOS jako przełącznika napięcia
\subsection{Schematy}

\begin{figure}[H]
\centering
\begin{circuitikz}[american voltages]
\draw[green]
(6.5,1) to (6.5,0)
(-1.5,1) to(-1.5,-0.25) 
(0,-1.25) to (0,-1)
(-1.5,1) to (6.5,1)
(-1.5,-1.25) to (0,-1.25)
(-0.5,-3.25) to (1,-3.25)
(1,-3.25) to (1,-2.5)
(6.5,0) to (6.5,-0.5)
(1,-1.5) to (1,-2.5)
(1,0) to (1.5,0)
(3,0) to (4,0)
(5.5,0) to (6.5,0)
(1,-2.5) to (6.5,-2.5)
(3.5,0) to [empty diode, color=red] (1.5,0);

\draw[red]
(4,0) to [R, l_=$R_{11} \ 1k$ , *-*, color=red] (6,0)
(-0.5,-1.25) to [R, l_=$R_{12} \ 1M$ , *-*, color=red] (-0.5,-3.25);

\draw[red]
(6.5,-2.5) -- (6.5,-1.5)
circle [radius = 10pt]node[circle,fill=white,minimum size=16pt]{}
(6.5,-1.5) -- (6.5,-0.5);
\node at (6.5,-2.5) [red]{o};
\node at (6.5,-0.5) [red]{o};
\draw[thick, red]
(7,-1.1) to (7,-1.2)
(7,-1.1) to (6.9,-1.1)
(6.8,-1.3) to (7,-1.1)
(6.3,-1.8) to (6.1,-2);
\draw[-latex][red] (6.5,-1.7) -- (6.5,-1.3);

\draw[-latex][red] (2.25,0.5) -- (2,0.75);
\draw[-latex][red] (2.45,0.5) -- (2.2,0.75);'

\draw[color=red]
(1,-0.75) node[nigfete] (nmos5) {}
(1,-1.25) to (1.3,-1.25)
(1.25,-0.75) to (1.35,-0.75)
(1.3,-0.75) to (1.3,-0.25)
(1,-0.25) to (1.3,-0.25);
\draw[-latex][red] (1.3,-1.25) -- (1.3,-0.75);

\node at (-1.5,-0.25) [green]{*};
\node at (-1.5,-1.25) [green]{*};

\node at (-1.5,-0.75) [pink]{A-A};
\node at (7.25,-2) [pink]{10V};
\node at (2.5,-0.75) [gray]{LED 1};
\node at (-0.25,0) [gray]{Q7 BS170};

\end{circuitikz}
\caption{Schemat układu do badania tranzystora nMOS w roli przełacznika}
\end{figure}

\begin{figure}[H]
\centering
\begin{circuitikz}[american voltages]
\draw[green]
(6.5,1) to (6.5,0)
(-3,1) to(-3,-0.25) 
(0,-1.25) to (0,-1)
(-3,1) to (6.5,1)
(-3,-1.25) to (0,-1.25)
(-2,-3.25) to (1,-3.25)
(1,-3.25) to (1,-2.5)
(6.5,0) to (6.5,-0.5)
(1,-1.5) to (1,-2.5)
(1,0) to (1.5,0)
(3,0) to (4,0)
(5.5,0) to (6.5,0)
(1,-2.5) to (6.5,-2.5)
(3.5,0) to [empty diode, color=red] (1.5,0);

\draw[red]
(4,0) to [R, l_=$R_{13} \ 1k$ , *-*, color=red] (6,0)
(-0.75,-1.25) to [R, l=$R_{14} \ 47k$ , *-*, color=red] (-0.75,-3.25)
(-2,-1.25) to [C, l_=$C_1 \ 100uF$, *-*, color=red] (-2,-3.25);

\draw[red]
(6.5,-2.5) -- (6.5,-1.5)
circle [radius = 10pt]node[circle,fill=white,minimum size=16pt]{}
(6.5,-1.5) -- (6.5,-0.5);
\node at (6.5,-2.5) [red]{o};
\node at (6.5,-0.5) [red]{o};
\draw[thick, red]
(7,-1.1) to (7,-1.2)
(7,-1.1) to (6.9,-1.1)
(6.8,-1.3) to (7,-1.1)
(6.3,-1.8) to (6.1,-2);
\draw[-latex][red] (6.5,-1.7) -- (6.5,-1.3);

\draw[-latex][red] (2.25,0.5) -- (2,0.75);
\draw[-latex][red] (2.45,0.5) -- (2.2,0.75);'

\draw[color=red]
(1,-0.75) node[nigfete] (nmos5) {}
(1,-1.25) to (1.3,-1.25)
(1.25,-0.75) to (1.35,-0.75)
(1.3,-0.75) to (1.3,-0.25)
(1,-0.25) to (1.3,-0.25);
\draw[-latex][red] (1.3,-1.25) -- (1.3,-0.75);

\node at (-3,-0.25) [green]{*};
\node at (-3,-1.25) [green]{*};

\node at (-3,-0.75) [pink]{A-A};
\node at (7.25,-2) [pink]{10V};
\node at (2.5,-0.75) [gray]{LED 2};
\node at (-0.75,-0.25) [gray]{Q8 BS170};

\end{circuitikz}
\caption{Model układu z opóznieniem wyłaczenia}
\end{figure}

\subsection{Wnioski}
Układy działały zgodnie z przewidywaniami, o czym poświadczyć mogą poniższe zdjęcia dokumentujące funkcjonowanie układu przełącznika z opóźnieniem.
\begin{figure}[H]
\centering
\includegraphics[width=100mm]{opozn1.jpg}
\caption{Układ przed zwarciem obwodu A-A: dioda nie świeci}
\end{figure}
\begin{figure}[H]
\centering
\includegraphics[width=100mm]{opozn2.jpg}
\caption{Układ w momencie zwarcia obwodu A-A: dioda świeci mocno}
\end{figure}
\begin{figure}[H]
\centering
\includegraphics[width=100mm]{opozn3.jpg}
\caption{Układ chwilę po ponownym rozwarciu obwodu A-A: dioda świeci coraz słabiej, aż w końcu gaśnie}
\end{figure}


\section{Czas załączania tranzystora}
\subsection{Cel} 
Poznawanie zjawisk związanych z opóźnieniami powodowanymi czasem załączania tranzystora. Zbadanie i wyznaczenie tego czasu.
\subsection{Schemat}
\begin{figure}[H]
\centering
\begin{circuitikz}[american voltages]
\draw[green]
(-4,-1.25) to (-4, 3.5)
(-4, 3.5) to (-1, 3.5)
(-1, 3) to (-2,3)
(-2,2.5) to (-2,3)
(4.5, 3) to (3.5,3)
(3.5,2.5) to (3.5,3)
(-4,-3.25) to (-4,-3.75);

\draw[red]
(-2.25,2.5) to (-1.75,2.5)
(3.25,2.5) to (3.75,2.5)
(-3.75,-3.75) to (-4.25,-3.75);
\node at (-2,2.25) {GND};
\node at (3.5,2.25) {GND};
\node at (-4,-4) {GND};

\draw[red]
(-1, 3.5) to (-0.5,3.5)  
(-1,3) to (-0.5,3)
(-0.5,4) to (-0.5,2.5)
(1.5,4) to (1.5,2.5)
(-0.5,2.5) to (1.5,2.5)
(-0.5,4) to (1.5,4)
(-0.25,3.75) to (-0.25,2.75)
(1,3.75) to (1,2.75)
(-0.25,2.75) to (1,2.75)
(-0.25,3.75) to (1,3.75);
\node at (-1, 3.5) [red]{o};
\node at (-1,3) [red]{o};
\node at (1.25,2.75) [red]{o};
\node at (1.25, 3.7) [red]{o};
\node at (1.25, 3.3) [red]{o};

\draw[red]
(4.5,3.5) to (5,3.5) 
(4.5,3) to (5,3)
(5,4) to (5,2.5)
(7,4) to (7,2.5)
(5,2.5) to (7,2.5)
(5,4) to (7,4)
(5.25,3.75) to (5.25,2.75)
(6.5,3.75) to (6.5,2.75)
(5.25,2.75) to (6.5,2.75)
(5.25,3.75) to (6.5,3.75);
\node at (4.5,3.5) [red]{o};
\node at (4.5,3) [red]{o};
\node at (6.75,2.75) [red]{o};
\node at (6.75,3.7) [red]{o};
\node at (6.75,3.3) [red]{o};

\draw[red]
(-4,-3.25) -- (-4,-2.25)
circle [radius = 10pt]node[circle,fill=white,minimum size=8pt]{}
(-4,-2.25) -- (-4,-1.25)
(-5.15,-2.2) to (-5.05,-2.2)
(-5.05,-2.2) to (-5.05,-2.3)
(-5.05,-2.3) to (-4.95,-2.3)
(-4.95,-2.3) to (-4.95,-2.2)
(-4.95,-2.2) to (-4.85,-2.2)
(-4.85,-2.2) to (-4.85,-2.3)
(-4.85,-2.3) to (-4.75,-2.3);
\node at (-7,-1.3) [gray]{*z niewiadoych przyczyn};
\node at (-7,-1.6) [gray]{symbol ten nie może być};
\node at (-7,-1.9) [gray]{umieszczony w środku okręgu};
\node at (-4,-3.25) [red]{o};
\node at (-4,-1.25) [red]{o};

\draw[thick, red]
(-3.5,-1.85) to (-3.5,-1.95)
(-3.5,-1.85) to (-3.6,-1.85)
(-3.7,-2.05) to (-3.5,-1.85)
(-4.2,-2.55) to (-4.4,-2.75);

\draw[green]
(-4,-1.25) to (-1,-1.25)
(-4,-3.25) to (-1,-3.25)
(1,0) to (1,1.5)
(1,1.5) to (2,1.5)
(2,1.5) to (2,3.5)
(2,3.5) to (4.5,3.5)
(0,-1.25) to (0,-1)
(-3,-1.25) to (0,-1.25)
(-2,-3.25) to (1,-3.25)
(1,-3.25) to (1,-2.5)
(6.5,0) to (6.5,-0.5)
(1,-1.5) to (1,-2.5)
(1,0) to (1.5,0)
(3,0) to (4,0)
(5.5,0) to (6.5,0)
(1,-2.5) to (6.5,-2.5)
(3.5,0) to [empty diode, color=red] (1.5,0);

\draw[red]
(4,0) to [R, l_=$R_{15} \ 1k$ , *-*, color=red] (6,0)
(-0.5,-1.25) to [R, l_=$R_{16} \ 1M$ , *-*, color=red] (-0.5,-3.25);

\draw[red]
(6.5,-2.5) -- (6.5,-1.5)
circle [radius = 10pt]node[circle,fill=white,minimum size=16pt]{}
(6.5,-1.5) -- (6.5,-0.5);
\node at (6.5,-2.5) [red]{o};
\node at (6.5,-0.5) [red]{o};
\draw[thick, red]
(7,-1.1) to (7,-1.2)
(7,-1.1) to (6.9,-1.1)
(6.8,-1.3) to (7,-1.1)
(6.3,-1.8) to (6.1,-2);
\draw[-latex][red] (6.5,-1.7) -- (6.5,-1.3);

\draw[-latex][red] (2.25,0.5) -- (2,0.75);
\draw[-latex][red] (2.45,0.5) -- (2.2,0.75);'

\draw[color=red]
(1,-0.75) node[nigfete] (nmos5) {}
(1,-1.25) to (1.3,-1.25)
(1.25,-0.75) to (1.35,-0.75)
(1.3,-0.75) to (1.3,-0.25)
(1,-0.25) to (1.3,-0.25);
\draw[-latex][red] (1.3,-1.25) -- (1.3,-0.75);

\node at (4,3.25) [pink]{masa};
\node at (-3.25,-3.5) [pink]{masa};
\node at (-1.5,3.25) [pink]{masa};
\node at (-1.5,3.75) [pink]{syg.};
\node at (4,3.75) [pink]{syg.};
\node at (-3.25,-0.95) [pink]{syg.};
\node at (2.5,3.75) [pink]{BNC};
\node at (-4.5,0) [pink]{BNC};
\node at (-3,-2.5) [pink]{BNC};
\node at (7.25,-2) [pink]{10V};
\node at (2.5,-0.75) [gray]{LED 3};
\node at (-0.75,-0.25) [gray]{Q9 BS170};

\end{circuitikz}
\caption{Obwód do pomiaru czasu przełączenia}
\end{figure}

\subsection{Wnioski}
W pierwszej części zadania sprawdziliśmy, jak zmiana czasu załączenia tranzystora wpływa na sposób świecenia diody. Rzeczywiście, zgodnie ze wzorem $P = \frac{U^2}{R}\sigma$, moc w układzie rośnie proporcjonalnie do wypełnienia przebiegu ($\sigma$), co można zaobserwować na poniższych zdjęciach:
\begin{figure}[H]
\centering
\includegraphics[width=100mm]{zal1.jpg}
\caption{Oscylogram: małe wypełnienie}
\end{figure}
\begin{figure}[H]
\centering
\includegraphics[width=100mm]{zal2.jpg}
\caption{Dioda świeci słabo przy małym wypełnieniu}
\end{figure}
\begin{figure}[H]
\centering
\includegraphics[width=100mm]{zal3.jpg}
\caption{Oscylogram: duże wypełnienie}
\end{figure}
\begin{figure}[H]
\centering
\includegraphics[width=100mm]{zal4.jpg}
\caption{Światło diody jest silne przy dużym wypełnieniu}
\end{figure}

Druga część doświadczenia polegała na wyznaczeniu czasu przełączania tranzystora przy wykorzystaniu oscyloskopu. Należało także wyznaczyć maksymalną stabilną częstotliwość pracy układu tranzystora i diody LED.
Wpierw wykonaliśmy oscylogram z przebiegu wyłączania tranzystora:
\begin{figure}[H]
\centering
\includegraphics[width=100mm]{opoz.jpg}
\caption{Oscylogram z przebiegu wyłączania tranzystora}
\end{figure}
Na wykresie widać wyraźnie przesunięcie między momentem załączenia wysokiego sygnału, a momentem wyłączenie tranzystora. Okno to wynosi w sumie ok. 55ns. Tak więc, korzystając ze wzoru\cite{naum} możemy stwierdzić, że maksymalna stabilna częstotliwość pracy układu wynosi:
$$f_{max} = \frac{1}{t_d} = \frac{1}{55*10^{-9}} \approx 18.181MHz$$


\begin{thebibliography}{3}
\bibitem{naum} http://mariusznaumowicz.ddns.net/materialy.html - materiały dostępne dla przedmiotu Podstawy Elektroniki
\bibitem{BS170} https://www.onsemi.com/pub/Collateral/BS170-D.PDF - dokumentacja układu BS170
\bibitem{pto}Podstawy teorii obwodów, Osiowski J., Szabatin J., WNT, Warszawa, 1998
\bibitem{pe}"Podstawy Elektrotechniki", R.Kurdziel, wyd II, WNT Warszawa 1972
\end{thebibliography}

\newpage
\tableofcontents{}
\end{document}
