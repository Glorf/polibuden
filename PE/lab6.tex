\documentclass[polish,a4paper]{article}
\usepackage{amsmath}
\usepackage{amssymb,amsfonts,amsthm}
\usepackage[english,main=polish]{babel}
\usepackage{polski}
\usepackage[utf8]{inputenc}
\usepackage[T1]{fontenc}
\usepackage{graphicx}
\usepackage{geometry}
\usepackage{tikz}
\usepackage{circuitikz}
\usepackage{float}
\usepackage{etoolbox}
\usepackage{pgfplots}
\patchcmd{\thebibliography}{\section*}{\section}{}{}

\selectlanguage{polish}
\title{Lab6}
\newgeometry{tmargin=3cm, bmargin=3cm, lmargin=2cm, rmargin=2cm}

\newcommand{\PRzFieldDsc}[1]{\sffamily\bfseries\scriptsize #1}
\newcommand{\PRzFieldCnt}[1]{\textit{#1}}
\newcommand{\PRzHeading}[8]{
\begin{center}
\begin{tabular}{ p{0.32\textwidth} p{0.15\textwidth} p{0.15\textwidth} p{0.12\textwidth} p{0.12\textwidth} }

  &   &   &   &   \\
\hline
\multicolumn{5}{|c|}{}\\[-1ex]
\multicolumn{5}{|c|}{{\LARGE #1}}\\
\multicolumn{5}{|c|}{}\\[-1ex]

\hline
\multicolumn{1}{|l|}{\PRzFieldDsc{Kierunek}}	& \multicolumn{1}{|l|}{\PRzFieldDsc{Specjalność}}	& \multicolumn{1}{|l|}{\PRzFieldDsc{Rok studiów}}	& \multicolumn{2}{|l|}{\PRzFieldDsc{Symbol grupy lab.}} \\
\multicolumn{1}{|c|}{\PRzFieldCnt{#2}}		& \multicolumn{1}{|c|}{\PRzFieldCnt{#3}}		& \multicolumn{1}{|c|}{\PRzFieldCnt{#4}}		& \multicolumn{2}{|c|}{\PRzFieldCnt{#5}} \\

\hline
\multicolumn{4}{|l|}{\PRzFieldDsc{Temat Laboratorium}}		& \multicolumn{1}{|l|}{\PRzFieldDsc{Numer lab.}} \\
\multicolumn{4}{|c|}{\PRzFieldCnt{#6}}				& \multicolumn{1}{|c|}{\PRzFieldCnt{#7}} \\

\hline
\multicolumn{5}{|l|}{\PRzFieldDsc{Skład grupy ćwiczeniowej oraz numery indeksów}}\\
\multicolumn{5}{|c|}{\PRzFieldCnt{#8}}\\

\hline
\multicolumn{3}{|l|}{\PRzFieldDsc{}}	& \multicolumn{2}{|l|}{\PRzFieldDsc{Ocena}} \\
\multicolumn{3}{|c|}{\PRzFieldCnt{Dawid Połomka(112765), Paweł Myszkowski(132292) }}		& \multicolumn{2}{|c|}{\PRzFieldCnt{\ }} \\

\hline
\end{tabular}
\end{center}
}
\pgfplotsset{compat=1.14}
\begin{document}
\PRzHeading{Laboratorium Podstaw Elektroniki}{Informatyka}{--}{I}{I2}{Wzmacniacze operacyjne}{6}{Martyna Maciejewska(132273), Michał Bień(132191), Ziemowit Sokołowski(132318),Kacper Maciejewski(132274)}{}

{\large Wszystkie odnotowane w zadaniu różnice między wartością znamionową elementu a jego wartością rzeczywistą wynikają z niedoskonałości elementu i mieszczą się w granicy tolerowanego błędu.}

\section{Konfiguracja nieodwracająca}

\subsection{Cel}
Zbadanie bezwzględnego wzmocnienia napięciowego wzmacniacza operacyjnego w konfiguracji nieodwracającej.

\subsection{Ćwiczenie}

Użyte elementy rezystancyjne: 
Rezystor R1 i R2 oznaczone kolorami brązowy, czarny, czerwony, złoty o rezystancji $1000\Omega$. Zmierzona rezystancja R1 wyniosła $983.8\Omega$, a R2 wyniosła $988.4\Omega$.
Na wyjściu generatora ustawiliśmy przebieg sinusoidalny o częstotliwości 4 kHz. Jako wartość międzyszczytową przebiegu obraliśmy $V_{pp}=1V$.
\newline
\newline
Amplitudy przebiegów wejściowych i wyjściowych:\newline
$V_{amp} = 1.02V$ \hspace{2cm} (1) wejściowy \newline
$V_{amp} = 2.04V$ \hspace{2cm} (2) wyjściowy \newline

Szacujemy wzmocnienie wzmacniacza a skali liniowej i decybelowej. 

\begin{center}
$k_u[\frac{V}{V}]=\frac{U_{wy}}{U_{we}}$ \hspace{2cm}  $k_u[\frac{V}{V}]=\frac{2.04}{1.02}=2$\\
$k_u[dB]=20log(\frac{U_{wy}}{U_{we}})$ \hspace{1.5cm} $k_u[dB]=20log(\frac{2.04}{1.02})=6.021$\\
\end{center}

Obliczamy wzmocnienie wzmacniacza korzystając z zależności $k_u=1+ \frac{Z_{f}}{Z_{in}}$

\begin{center}

$1+ \frac{Z_{f}}{Z_{in}}=1 + \frac{1000\Omega}{1000\Omega}=2$ \\
$k_u=\frac{U_{wy}}{U_{we}}=1+ \frac{Z_{f}}{Z_{in}}$
\end{center}


\begin{figure}[H]
\centering
\includegraphics[width=100mm]{nieodwr.jpg}
\caption{Przebiegi dla konfiguracji nieodwracającej}
\end{figure}

\subsection{Wnioski}

Powyższa  zależność  pokazuje,  że  wzmacniacz  operacyjny  w  konfiguracji  nieodwracającej  osiąga bezwzględne wzmocnienie napięciowe, które będzie zawsze co najmniej równe 1$\frac{V}{V}$
W teori wzmocnienie dla układu tzw. wtórnika napięciowego wynosi 1.
Taki układ charakteryzuje się bardzo dużą rezystancją wejściową i małą rezystancją wyjściową. Dzięki temu używany jest często jako prosty układ separujący, ponieważ jego dołączenie nie obciąża układu badanego. 


\section{Konfiguracja odwracająca}
\subsection{Cel}
Zbadanie bezwzględnego wzmocnienia napięciowego wzmacniacza operacyjnego w konfiguracji odwracającej.

\subsection{Ćwiczenie}
Na wyjściu generatora ustawiliśmy przebieg sinusoidalny o częstotliwości 4 kHz. Jako wartość międzyszczytową przebiegu obraliśmy $V_{pp}=1V$.

Użyte elementy rezystancyjne: 

$Z_{in}$ :
Rezystor oznaczony kolorami brązowy, czarny, czerwony, złoty o rezystancji $1000\Omega$. Zmierzona rezystancja wyniosła $983.8\Omega$. Rezystor oznaczony kolorami czerwony, czarny, czerwony, złoty o rezystancji $2000\Omega$. Zmierzona rezystancja R1 wyniosła $1993.2\Omega$. 

$Z_{f}$ :
Rezystor oznaczony kolorami brązowy, czarny, czerwony, złoty o rezystancji $1000\Omega$. Zmierzona rezystancja wyniosła $988.5\Omega$. Rezystor oznaczony kolorami czerwony, czarny, czerwony, złoty o rezystancji $2000\Omega$. Zmierzona rezystancja R1 wyniosła $1982.2\Omega$.Rezystor oznaczony kolorami zielony, czarny, czerwony, złoty o rezystancji $5000\Omega$. Zmierzona rezystancja R1 wyniosła $5082.2\Omega$.

\begin{center}
$k_U[\frac{V}{V}]=\frac{U_{wy}}{U_{we}}$\\
$k_U[dB]=20log(\frac{U_{wy}}{U_{we}})$
\end{center}
\begin{table}[H]
\centering
\begin{tabular}{|c|c|c|c||c|c|c|c|c|}
\hline
$Z_{in}$ &nr przełącznika & $Z_{f}$ & nr przełącznika & $k_{u}$ teoretyczne & $U_{we}$ & $U_{wy}$ & $k_{u}$ $[\frac{V}{V}]$ & $k_{u}$ $[dB]$\\
\hline 
$1k\Omega$ & 1 & $2k\Omega$ & 1 & 2 & $1.01V$ & $1.96V$ & 1.94 & 2.88  \\
\hline
$1k\Omega$ & 1 & $1k\Omega$ & 2 & 1 & $1.01V$ & $1.01V$ & 1 & 0  \\
\hline
$1k\Omega$ & 1 & $5k\Omega$ & 3 & 5 & $1.01V$ & $5.16V$ & 5.11 & 7.08 \\
\hline
$2k\Omega$ & 2 & $1k\Omega$ & 2 & 0.5 & $1.04V$ & $0.485V$ & 0.46 & -3.37 \\
\hline
\end{tabular}
\caption{Tabela pomiarów}
\end{table}

Różnice między $k_u$ teoretycznym a otrzymanym wynikają z faktu obrania wartości napięcia wyjściowego nieznacznie różniących się od zakładanych, a także z niedoskonałości elementów układu, mieszczących się w dopuszczalnej granicy błędu.

\begin{figure}[H]
\centering
\includegraphics[width=100mm]{odwr1.jpg}
\caption{Przebieg 1 dla konfiguracji odwracającej}
\end{figure}

\begin{figure}[H]
\centering
\includegraphics[width=100mm]{odwr2.jpg}
\caption{Przebieg 2 dla konfiguracji odwracającej}
\end{figure}

\begin{figure}[H]
\centering
\includegraphics[width=100mm]{odwr3.jpg}
\caption{Przebieg 3 dla konfiguracji odwracającej}
\end{figure}

\begin{figure}[H]
\centering
\includegraphics[width=100mm]{odwr4.jpg}
\caption{Przebieg 4 dla konfiguracji odwracającej}
\end{figure}

\subsection{Wnioski}
Przesunięcie fazowe jest to różnica pomiędzy fazami dwóch fal. W tym przypadku wynosi ona 180 stopni, co świadczy o tym, że fale są w przeciwfazie. Fakt ten jest wynikiem tego, że układ pracuje w konfiguracji odwracającej.

\section{Blok integratora}

\subsection{Cel}
Analiza sytuacji, gdy w roli impedancji $Z_{in}$ występuje rezystor R, a w roli $Z_{f}$ kondensator C.

\subsection{Ćwiczenie}
Na wyjściu generatora ustawiliśmy przebieg sinusoidalny o częstotliwości 4 kHz. Jako wartość międzyszczytową przebiegu obraliśmy $V_{pp}=1V$.

Użyte elementy rezystancyjne i pojemnościowe:

Rezystor oznaczony kolorami brązowy, czarny, czerwony, złoty o rezystancji $1000\Omega$. Zmierzona rezystancja wyniosła $983.8\Omega$. Rezystor oznaczony kolorami czerwony, czarny, czerwony, złoty o rezystancji $2000\Omega$. Zmierzona rezystancja R1 wyniosła $1993.2\Omega$. 
Kondensator o pojemności 10nF. Zmierzona pojemność wyniosła 9.9nF.

$$U_{out}(t) = -\frac{1}{RC} \int{u_{in}(t)dt} = -\frac{1}{T_i} \int{u_{in}(t)dt}$$
$$\frac{1}{RC} = \frac{1}{T_i}$$

Zmierzone wartości:
\begin{enumerate}
\item $V_{amp}=7.12V$ i fall-time=$113\mu$s
\item $V_{amp}=3.84V$ i fall-time=$117\mu$s
\end{enumerate}
\begin{table}[H]
\centering
\begin{tabular}{|c|c|c|c||c|c|}
\hline
R &nr przełącznika & C & nr przełącznika & $\frac{1}{T_{i}}$ teoretyczne &  $\frac{1}{T_{i}}$ obliczone\\
\hline 
$1k\Omega$ & 1 & 10nF & 4 & $10^5$  &  $1.023 * 10^5$ \\
\hline
$2k\Omega$ & 2 & 10nF & 4 & $5 * 10^4$  &   $4.897 * 10^4$\\
\hline
\end{tabular}
\caption{Tabela pomiarów}
\end{table}

\begin{figure}[H]
\centering
\includegraphics[width=100mm]{calk1.jpg}
\caption{Przebieg 1 dla układu całkującego}
\end{figure}

\begin{figure}[H]
\centering
\includegraphics[width=100mm]{calk2.jpg}
\caption{Przebieg 2 dla układu całkującego}
\end{figure}

\subsection{Wnioski}
Układ staje się integratorem czasu ciągłego o stałej całkowania
$T_{i}$ uzależnionej od iloczynu elementów R, C.
Wzmocnienie integratora zależy od częstotliwości sygnału.

\section{Blok różniczkujący}
\subsection{Cel}
Analiza sytuacji, gdy w roli impedancji$Z_{in}$ występuje pojemność C, a w roli $Z_{f}$ rezystor R.

\subsection{Ćwiczenie}
Na wyjściu generatora ustawiliśmy przebieg sinusoidalny o częstotliwości 1 kHz. Jako wartość międzyszczytową przebiegu obraliśmy $V_{pp}=1V$.

pojemność $C_{1} = 100nF$ i rezystor $R_{6} = 5k\Omega$ to gorsze dobranie danych ($T_d = 5 * 10^{-4}$)
\begin{figure}[H]
\centering
\includegraphics[width=100mm]{rozngorszy.jpg}
\caption{Przebieg różniczkujący dla gorzej dobranych parametrów pracy}
\end{figure}
pojemność $C_{1} = 100nF$ i rezystor $R_{7} = 1k\Omega$ to lepsze dobranie danych ($T_d = 10^{-4}$)
\begin{figure}[H]
\centering
\includegraphics[width=100mm]{roznlepszy.jpg}
\caption{Przebieg różniczkujący dla lepiej dobranych parametrów pracy}
\end{figure}

Zniekształcenia przebiegu wyjściowego w pobliżu przełączeń poziomów wejściowego przebiegu prostokątnego tłumaczyć można czasem reakcji procesu ładowania i rozładowywania kondensatora w układzie na zmianę tendencji prądowej.

\begin{thebibliography}{3}
\bibitem{naum} http://mariusznaumowicz.ddns.net/materialy.html - materiały dostępne dla przedmiotu Podstawy Elektroniki
\bibitem{BS170} https://www.onsemi.com/pub/Collateral/BS170-D.PDF - dokumentacja układu BS170
\bibitem{pto}Podstawy teorii obwodów, Osiowski J., Szabatin J., WNT, Warszawa, 1998
\bibitem{pe}"Podstawy Elektrotechniki", R.Kurdziel, wyd II, WNT Warszawa 1972
\end{thebibliography}

\newpage
\tableofcontents{}
\end{document}

