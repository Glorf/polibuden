\documentclass[polish,a4paper]{article}
\usepackage{amsmath}
\usepackage{amssymb,amsfonts,amsthm}
\usepackage[english,main=polish]{babel}
\usepackage{polski}
\usepackage[utf8]{inputenc}
\usepackage[T1]{fontenc}
\usepackage{graphicx}
\usepackage{geometry}
\usepackage{tikz}
\usepackage{circuitikz}
\usepackage{float}
\usepackage{etoolbox}
\usepackage{pgfplots}
\patchcmd{\thebibliography}{\section*}{\section}{}{}

\selectlanguage{polish}
\title{Lab4}
\newgeometry{tmargin=3cm, bmargin=3cm, lmargin=2cm, rmargin=2cm}

\newcommand{\PRzFieldDsc}[1]{\sffamily\bfseries\scriptsize #1}
\newcommand{\PRzFieldCnt}[1]{\textit{#1}}
\newcommand{\PRzHeading}[8]{
\begin{center}
\begin{tabular}{ p{0.32\textwidth} p{0.15\textwidth} p{0.15\textwidth} p{0.12\textwidth} p{0.12\textwidth} }

  &   &   &   &   \\
\hline
\multicolumn{5}{|c|}{}\\[-1ex]
\multicolumn{5}{|c|}{{\LARGE #1}}\\
\multicolumn{5}{|c|}{}\\[-1ex]

\hline
\multicolumn{1}{|l|}{\PRzFieldDsc{Kierunek}}	& \multicolumn{1}{|l|}{\PRzFieldDsc{Specjalność}}	& \multicolumn{1}{|l|}{\PRzFieldDsc{Rok studiów}}	& \multicolumn{2}{|l|}{\PRzFieldDsc{Symbol grupy lab.}} \\
\multicolumn{1}{|c|}{\PRzFieldCnt{#2}}		& \multicolumn{1}{|c|}{\PRzFieldCnt{#3}}		& \multicolumn{1}{|c|}{\PRzFieldCnt{#4}}		& \multicolumn{2}{|c|}{\PRzFieldCnt{#5}} \\

\hline
\multicolumn{4}{|l|}{\PRzFieldDsc{Temat Laboratorium}}		& \multicolumn{1}{|l|}{\PRzFieldDsc{Numer lab.}} \\
\multicolumn{4}{|c|}{\PRzFieldCnt{#6}}				& \multicolumn{1}{|c|}{\PRzFieldCnt{#7}} \\

\hline
\multicolumn{5}{|l|}{\PRzFieldDsc{Skład grupy ćwiczeniowej oraz numery indeksów}}\\
\multicolumn{5}{|c|}{\PRzFieldCnt{#8}}\\

\hline
\multicolumn{3}{|l|}{\PRzFieldDsc{Uwagi}}	& \multicolumn{2}{|l|}{\PRzFieldDsc{Ocena}} \\
\multicolumn{3}{|c|}{\PRzFieldCnt{\ }}		& \multicolumn{2}{|c|}{\PRzFieldCnt{\ }} \\

\hline
\end{tabular}
\end{center}
}
\pgfplotsset{compat=1.14}
\begin{document}
\PRzHeading{Laboratorium Podstaw Elektroniki}{Informatyka}{--}{I}{I2}{Diody}{4}{Martyna Maciejewska(132273), Michał Bień(132191), Ziemowit Sokołowski(132318)}{}

{\large Wszystkie odnotowane w zadaniu różnice między wartością znamionową elementu a jego wartością rzeczywistą wynikają z niedoskonałości elementu i mieszczą się w granicy tolerowanego błędu.}

\section{Charakterystyka stałoprądowa dla diody złączowej}
\subsection{Cel} 
\begin{quotation} Niniejsze ćwiczenie ma na celu praktyczną weryfikację zasady działania półprzewodnikowego złącza PN. W pierwszej części ćwiczenia złącze jest podłączone do obwodu w sposób pozwalający na przepływ prądu (w tzw. kierunku przewodzenia), następnie połączenie diody zostaje odwrócone i bada się jej zachowanie w stanie braku przewodzenia (w przy tzw. polaryzacji zaporowej).\cite{naum}
\end{quotation}
\subsection{Dioda podłączona w kierunku przewodzenia}
Użyte elementy rezystancyjne: 
Rezystor oznaczony kolorami brązowy, czarny, czerwony, złoty o rezystancji $1000\Omega$. Zmierzona rezystancja wyniosła $983.8\Omega$. 
\begin{figure}[H]
\centering
\begin{circuitikz}[american voltages]
\draw[green]
(1,-0.5) to (2,-0.5)
(5.5,-0.5) to (6.5,-0.5)
(1,-2.5) to (6.5,-2.5)
(2,-0.5) to [empty diode, color=red] (4,-0.5)
(4,-0.5) to (6.5,-0.5);

\draw[red]
(6.5,-0.5) -- (6.5,-1.5)
circle [radius = 12pt]node[circle,fill=white,minimum size=10pt]{$V$} 
(6.5,-1.5) -- (6.5,-2.5)
(5,-0.5) to [R, l_=$R_1 \ 1k$ , *-*, color=red] (5,-2.5);;

\draw[red]
(1,-2.5) -- (1,-1.5)
circle [radius = 10pt]node[circle,fill=white,minimum size=16pt]{}
(1,-1.5) -- (1,-0.5);
\node at (1,-2.5) [red]{o};
\node at (1,-0.5) [red]{o};
\draw[thick, red]
(1.5,-1.1) to (1.5,-1.2)
(1.5,-1.1) to (1.4,-1.1)
(1.3,-1.3) to (1.5,-1.1)
(0.8,-1.8) to (0.6,-2);

\node at (2.2,0.2) [gray]{1N4007};
\node at (1.8,-2) [gray]{$U_z$};
\node at (3.1,-2) [gray]{$U_{R2}$};
\node at (3,-1.3) [gray]{$U_d$};

\draw[-latex][red] (1,-1.7) -- (1,-1.3);
\draw[-latex][gray] (2,-2) -- (2,-1);
\draw[-latex][gray] (3.5,-2) -- (3.5,-1);
\draw[-latex][gray] (3.5,-1) -- (2,-1);

\end{circuitikz}
\caption{Schemat układu doświadczalnego}
\end{figure}

\begin{table}[H]
\centering
\begin{tabular}{|c|c||c|c|}
\hline
$U_{z}$ & $U_{R}$ & $U_{d}=U_{z}-U_{R}$ & $I_{d}=\frac{U_{R}}{R}$\\
\hline 
0.3V & 48.81mV & 0.25119V & 0.04881mA\\
\hline
0.5V & 164.63mV& 0.33537V & 0.016463mA\\
\hline
0.6V & 250.44mV & 0.34956V & 0.25044mA\\
\hline
0.7V & 333.01mV & 0.36699V & 0.33301mA\\
\hline
0.8V & 394.8mV & 0.4052V & 0.3948mA\\
\hline
0.9V & 0.5012V & 0.3988V & 0.5012mA\\
\hline
1V & 0.6544V & 0.3456V & 0.6544mA\\
\hline
1.5V & 1.048V & 0.452V & 1.048mA\\
\hline
2V & 1.544V & 0.456V & 1.544mA\\
\hline
2.5V & 2.061V & 0.439V & 2.061mA \\
\hline
3V & 2.514V & 0.486V & 2.514mA\\
\hline
3.5V & 3.041V & 0.459V & 3.041mA\\
\hline
4V & 3.563V & 0.437V & 3.563mA\\
\hline
4.5V & 4.063V & 0.437V & 4.063mA\\
\hline
5V & 4.526V & 0.474V & 4.526mA\\
\hline
\end{tabular}
\caption{Tabela pomiarów}
\end{table}

\subsection{Dioda podłączona przeciwnie do kierunku przewodzenia}
Użyte elementy rezystancyjne:
Rezystor oznaczony kolorami pomarańczowy, czarny, zielony, złoty o rezystancji $3M\Omega$. Zmierzona rezystancja wyniosła $3.002M\Omega$.

\begin{figure}[H]
\centering
\begin{circuitikz}[american voltages]
\draw[green]
(1,-0.5) to (2,-0.5)
(5.5,-0.5) to (6.5,-0.5)
(1,-2.5) to (6.5,-2.5)
(4,-0.5) to [empty diode, color=red] (2,-0.5)
(4,-0.5) to (6.5,-0.5);

\draw[red]
(6.5,-0.5) -- (6.5,-1.5)
circle [radius = 12pt]node[circle,fill=white,minimum size=10pt]{$V$} 
(6.5,-1.5) -- (6.5,-2.5)
(5,-0.5) to [R, l_=$R_1 \ 3M$ , *-*, color=red] (5,-2.5);;

\draw[red]
(1,-2.5) -- (1,-1.5)
circle [radius = 10pt]node[circle,fill=white,minimum size=16pt]{}
(1,-1.5) -- (1,-0.5);
\node at (1,-2.5) [red]{o};
\node at (1,-0.5) [red]{o};
\draw[thick, red]
(1.5,-1.1) to (1.5,-1.2)
(1.5,-1.1) to (1.4,-1.1)
(1.3,-1.3) to (1.5,-1.1)
(0.8,-1.8) to (0.6,-2);

\node at (2.2,0.2) [gray]{1N4007};
\node at (1.8,-2) [gray]{$U_z$};
\node at (3.1,-2) [gray]{$U_{R2}$};
\node at (3,-1.3) [gray]{$U_d$};

\draw[-latex][red] (1,-1.7) -- (1,-1.3);
\draw[-latex][gray] (2,-2) -- (2,-1);
\draw[-latex][gray] (3.5,-2) -- (3.5,-1);
\draw[-latex][gray] (3.5,-1) -- (2,-1);

\end{circuitikz}
\caption{Schemat układu doświadczalnego}
\end{figure}

\begin{table}[H]
\centering
\begin{tabular}{|c|c||c|c|}
\hline
$U_{z}$ & $U_{R}$ & $U_{d}=U_{z}-U_{R}$ & $I_{d}=\frac{U_{R}}{R}$\\
\hline 
5V & 3.703mV & 4.996297V & 1.2343nA \\
\hline
10V & 4.356mV & 9.995644V & 1.452nA\\
\hline
15V & 4.847mV & 14.995153V & 1.6157nA\\
\hline
20V & 5.257mV & 19.994743V & 1.7523nA\\
\hline
\end{tabular}
\caption{Tabela pomiarów}
\end{table}

\subsection{Wnioski}
	Z doświadczenia łatwo wysnuć można wnioski, które dobrze obrazuje poniższy wykres (Rysunek 3): dioda przewodzi prąd tylko w jednym kierunku (który jest jej kierunkiem przewodzenia). Spadki napięcia na diodzie w kierunku przewodzenia są małe, ale nie pomijalne. Natężenie prądu rośnie proporcjonalnie do wzrostu napięcia.
	W kierunku przeciwnym do kierunku przewodzenia diody prąd nie płynie: spadek napięcia na tak ustawionej diodzie jest niemal całkowity (rzędu $99.97\%$), płynący w obwodzie prąd jest pomijalnie mały i jego natężenie nie rośnie proporcjonalnie do wzrostu napięcia.


\begin{figure}[H]
\centering
\begin{tikzpicture}
\begin{axis}[
xlabel={$U_{d} [V]$ },
ylabel={$I_{d} [mA]$},
xmin=0,xmax=15,
ymin=0,ymax=4.6,
legend pos=north east,
ymajorgrids=true,grid style=dashed
]

\addplot[color=red,mark=*]
coordinates {
(0.25119, 0.04881)
(0.33537, 0.016463)
(0.34956, 0.25044)
(0.36699, 0.33301)
(0.4052, 0.3948)
(0.3988, 0.5012)
(0.3456, 0.6544)
(0.452, 1.048)
(0.456, 1.544)
(0.439, 2.061)
(0.486, 2.514)
(0.459, 3.041)
(0.437, 3.563)
(0.437, 4.063)
(0.474, 4.526)
};

\addplot[color=blue,mark=square]
coordinates {
(4.996297, 0.0000012343)
(9.995644, 0.000001452)
(14.995153, 0.0000016157)
(19.994743, 0.0000017523)
};

\legend{dioda w kierunku przewodzenia,dioda w kierunku zaporowym}
\end{axis}
\end{tikzpicture}
\caption{Przebieg charakterystyki $I_{d}=f(U_{d})$}
\end{figure}
\newpage

\section{Badanie prostownika jednopołówkowego}
\subsection{Cel}
\begin{quotation}
Zastosowanie podstawowej własności złącza PN - przepuszczania prądu płynącego tylko w jednym wyróżnionym kierunku przyczyniło się do rozwoju układów prostownikowych. Zadaniem prostownika jest zamiana przemiennego napięcia wejściowego w napięcie o ustalonym w czasie kierunku przepływu. Wejściowy przebieg prądowy rejestrowany przez amperomierz jest przemienny w czasie:  prąd wejściowy płynie okresami raz w jednym, raz w drugim kierunku względem gałęzi obwodu. Prąd wyjściowy prostownika dzięki zasadzie działania diody złączowej popłynie tylko w jednym kierunku, sprawiając, że wskazania amperomierza A będą  zawsze dodatnie. Opcjonalnie, do układu prostownika można dołożyć pojemność C (tzw. pojemność filtrująca), która będzie doładowywała się w momencie przewodzenia diody D, i podtrzymywała przepływ prądu wyjściowego, gdy dioda D przestanie przewodzić. Podczas bieżącego ćwiczenia zbadany zostanie układ prostownika jednopołówkowego z filtracją napięcia.\cite{naum}
\end{quotation}
\subsection{Badanie własności prostownika jednopołówkowego}
Rezystor oznaczony kolorami brązowy, czarny, czerwony, złoty o rezystancji $10k\Omega$. Zmierzona rezystancja wyniosła $9.875k\Omega$.

\begin{figure}[H]
\centering
\begin{circuitikz}[american voltages]
\draw[green]
(1,-0.5) to (1, 1.5)
(1, 1.5) to (3, 1.5)
(3, 1) to (2,1)
(2,1) to (2,0.5)
(1,-2.5) to (5.5,-2.5)
(5.5,-2.5) to (5.5,-2)
(1,-0.5) to[empty diode, color=red] (3,-0.5)
(3,-0.5) to (5.5,-0.5)
(5.5,-0.5) to (5.5,-1.5)
(3,-2.5) to (3,-3);;

\draw[red]
(1.75,0.5) to (2.25,0.5)
(2.75,-3) to (3.25,-3)
(3,-0.5) to [R, l=$R_1 \ 10k$ , *-*, color=red] (3,-2.5);
\node at (2,0.25) {GND};
\node at (3,-3.25) {GND};

\draw[red]
(3, 1.5) to (3.5,1.5)  
(3,1) to (3.5,1)
(3.5,2) to (3.5,0.5)
(5.5,2) to (5.5,0.5)
(3.5,0.5) to (5.5,0.5)
(3.5,2) to (5.5,2)
(3.75,1.75) to (3.75,0.75)
(5,1.75) to (5,0.75)
(3.75,0.75) to (5,0.75)
(3.75,1.75) to (5,1.75);
\node at (4.5,2.25) {channel A};
\node at (3, 1.5) [red]{o};
\node at (3,1) [red]{o};
\node at (5.25,0.75) [red]{o};
\node at (5.25, 1.7) [red]{o};
\node at (5.25, 1.3) [red]{o};

\draw[red]
(5.5,-1.5) to (6,-1.5) 
(5.5,-2) to (6,-2)
(6,-1) to (6,-2.5)
(8,-1) to (8,-2.5)
(6,-2.5) to (8,-2.5)
(6,-1) to (8,-1)
(6.25,-1.25) to (6.25,-2.25)
(7.5,-1.25) to (7.5,-2.25)
(6.25,-2.25) to (7.5,-2.25)
(6.25,-1.25) to (7.5,-1.25);
\node at (7,-0.75) {channel B};
\node at (5.5,-1.5) [red]{o};
\node at (5.5,-2) [red]{o};
\node at (7.75,-1.3) [red]{o};
\node at (7.75,-2.25) [red]{o};
\node at (7.75,-1.7) [red]{o};

\draw[red]
(1,-2.5) -- (1,-1.5)
circle [radius = 10pt]node[circle,fill=white,minimum size=16pt]{$\sim$}
(1,-1.5) -- (1,-0.5);
\node at (1,-2.5) [red]{o};
\node at (1,-0.5) [red]{o};
\node at (0.05,-1.8) [green]{Sine};
\draw[thick, red]
(1.5,-1.1) to (1.5,-1.2)
(1.5,-1.1) to (1.4,-1.1)
(1.3,-1.3) to (1.5,-1.1)
(0.8,-1.8) to (0.6,-2);

\node at (3,-0.25) [gray]{1N4007};

\end{circuitikz}
\caption{Układ pomiarowy dla badania własności prostownika jednopołówkowego}
\end{figure}

\begin{figure}[H]
\centering
\includegraphics[width=100mm]{prostownik.jpg}
\caption{Oscylogramy dla prostownika jednopołówkowego}
\end{figure}

Różnica amplitud napięcia między przebiegiem wejściowym oraz wyjściowym wyniosła $1.69V - 0.736V =  0.954V$. Powodem jest działanie prostownika jednopołówkowego, który ,,odcina" prąd płynący w kierunku przeciwnym do kierunku przepływu prądu przez diodę. Zjawisko to możemy zaobserwować w postaci poziomych pasków na oscylogramie 2 (Rysunek 5) 

\subsection{Badanie własności prostownika jednopołówkowego z filtracją}

Kondensatory:w celu uzyskania pojemności $20\mu$F połączono równolegle dwa kondensatory, każdy o pojemności $10\mu$F. Zmierzona pojemność wyniosła $20.46\mu$F. W celu uzyskania pojemności $2.2\mu$F podłączono kondensator o zmierzonej pojemności $2.24\mu$F.
\newline
Użyte elementy rezystancyjne: \\
Rezystor oznaczony kolorami czerwony, czerwony, czerwony, złoty o rezystancji $2.2k\Omega$. Zmierzona rezystancja wyniosła $2.143k\Omega$. \\
Rezystor oznaczony kolorami czerwony, czerwony, czarny, czarny, złoty o rezystancji $220\Omega$. Zmierzona rezystancja wyniosła $218.88\Omega$.\\
Częstotliwość prądu w trakcie wykonywania pomiarów wynosiła 504Hz

\begin{figure}[H]
\centering
\begin{circuitikz}[american voltages]
\draw[green]
(0,-0.5) to (0, 1.5)
(0, 1.5) to (3, 1.5)
(3, 1) to (2,1)
(2,1) to (2,0.5)
(0,-2.5) to (8.5,-2.5)
(8.5,-2.5) to (8.5,-2)
(0,-0.5) to[empty diode, color=red] (2,-0.5)
(2,-0.5) to (8.5,-0.5)
(8.5,-0.5) to (8.5,-1.5)
(2,-2.5) to (2,-3);

\draw[red]
(1.75,0.5) to (2.25,0.5)
(1.75,-3) to (2.25,-3)
(2,-0.5) to [R, l=$R_1 \ 10k$ , *-*, color=red] (2,-2.5)
(4,-0.5) to [C,l=$CF$,*-*,color=red] (4,-2.5)
(6,-0.5) -- (6,-1.5)
circle [radius = 13pt]node[circle,fill=white,minimum size=10pt]{$V_{ac}$} 
(6,-1.5) -- (6,-2.5)
(7.5,-0.5) -- (7.5,-1.5)
circle [radius = 13pt]node[circle,fill=white,minimum size=10pt]{$V_{dc}$} 
(7.5,-1.5) -- (7.5,-2.5);
\node at (2,0.25) {GND};
\node at (2,-3.25) {GND};

\draw[red]
(3, 1.5) to (3.5,1.5)  
(3,1) to (3.5,1)
(3.5,2) to (3.5,0.5)
(5.5,2) to (5.5,0.5)
(3.5,0.5) to (5.5,0.5)
(3.5,2) to (5.5,2)
(3.75,1.75) to (3.75,0.75)
(5,1.75) to (5,0.75)
(3.75,0.75) to (5,0.75)
(3.75,1.75) to (5,1.75);
\node at (4.5,2.25) {channel A};
\node at (3, 1.5) [red]{o};
\node at (3,1) [red]{o};
\node at (5.25,0.75) [red]{o};
\node at (5.25, 1.7) [red]{o};
\node at (5.25, 1.3) [red]{o};

\draw[red]
(8.5,-1.5) to (9,-1.5) 
(8.5,-2) to (9,-2)
(9,-1) to (9,-2.5)
(11,-1) to (11,-2.5)
(9,-2.5) to (11,-2.5)
(9,-1) to (11,-1)
(9.25,-1.25) to (9.25,-2.25)
(10.5,-1.25) to (10.5,-2.25)
(9.25,-2.25) to (10.5,-2.25)
(9.25,-1.25) to (10.5,-1.25);
\node at (10,-0.75) {channel B};
\node at (8.5,-1.5) [red]{o};
\node at (8.5,-2) [red]{o};
\node at (10.75,-1.3) [red]{o};
\node at (10.75,-2.25) [red]{o};
\node at (10.75,-1.7) [red]{o};

\draw[red]
(0,-2.5) -- (0,-1.5)
circle [radius = 10pt]node[circle,fill=white,minimum size=16pt]{$\sim$}
(0,-1.5) -- (0,-0.5);
\node at (0,-2.5) [red]{o};
\node at (0,-0.5) [red]{o};
\node at (-0.95,-1.8) [green]{Sine};
\draw[thick, red]
(0.5,-1.1) to (0.5,-1.2)
(0.5,-1.1) to (0.4,-1.1)
(0.3,-1.3) to (0.5,-1.1)
(-0.2,-1.8) to (-0.4,-2);

\node at (2,-0.25) [gray]{1N4007};

\end{circuitikz}
\caption{Układ prostownika jednopołówkowego z filtracją}
\end{figure}

\begin{table}[H]
\centering
\begin{tabular}{|c|c||c|c|c|}
\hline
R [$\Omega$] & $C_{f}$[$\mu$F] & $U_{R(DC)}$[V] & $U_{R(AC)}$[V]&$ U_{R(pp)}$[V]\\
\hline 
$2.2k\Omega$ & $20\mu$F & 1.776V & 22.67mV & 5.36V\\
\hline
$220\Omega$ & $20\mu$F & 1.001V & 108.6mV & 4.96V\\
\hline
$220\Omega$ & $2.2\mu$F & 0.681V &152.3mV & 5.44V\\
\hline
$2.2k\Omega$ & $2.2\mu$F & 1.727V & 194.21mV & 5.68V\\
\hline
\end{tabular}
\caption{Tabela pomiarów dla częstotliwości 504Hz}
\end{table}

Ze względu na duże tętnienia i małą sprawność energetyczną,
prostownik jednopołówkowy jest rzadko stosowany. W celu zmniejszenia tętnień oraz
zwiększenia wydatkowania energii, w obciążeniu prostownika stosuje się
elementy,   które   magazynują   energię. 
Wartość napięć stałych jest największa w przypadku obciążęnia $2.2k\Omega$.W przypadku napięć zmiennych wartości największe występują w przypadku kondensatora o pojemności $2.2\mu$F. Najmniejsza wartość  napięcia międzyszczytowego tętnień jest odnotowana dla kondensatora o większej pojemności oraz rezystora o mniejszym oporze. Największa została zmierzona dla kondensatora o mniejszej ojemności oraz rezystora o większym oporze. 
\\
\\
\begin{figure}[H]
\centering
\includegraphics[width=100mm]{filtracja.jpg}
\caption{Oscylogramy dla prostownika jednopołówkowego z filtracją}
\end{figure}

Wartość napięcia $U_{R(AC)}$ i $U_{R(DC)}$ zależy od wartości rezystancji R i pojemności $C_f$. Napięcia możemy interpretować odpowiednio jako wartość prądu sinusoidalnie zmiennego, który nie został wyprostowany przez prostownik, oraz wartość napięcia prądu stałego, który został zgromadzony przez kondensator i jest uwalniany w procesie ,,tłumienia".
\section{Diody świecące}
\subsection{Cel}
Ćwiczenie ma na celu zbadanie wpływu wartości napięcia na sposób świecenia diody czerwonej oraz żółtej.
\subsection{Pomiar}
Użyte elementy rezystancyjne: 
Rezystor oznaczony kolorami brązowy, czarny, czerwony, złoty o rezystancji $1000\Omega$. Zmierzona rezystancja wyniosła $983.8\Omega$. 
\begin{figure}[H]
\centering
\begin{circuitikz}[american voltages]
\draw[dashed][green]
(1,-0.5) to (1,1)
(3,-0.5) to (3,1)
(4,-0.5) to (4,1)
(6,-0.5) to (6,1);

\draw[green]
(3,-0.5) to (4,-0.5)
(5.5,-0.5) to (6.5,-0.5)
(1,-2.5) to (6.5,-2.5)
(4,-0.5) to [empty diode, color=red](5.5,-0.5)
(3,-2.5) to (3,-3)
(1,-0.5) to [empty diode, color=red] (3,-0.5);

\draw[red]
(2.75,-3) to (3.25,-3)
(6.5,-2.5) to [R, l_=$R_1 \ 1k$ , *-*, color=red] (6.5,-0.5)
(1,1) -- (2,1)
circle [radius = 14pt]node[circle,fill=white,minimum size=10pt]{$mV$} 
(2,1) -- (3,1)
(4,1) -- (5,1)
circle [radius = 14pt]node[circle,fill=white,minimum size=10pt]{$mV$} 
(5,1) -- (6,1);
\node at (3,-3.25) {GND};


\draw[red]
(1,-2.5) -- (1,-1.5)
circle [radius = 10pt]node[circle,fill=white,minimum size=16pt]{}
(1,-1.5) -- (1,-0.5);
\node at (1,-2.5) [red]{o};
\node at (1,-0.5) [red]{o};
\draw[thick, red]
(1.5,-1.1) to (1.5,-1.2)
(1.5,-1.1) to (1.4,-1.1)
(1.3,-1.3) to (1.5,-1.1)
(0.8,-1.8) to (0.6,-2);

\node at (2.2,0.2) [gray]{LED1};
\node at (5.2,0.2) [gray]{LED2};

\node at (2.9,-0.75) {green};
\node at (5.5,-0.75) {red};

\draw[-latex][red] (1,-1.7) -- (1,-1.3);
\draw[-latex][red] (2,-1) -- (2.25,-1.25);
\draw[-latex][red] (2.2,-1) -- (2.45,-1.25);
\draw[-latex][red] (5,-1) -- (5.25,-1.25);
\draw[-latex][red] (5.2,-1) -- (5.45,-1.25);

\node at (2,-2) {max 15V};

\end{circuitikz}
\caption{Układ doświadczalny dla układu diod świecących}
\end{figure}

\begin{table}[H]
\centering
\begin{tabular}{|c|c|c|}
\hline
napięcie & dioda czerwona & dioda zielona\\
\hline 
6V & 1.9174V & 1.9173V\\
\hline
9.1V & 2.0065V & 1.9553V\\
\hline
13.2V & 2.113V & 1.982V\\
\hline
\end{tabular}
\caption{Tabela pomiarów}
\end{table}

\subsection{Wnioski}
Kolor diody zależy od materiałów stosowanych w strukturze złącza. Diody o różnych kolorach mają róźne napięcia przewodzenia. Z tego powodu dioda zielona świeci mocniej niż czerwona.

\begin{figure}[H]
\centering
\includegraphics[width=100mm]{diody1.jpg}
\caption{Diody świecące pod napięciem $6V$}
\end{figure}

\begin{figure}[H]
\centering
\includegraphics[width=100mm]{diody2.jpg}
\caption{Diody świecące pod napięciem $9.1V$}
\end{figure}

\begin{figure}[H]
\centering
\includegraphics[width=100mm]{diody3.jpg}
\caption{Diody świecące pod napięciem $13.2V$}
\end{figure}

\section{Wyświetlacz LED}

\begin{figure}[H]
\centering
\begin{circuitikz}[american voltages]
\draw[thick]
(0,-0.5) to (7,-0.5)
(0,-0.5) to (0,10.5)
(0,10.5) to (7,10.5)
(7,10.5) to (7,-0.5);

\draw
(0.5,10.5) to (0.5,11.5)
(2,10.5) to (2,11.5)
(5,10.5) to (5,11.5)
(6.5,10.5) to (6.5,11.5)
(0.5,-0.5) to (0.5,-1.5)
(2,-0.5) to (2,-1.5)
(5,-0.5) to (5,-1.5)
(6.5,-0.5) to (6.5,-1.5);

\draw[green]
(0.5,11.5) to (0.5,12)
(5,11.5) to (5,12)
(6.5,11.5) to (6.5,12)
(0.5,-1.5) to (0.5,-2)
(2,-1.5) to (2,-2)
(6.5,-1.5) to (6.5,-2)
(0.5,12) to (9,12)
(0.5,-2) to (9,-2)
(8,-3) to (13,-3)
(13,-3) to (13,5)
(13,7) to (13,12)
(9,-2) to (9,12)
(9,12) to (13,12);

\draw
(3.5,-0.5) to (3.5,-3)
(3.5,-3) to (6,-3);

\draw[red]
(13,5) -- (13,6)
circle [radius = 10pt]node[circle,fill=white,minimum size=16pt]{} 
(13,6) -- (13,7)
(6,-3) to [R, l_=$R_1 \ 1k\Omega$ , *-*, color=red] (8,-3);

\fill[black!20!white] (0,-0.5) rectangle (7,10.5);

\fill[red!70!white] (0.5,1) rectangle (1.5,4.5);
\fill[red!70!white] (5.5,9) rectangle (4.5,5.5);
\fill[red!70!white] (1.5,4.5) rectangle (4.5,5.5);
\fill[red!70!white] (1.5,0) rectangle (4.5,1);
\fill[red!70!white] (1.5,9) rectangle (4.5,10);
\fill[white](5.5,1) rectangle (4.5,4.5);
\fill[white](0.5,9) rectangle (1.5,5.5);
\fill[red!70!white] (6.2,0.5) circle (0.5cm);

\draw[thick] (0.5,1) rectangle (1.5,4.5);
\draw[thick]  (5.5,1) rectangle (4.5,4.5);
\draw[thick]  (0.5,9) rectangle (1.5,5.5);
\draw[thick]  (5.5,9) rectangle (4.5,5.5);
\draw[thick]  (1.5,4.5) rectangle (4.5,5.5);
\draw[thick]  (1.5,0) rectangle (4.5,1);
\draw[thick]  (1.5,9) rectangle (4.5,10);
\draw[thick] (6.2,0.5) circle (0.5cm);

\node at (13.7,6) {$5V$};

\node at (0.5,11.5) [red]{o};
\node at (2,11.5) [red]{o};
\node at (5,11.5)[red]{o};
\node at (6.5,11.5)[red]{o};
\node at (0.5,-1.5) [red]{o};
\node at (2,-1.5) [red]{o};
\node at (5,-1.5) [red]{o};
\node at (6.5,-1.5) [red]{o};

\node at (0,11.5) {a};
\node at (1.5,11.5) {b};
\node at (4.5,11.5) {c};
\node at (6,11.5) {d};
\node at (0,-1.5) {e};
\node at (1.5,-1.5) {f};
\node at (4.5,-1.5) {g};
\node at (6,-1.5) {dp};

\node at (1,3) {e};
\node at (5,3) {g};
\node at (1,7.5) {b};
\node at (5,7.5) {d};
\node at (3,9.5) {c};
\node at (3,5) {a};
\node at (3,0.5) {f};
\node at (6.2,0.5) {dp};


\draw[-latex][red] (13,5.8) -- (13,6.2);

\fill [red] (0,0) node[anchor=north]{}
  -- (0,0) node[anchor=north]{}
  -- (0,0) node[anchor=south]{}
  -- cycle;


\end{circuitikz}
\caption{Schemat podłączenia wyświetlacza LED dla cyfry 2}
\end{figure}

\begin{figure}[H]
\centering
\includegraphics[width=100mm]{wyswietlacz.jpg}
\caption{Wyświetlacz LED w działaniu}
\end{figure}

\newpage


\begin{thebibliography}{3}
\bibitem{naum} http://mariusznaumowicz.ddns.net/materialy.html - materiały dostępne dla przedmiotu Podstawy Elektroniki
\bibitem{pto}Podstawy teorii obwodów, Osiowski J., Szabatin J., WNT, Warszawa, 1998
\bibitem{pe}"Podstawy Elektrotechniki", R.Kurdziel, wyd II, WNT Warszawa 1972
\end{thebibliography}

\newpage
\tableofcontents{}
\end{document}
