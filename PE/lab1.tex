\documentclass[11pt]{article}
\usepackage{amsmath}          %koniecnie
\usepackage{amssymb,amsfonts,amsthm}%dodatkowo
\usepackage[polish]{babel}
\usepackage{polski}
\usepackage[utf8]{inputenc}
\usepackage[T1]{fontenc}
\usepackage{graphicx}
\usepackage{geometry}
\usepackage{tikz}
\usepackage{circuitikz}

\selectlanguage{polish}

\newgeometry{tmargin=3cm, bmargin=3cm, lmargin=2cm, rmargin=2cm}

\newcommand{\PRzFieldDsc}[1]{\sffamily\bfseries\scriptsize #1}
\newcommand{\PRzFieldCnt}[1]{\textit{#1}}
\newcommand{\PRzHeading}[8]{
\begin{center}
\begin{tabular}{ p{0.32\textwidth} p{0.15\textwidth} p{0.15\textwidth} p{0.12\textwidth} p{0.12\textwidth} }

  &   &   &   &   \\
\hline
\multicolumn{5}{|c|}{}\\[-1ex]
\multicolumn{5}{|c|}{{\LARGE #1}}\\
\multicolumn{5}{|c|}{}\\[-1ex]

\hline
\multicolumn{1}{|l|}{\PRzFieldDsc{Kierunek}}	& \multicolumn{1}{|l|}{\PRzFieldDsc{Specjalność}}	& \multicolumn{1}{|l|}{\PRzFieldDsc{Rok studiów}}	& \multicolumn{2}{|l|}{\PRzFieldDsc{Symbol grupy lab.}} \\
\multicolumn{1}{|c|}{\PRzFieldCnt{#2}}		& \multicolumn{1}{|c|}{\PRzFieldCnt{#3}}		& \multicolumn{1}{|c|}{\PRzFieldCnt{#4}}		& \multicolumn{2}{|c|}{\PRzFieldCnt{#5}} \\

\hline
\multicolumn{4}{|l|}{\PRzFieldDsc{Temat Laboratorium}}		& \multicolumn{1}{|l|}{\PRzFieldDsc{Numer lab.}} \\
\multicolumn{4}{|c|}{\PRzFieldCnt{#6}}				& \multicolumn{1}{|c|}{\PRzFieldCnt{#7}} \\

\hline
\multicolumn{5}{|l|}{\PRzFieldDsc{Skład grupy ćwiczeniowej oraz numery indeksów}}\\
\multicolumn{5}{|c|}{\PRzFieldCnt{#8}}\\

\hline
\multicolumn{3}{|l|}{\PRzFieldDsc{Uwagi}}	& \multicolumn{2}{|l|}{\PRzFieldDsc{Ocena}} \\
\multicolumn{3}{|c|}{\PRzFieldCnt{\ }}		& \multicolumn{2}{|c|}{\PRzFieldCnt{\ }} \\

\hline
\end{tabular}
\end{center}
}
\begin{document}
\PRzHeading{Laboratorium Podstaw Elektroniki}{Informatyka}{--}{I}{I2}{Laboratoria wprowadzające}{1}{Martyna Maciejewska(132273), Michał Bień(132191), Ziemowit Sokołowski(132318)}{}


\section{Laboratorium 1}
\begin{center}
Odczytanie wartości rezystancji na podstawie kodu paskowego rezystorów.
\begin{tabular}{|c|c|c|c|}
\hline
R&Barwy&Odczyt&Pomiar\\
\hline 
R1&brązowy, czarny, brązowy, złoty&$100\Omega\pm 5\%$ & $101.5\Omega$\\
\hline
 R2& brązowy, czarny, czerwony, złoty & $1000\Omega\pm 5\%$ & $0.98k\Omega$ \\
\hline
 R3& czerwony, czarny, czerwony, złoty & $2000\Omega\pm 5\%$ & $1.972k\Omega$ \\
\hline
 R4& pomarańczowy, pomarańczowy, zielony, złoty & $3.3M\Omega\pm 5\%$ & $3.23M\Omega$ \\
\hline
 R5& czerwony, czarny, zielony, złoty & $2M\Omega\pm 5\%$ & $1.970M\Omega$ \\
\hline
 R6& szary, czerwony, brązowy, złoty & $820\Omega\pm 5\%$ & $0.822k\Omega$ \\
\hline
\end{tabular}
\newline
\end{center}
\begin{center}
Odczyt wartości pojemności kondensatorów na podstawie ich oznaczeń.
\end{center}
$$
\begin{array}{|c|c|c|c|}
\hline
 C&Oznaczenie&Odczyt&Pomiar\\
\hline 
C1&10\mu F,  250V&10\mu F& 10.61\mu F\\
\hline
C2&22\mu F,  35V&22\mu F& 20.939\mu F\\
\hline
C3&10nF&10nF& 8.35nF\\
\hline
 C4&222&2.2nF& 2.29nF\\
\hline
 C5&223&22nF&28.62nF\\
\hline
C6&332&3.3nF&3.05nF\\
\hline
\end{array}
$$
\\
\\
\begin{center}
Pomiar indukcyjności cewek przy pomocy mostka pomiarowego.
\end{center}
$$
\begin{array}{|c|c|}
\hline
 L&Pomiar\\
\hline 
L1&30.5mH\\
\hline
L2&38.55\mu H\\
\hline
L3&30.31mH\\
\hline
\end{array}
$$
\newline
\newline
{\bfseries Wnioski:} Wartości odczytane z rezystorów oraz  kondensatorów są zbliżone do wartości rezystancji i pojemności, które uzyskaliśmy w wyniku pomiaru.
\newline
\newline
\section{Laboratorium 2}
{\bfseries Cel:} Obliczenie wartości rezystancji zastępczych od strony zacisków A B.
\subsection{Część I}
\begin{center}
\begin{circuitikz}[american voltages]
\draw
(0,3) to (0,0)
(0,3) to[R, l=$R_7 \ 1k$](2,3)
(2,3) to (2,3.5)
(2,3) to (2,2)
(2,3.5) to[R, l=$R_7 \ 1k$](4,3.5)
(2,2) to[R, l=$R_7 \ 1k$](4,2)
(4,3.5) to (4,3)
(4,2) to (4,3)
(4,3) to (4.2,3)
(4.2,3) to (4.2,7)
(4.2,3) to (4.2,2)
(4.2,7) to [R, l=$R_7 \ 1k$](6.2,7)
(4.2,5.5) to [R, l=$R_2 \ 3k$](6.2,5.5)
(4.2,4) to [R, l=$R_3 \ 1k$](6.2,4)
(4.2,2) to [R, l=$R_4 \ 270$](6.2,2)
(6.2,3) to (6.5,3)
(6.2,7) to (6.2,2)
(6.5,2.5) to (6.5,4)
(6.5,4) to [R, l=$R_8 \ 1$](8.5,4)
(6.5,2.5) to [R, l=$R_9 \ 100$](8.5,2.5)
(8.5,2.5) to (8.5, 4)
(8.5,3) to (9,3)
(9,3) to (9,0)
(0,0) to (4,0) 
node[above] at(4,0){A}
(9,0) to (5,0) 
node[above] at(5,0){B}
  ;
\end{circuitikz}
\end{center}


\[
\frac{1}{R_{a}}= \frac{1}{R_{5}} + \frac{1}{R_{6}}  
\hspace{1.5cm} 
R_{a}=\frac{R_{5}*R_{6}}{R_{5}+R_{6}}
\]

\[
\frac{1}{R_{b}}= \frac{1}{R_{1}} + \frac{1}{R_{2}} + \frac{1}{R_{3}} + \frac{1}{R_{4}}
\hspace{1cm}
R_{b}=\frac{R_{1}*R_{2}*R_{3}*R_{4}}{R_{2}*R_{3}*R_{4}+R_{1}*R_{3}*R_{4}+R_{1}*R_{2}*R_{4}+R_{1}*R_{2}*R_{3}}
\]

\[
\frac{1}{R_{c}}= \frac{1}{R_{8}} + \frac{1}{R_{9}}  
\hspace{1.5cm} 
R_{c}=\frac{R_{8}*R_{9}}{R_{8}+R_{9}}
\]

\[ R_{z}=R_{7}+R_{a}+R_{b}+R_{c}\]

\[ R_{z}=R_{7}+\frac{R_{5}*R_{6}}{R_{5}+R_{6}}+\frac{R_{1}*R_{2}*R_{3}*R_{4}}{R_{2}*R_{3}*R_{4}+R_{1}*R_{3}*R_{4}+R_{1}*R_{2}*R_{4}+R_{1}*R_{2}*R_{3}}+\frac{R_{8}*R_{9}}{R_{8}+R_{9}}
\]

\[ R_{z}=1248,27\Omega\]
\newpage
\subsection{Część II}

Obwód 1

\begin{center}
\begin{circuitikz}[american voltages]
\draw
(-1,0) to (-1,0.5)
			node[left] at(-1,0.5){B}
	  (-1,1.5) to (-1,2)
	  		node[left] at(-1,1.5){A}
	  (-1,2) to (2,2) --
	  (2,2) to [R, l=$R_2 \ 2k$] (4,2) --
      (4,2) to (4,0) --
      (4,0) to [R, l=$R_3 \ 2k$] (2,0) --
      (2,0) to (-1,0) --
      (1,0) to [R, l=$R_1 \ 1k$] (1,2)

;

\end{circuitikz}
\end{center}

\[
R_{a}=R_{2}+R_{3}
\hspace{1.5cm}
\frac{1}{R_{z}}= \frac{1}{R_{a}} + \frac{1}{R_{1}}  
\hspace{1.5cm} 
R_{z}=\frac{R_{a}*R_{1}}{R_{a}+R_{1}}
\]
Obwód 2
\begin{center}
\begin{circuitikz}[american voltages]
\draw

(0,0) to [R, l=$R_1 \ 1k$] (0,2) --
	(0,2) to (3,2) -- 
	(3,2) to [R, l=$R_3 \ 1k$] (6,2) --
	(6,2) to [R, l=$R_5 \ 100$] (6,0) --
	(6,0) to [R, l=$R_4 \ 2k$] (3,0) --	
	(3,0) to (0,0) --
	(2,0) to [R, l=$R_2 \ 1k$] (2,2)
	(3,0) to (3,-1)
	(3,-1) to (4,-1)
	node[right] at(4,-1){A} 
	(6,0) to (6,-1)
	(6,-1) to (5,-1)
	node[left] at(5,-1){B} 

;

\end{circuitikz}
\end{center}

\[
\frac{1}{R_{a}}= \frac{1}{R_{1}} + \frac{1}{R_{2}}  
\hspace{1.5cm} 
R_{b}=R_{a}+R_{3}+R_{5}
\hspace{1.5cm}
\frac{1}{R_{z}}= \frac{1}{R_{b}} + \frac{1}{R_{4}}
\]
Obwód 3
\begin{center}
\begin{circuitikz}[american voltages]
\draw

(-1,-1) to (-1,-0.5)
	node[above] at(-1,-0.5){B}
	(-1,1) to (-1,0.5)
	node[below] at(-1, 0.5){A}
	(-1,1) to (2,1)
	(2,1) to [R, l=$R_3 \ 100$] (4,1)
	(4,1) to (5,1)
	(5,1) to [R, l=$R_2 \ 2k$] (5,-1)
	(1,-1) to [R, l=$R_1 \ 2k$] (1,1)
	(1.5,1) to (1.5,2.5) 
	(4,1) to (4,2.5) 
	(1.5,2.5) to [R, l=$R_4 \ 1k$] (4,2.5)
	(5,-1) to (6,-1)
	(5,1) to (6,1)
	(-1,-1) to (1,-1)
	
;


\end{circuitikz}
\end{center}
\[R_{z}=R_{1}\]

\newpage


Obwód 4
\begin{center}
\begin{circuitikz}[american voltages]
\draw

(0,2) to (0,3)
	(0,3) to (9,3)
	(9,3) to (9,2)
	(9,2) to[R, l=$R_5 \ 100$] (6,2)
	(6,2) to[R, l=$R_2 \ 2k$] (3,2)
	(3,2) to[R, l=$R_1 \ 1k$] (0,2)
	(3,2) to[R, l=$R_3 \ 2k$] (3,-1)
	(3,-1) to (6,-1)
	(6,-1) to[R, l=$R_4 \ 1k$] (6,1)
	(6, 1) to (6, 2)
	(6,-1) to (7,-1)
	(6,1) to (7,1)
	(7,-1) to (7,-0.5)
	node[right] at(7,-0.5){B}
	(7,1) to (7,0.5)
	node[right] at(7,0.5){A}

	
;


\end{circuitikz}
\end{center}
\[
R_{a}=R_{1}+R_{5}
\hspace{1.5cm}
\frac{1}{R_{b}}= \frac{1}{R_{a}} + \frac{1}{R_{2}}  
\hspace{1.5cm}
R_{c}=R_{b}+R_{3}
\hspace{1.5cm}
\frac{1}{R_{z}}= \frac{1}{R_{c}} + \frac{1}{R_{4}}
\]
Obwód 5
\begin{center}
\begin{circuitikz}[american voltages]
\draw

(-1,1) to (-1,0.5)
	node[left] at(-1,0.5){A}
	(-1,-1) to (-1,-0.5)
	node[left] at(-1,-0.5){B}
	(-1,1) to (1.5,1)
	(1.5,1) to [R, l=$R_3 \ 2k$] (4,1)
	(4,1) to (5,1)
	(5,1) to [R, l=$R_2 \ 2k$] (5,-1)
	(5,-1) to (1,-1)
	(1,-1) to (-1,-1)
	(1,-1) to [R, l=$R_1 \ 2k$] (1,1)
	(1.5,1) to (1.5,3) 
	(4,1) to (4,3) 
	(1.5,3) to [R, l=$R_4 \ 1k$] (4,3)
	(5,-1) to (6,-1)
	(5,1) to (6,1)


	
;


\end{circuitikz}
\end{center}
\[
\frac{1}{R_{a}}= \frac{1}{R_{4}} + \frac{1}{R_{3}}  
\hspace{1.5cm}
R_{b}=R_{a}+R_{2}
\hspace{1.5cm}
\frac{1}{R_{z}}= \frac{1}{R_{b}} + \frac{1}{R_{1}}
\]
Obwód 6
\begin{center}
\begin{circuitikz}[american voltages]
\draw
  (0,0) to (8,0)
  (0,0) to (0,2)
  (0,2) to[R, l=$R_5 \ 1k$] (2,2)
  (2,2) to (2,1)
  (2,1) to (4,1)
  (4,1) to[R, l=$R_2 \ 2k$] (4,3)
  (4,3) to[R, l=$R_4 \ 2k$] (8,3)
  (8,3) to (8,0)
  (2,2) to[R, l=$R_1 \ 100$] (2,4)
  (4,3) to (4,4)
  (2,4) to (4,4)
  (3,4) to[R, l=$R_3 \ 1k$] (3,6)
  (3,6) to (8,6)
  (8,6) to (8,5.7)
  (8,4) to (8,3)
  node[below] at(8,5.7){A}
  node[above] at(8,4){B}
  ;
  \end{circuitikz}
\end{center}
\[
\frac{1}{R_{a}}= \frac{1}{R_{1}} + \frac{1}{R_{2}}  
\hspace{1.5cm}
R_{b}=R_{a}+R_{5}
\hspace{1.5cm}
\frac{1}{R_{c}}= \frac{1}{R_{b}} + \frac{1}{R_{4}}
\hspace{1.5cm}
R_{z}=R_{c}+R_{3}
\]

\begin{center}
\begin{tabular}{|c|c|c|}
\hline
Obwód&Obliczenia $R_{z}$ &Pomiar\\
\hline 
Obwód 1&$800\Omega$& $784.17\Omega$\\
\hline
Obwód 2&$94.6\Omega$& $94.8\Omega$\\
\hline
Obwód 3&$2000\Omega$& $1952\Omega$\\
\hline
Obwód 4&$730.4\Omega$&$720.5\Omega$\\
\hline
Obwód 5&$1142.85\Omega$&$1120.5\Omega$\\
\hline
Obwód 6&$1707.7\Omega$&$1680\Omega$\\
\hline
\end{tabular}
\end{center}

{\bfseries Wnioski:} Różnica między pomiarem i odczytem wynika z nieuwzględnionego w obliczeniach oporu przewodów doprowadzających oraz niedokładności rezystorów\(\pm5\%\). 
\newline
\newline 
\section{Laboratorium 3}
\subsection{Część I}

\begin{center}
Pomiar napięcia z sekcji DC POWER SUPPLY zestawu laboratoryjnego DF 6911.
\end{center}

$$
\begin{array}{|c|c|c|}
\hline
U&Pomiar&Odczyt\\
\hline 
1[V]&1V& 1.128V\\
\hline
3[V]&3V& 3.21V\\
\hline
4.5[V]&4.5V& 4.691V\\
\hline
11[V]&11V& 11.203V\\
\hline
13[V]&13V& 13.18V\\
\hline
25[V]&25V& 25.256V\\
\hline
28[V]&28V& 28.237V\\
\hline
\end{array}
$$

\subsection{Część II}

\begin{center}
Testy i obliczenia przeprowadzone dla dzielnika, który dla napięcia wejściowego 15V miał otrzymywać na wyjściu 3.3V
Wnioskując z prawa Ohma, skorzystaliśmy z faktu, że dzielnik napięć rozkłada napięcie w obwodzie proporcjonalnie do oporu opornika. A więc: 
\end{center}
$$R_z = R_1 + R_2$$
$$I_z = I_1 = I_2$$
$$U_z = U_1 + U_2$$
$$R_2 = \frac{U_2}{I_2}$$
$$\frac{R_1}{R_2} = \frac{U_1}{U_2}$$
$$\frac{R_1}{R_2} = \frac{U_z-U_2}{U_2}$$
$$U_2 = \frac{U_z}{\frac{R_1}{R_2}+1}$$
\begin{center}
Tak więc ,,skala" dzielnika nie wynika z oporu poszczególnych rezystorów, a stosunku ich oporów. Dla naszych potrzeb stosunek ten wyniósł $\frac{R_1}{R_2} = 3.55$ Podłączyliśmy więc rezystor $R_1 = 3.5k\Omega$ oraz $R_2 = 1k\Omega$
\end{center}

\begin{center}
\begin{circuitikz}[american voltages]
\draw
(1,2) to (1,0)
(1,2) to (1.5,2)
(1.5,2) to[R, l=$R_1 \ 3.5k$](3.5,2)
(4.5,2) to (4.5,3.5)
(4.5,3.5) to (5,3.5)
(7,3.5) to (7.5,3.5)
node[right] at(5,3.5){U out 3.3V}
(7.5,3.5) to (7.5,2)
(3.5,2) to (5,2)
(7,2) to (8,2)
(5,2) to [R, l=$R_2 \ 1k$](7,2)
(8,2) to (8,0)
(1,0) to (4,0) 
(4,0)to[V,v=$U in \ 15V$ ] (5,0)
(8,0) to (5,0) 
  ;
\end{circuitikz}
\end{center}

\subsection{Część III} 

\begin{center}
Dane pomiarowe dla obwodu II.
\end{center}

\begin{center}
\begin{tabular}{|c|c|c|}
\hline
$R$&Spadek napięcia&Natężenie\\
\hline 
$R_1$&$5.15V$& $2.575mA$\\
\hline
$R_2$&$3.88V$& $1.967mA$\\
\hline
$R_3$&$2.585V$& $0.652mA$\\
\hline
$R_4$&$1.64V$& $1.313mA$\\
\hline
\end{tabular}
\end{center}


\begin{center}
\begin{circuitikz}[american voltages]
\draw

(-1,1) to  (-1,0.5)
(-1,0.5) to [V,v=$V_1 \ 5V$ ] (-1,-0.5)
	(-1,-1) to (-1,-0.5)
	(-1,1) to (0,1)
	(0,1) to [R, l=$R_1 \ 2100$] (2,1)
	(-1,-1) to (3,-1)
	(1,-1) to (-1,-1) 
	(2,1) to (3,1)
	(3,1) to (3,-1)


	
;


\end{circuitikz}
\end{center}

\subsection{Część IV}

\begin{center}
Dane pomiarowe i obliczone wartości dla obwodu 2
\end{center}


$$
\begin{array}{|c|c|c|c|c|c|}
\hline
&R&U obliczone&U pomiaru&I obliczone&I pomiaru\\
\hline
R_1&2k\Omega&5V&5.15V&2.5mA&2.631mA\\
\hline
R_2&2k\Omega&3.74V&3.88V&1.87mA&1.967mA\\
\hline
R_3&2k\Omega&1.26V&1.08V&0.63mA&0.652mA\\
\hline
R_4&1k\Omega&1.26V&1.64V&1.26mA&1.313mA\\
\hline
R_z&1142.86\Omega&5V&5.15V&3.9mA&3.944mA\\
\hline
\end{array}
$$
$$R_z = \frac{R_1R_{234}}{R_1+R_{234}} = \frac{R_1(R_2+R_{34})}{R_1+R_2+R_{34}} = \frac{R_1(R_2+\frac{R_3R_4}{R_3+R_4})}{R_1+R_2+\frac{R_3R_4}{R_3+R_4}}$$
$$I_z = I_1 + I_2 $$
$$U_z = U_1 = U_{234}$$
$$ I_2 = I_{34} = I_3+I_4$$

\begin{center}
\begin{circuitikz}[american voltages]
\draw

(-1,1) to  (-1,0.5)
(-1,0.5) to [V,v=$V_1 \ 5V$ ] (-1,-0.5)
	(-1,-1) to (-1,-0.5)
	(-1,1) to (1.5,1)
	(1.5,1) to [R, l=$R_3 \ 2k$] (4,1)
	(4,1) to (5,1)
	(5,1) to [R, l=$R_2 \ 2k$] (5,-1)
	(5,-1) to (1,-1)
	(1,-1) to (-1,-1)
	(1,1) to [R, l=$R_1 \ 2k$] (1,-1)
	(1.5,1) to (1.5,3) 
	(4,1) to (4,3) 
	(1.5,3) to [R, l=$R_4 \ 1k$] (4,3)
	(5,-1) to (6,-1)
	(5,1) to (6,1)


	
;


\end{circuitikz}
\end{center}

\newpage
\section{Bibliografia}


http://mariusznaumowicz.ddns.net/materialy.html - materiały dostępne dla przedmiotu Podstawy Elektroniki
\newline
\newline
Podstawy teorii obwodów, Osiowski J., Szabatin J., WNT, Warszawa, 1998
\newpage
\tableofcontents{}
\end{document}