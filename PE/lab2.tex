\documentclass[polish,a4paper]{article}
\usepackage{amsmath}
\usepackage{amssymb,amsfonts,amsthm}
\usepackage[english,main=polish]{babel}
\usepackage{polski}
\usepackage[utf8]{inputenc}
\usepackage[T1]{fontenc}
\usepackage{graphicx}
\usepackage{geometry}
\usepackage{tikz}
\usepackage{circuitikz}
\usepackage{float}
\usepackage{etoolbox}
\patchcmd{\thebibliography}{\section*}{\section}{}{}

\selectlanguage{polish}
\title{Lab2}
\newgeometry{tmargin=3cm, bmargin=3cm, lmargin=2cm, rmargin=2cm}

\newcommand{\PRzFieldDsc}[1]{\sffamily\bfseries\scriptsize #1}
\newcommand{\PRzFieldCnt}[1]{\textit{#1}}
\newcommand{\PRzHeading}[8]{
\begin{center}
\begin{tabular}{ p{0.32\textwidth} p{0.15\textwidth} p{0.15\textwidth} p{0.12\textwidth} p{0.12\textwidth} }

  &   &   &   &   \\
\hline
\multicolumn{5}{|c|}{}\\[-1ex]
\multicolumn{5}{|c|}{{\LARGE #1}}\\
\multicolumn{5}{|c|}{}\\[-1ex]

\hline
\multicolumn{1}{|l|}{\PRzFieldDsc{Kierunek}}	& \multicolumn{1}{|l|}{\PRzFieldDsc{Specjalność}}	& \multicolumn{1}{|l|}{\PRzFieldDsc{Rok studiów}}	& \multicolumn{2}{|l|}{\PRzFieldDsc{Symbol grupy lab.}} \\
\multicolumn{1}{|c|}{\PRzFieldCnt{#2}}		& \multicolumn{1}{|c|}{\PRzFieldCnt{#3}}		& \multicolumn{1}{|c|}{\PRzFieldCnt{#4}}		& \multicolumn{2}{|c|}{\PRzFieldCnt{#5}} \\

\hline
\multicolumn{4}{|l|}{\PRzFieldDsc{Temat Laboratorium}}		& \multicolumn{1}{|l|}{\PRzFieldDsc{Numer lab.}} \\
\multicolumn{4}{|c|}{\PRzFieldCnt{#6}}				& \multicolumn{1}{|c|}{\PRzFieldCnt{#7}} \\

\hline
\multicolumn{5}{|l|}{\PRzFieldDsc{Skład grupy ćwiczeniowej oraz numery indeksów}}\\
\multicolumn{5}{|c|}{\PRzFieldCnt{#8}}\\

\hline
\multicolumn{3}{|l|}{\PRzFieldDsc{Uwagi}}	& \multicolumn{2}{|l|}{\PRzFieldDsc{Ocena}} \\
\multicolumn{3}{|c|}{\PRzFieldCnt{\ }}		& \multicolumn{2}{|c|}{\PRzFieldCnt{\ }} \\

\hline
\end{tabular}
\end{center}
}
\begin{document}
\PRzHeading{Laboratorium Podstaw Elektroniki}{Informatyka}{--}{I}{I2}{Twierdzenie Thevenina}{2}{Martyna Maciejewska(132273), Michał Bień(132191), Ziemowit Sokołowski(132318)}{}
\section{Cel}


Celem laboratorium było zapoznanie się z Twierdzeniem Thevenina, oraz doświadczalne unaocznienie istotnych następstw i efektów tego twierdzenia dla elektroniki. Brzmienie twierdzenia poznaliśmy z materiałów udostępnionych do realizacji ćwiczenia\cite{naum}:
\begin{quotation}
Twierdzenie Thevenina można sformułować w następujący sposób, cytując\cite{pe}:
\begin{quotation}
Prąd płynący przez odbiornik rezystancyjny R, przyłączony do dwóch zacisków AB dowolnego
liniowego układu zasilającego prądu stałego jest równy ilorazowi napięcia $U_0$ mierzonego na zaciskach AB w stanie jałowym przez rezystancję R powiększoną o rezystancję zastępczą $R_w$ układu zasilającego mierzoną na zaciskach AB
\end{quotation}
Twierdzenie to spotykane jest również pod nazwą twierdzenia o zastępczym źródle napięcia i bywa
sformułowane następująco:
\begin{quotation}
Obwód elektryczny liniowy o dowolnym ukształtowaniu, traktowany jako złożony dwójnik liniowy
aktywny o zaciskach AB, można zastąpić jednym źródłem o napięciu źródłowym E, równym
napięciu stanu jałowego $U_0$ na zaciskach AB i o rezystancji wewnętrznej $R_w$, równej rezystancji zastępczej mierzonej na zaciskach AB obwodu.
\end{quotation}
\end{quotation}
Zadaniem laboratoryjnym było obliczenie zadanych prądów w obwodzie wskazanym przez prowadzącego zajęcia (zadanie 2, podpunkt 4): (rysunek 1)

\begin{figure}[H]
\centering
\begin{circuitikz}[american voltages]
\draw

(-1,0) to [R, l=$R_4 \ 100\Omega$] (-1,2)
(1,0) to [R, l=$R_5 \ 510\Omega$] (1,2) 
(-1,2) to (3,2) 
(3,2) to [R, l=$R_2 \ 510\Omega$] (6,2) 
(6,2) to [V,v=$V_1 \ 5V$] (6,0) 
(6,0) to [R, l=$R_3 \ 100\Omega$] (3,0) 
(3,0) to (-1,0) --
(3,0) to [R, l=$R_1 \ 220\Omega$] (3,2)


;
\end{circuitikz}
\caption{zadany obwód, wraz z naniesionymi oporami}
\end{figure}

Szukane prądy to $I_{R4}$, $I_{R5}$, oraz $I_{R1}$ 

\pagebreak
\section{Pomiar}
\subsection{Dobór elementów rezystancyjnych}

W tablicy 1 spisaliśmy elementy wybrane do zbudowania obwodu na płytce prototypowej. Różnice między wartością znamionową a pomiarem oporu, dokonanym multimetrem RIGOL, wynikają z niedoskonałości elementów oraz środowiska pomiaru i znajdują się w granicach błędu pomiarowego
\begin{table}[H]
\centering
\begin{tabular}{|c|c|c|c|c|}
\hline
 Lp.& R & Kod\hspace{0.1cm} paskowy (KP) & Wartość\hspace{0.1cm} odczytana\hspace{0.1cm} z\hspace{0.1cm} KP & Pomiar\\
\hline 
1 & $R_{1}$ & czerwony, czerwony, brązowy, złoty & $220\Omega\pm 5\%$ & $217.87\Omega$\\
\hline
2&$R_{2}$& zielony, brązowy, brązowy, złoty & $510\Omega\pm 5\%$ & $498.06\Omega$ \\
\hline
3 & $R_{3}$& brązowy, czarny, brązowy, złoty&$100\Omega\pm 5\%$ & $98.84\Omega$\\
\hline
4 & $R_{4}$& brązowy, czarny, brązowy, złoty&$100\Omega\pm 5\%$ & $97.32\Omega$\\
\hline
5 & $R_{5}$& zielony, brązowy, brązowy, złoty & $510\Omega\pm 5\%$ & $499.88\Omega$ \\
\hline
\end{tabular}
\caption{elementy rezystancyjne wybrane do realizacji obwodu}
\end{table}

\subsection{Pomiar napięć oraz rezystancji zastępczej}
Pomiary dokonaliśmy zgodnie z ,,Algorytmem stosowania twierdzenia Thevenina tzw. metodą laboratoryjną''.\cite{naum} Za stabilne źródło prądu stałego w trakcie przeprowadzania pomiarów posłużył zasilacz  stałoprądowy, będący elementem uniwersalnego urządzenia zasilającego NDN DF6911, ustawiony na napięcie $5V$. Na pomiar napięć i rezystancji układu pozwolił multimetr RIGOL Otrzymane wyniki pomiarów zebrane są w tablicy 2.
\begin{center}
$U_{th}$ - napięcie\hspace{0.15cm} panujące\hspace{0.15cm} od\hspace{0.15cm} strony\hspace{0.15cm} zacisków\hspace{0.15cm} AB\\
$R_{th}$ - rezystancja\hspace{0.15cm} zastępcza\hspace{0.15cm} widziana\hspace{0.15cm} od\hspace{0.15cm} strony\hspace{0.15cm} zacisków\hspace{0.15cm} AB
\end{center}

\begin{table}[H]
\centering
$$
\begin{array}{|c|c|c|}
\hline
 Lp.& U_{th} & R_{th}\\
\hline 
1 & 1.0523V & 121.908\Omega\\
\hline
 2&0.533V & 61.63\Omega\\
\hline
3 & 0.634V & 72.71\Omega\\
\hline
\end{array}
$$
\caption{Wyniki pomiarów dla twierdzenia Thevenina}
\end{table}
\pagebreak
\section{Obliczenia}
\subsection{Obliczanie zadanych prądów w oparciu o wyniki pomiarów oraz twierdzenie Thevenina} 
Głównym zadaniem laboratoryjnym było obliczenie zadanych prądów w gałęzi z rezystorem $R_x$ w oparciu o twierdzenie Thevenina. Korzystając z zależności oraz poznanych wzorów\cite{naum}, a także z pomocniczego rysunku 2, dokonujemy niezbędnych obliczeń.

\begin{figure}[H]
\centering
\begin{circuitikz}[american voltages]
\draw

(-1,1) to  (-1,0.5)
(-1,0.5) to [V,v=$U \ th$ ] (-1,-0.5)
(-1,-1) to (-1,-0.5)
(-1,1) to (0,1)
(0,1) to [R, l=$R \ th$] (2,1)
(-1,-1) to (3,-1)
(1,-1) to (-1,-1) 
(2,1) to (3,1)
(3,1) to [R, l=$R \ x$](3,-1)
;
\end{circuitikz}
\caption{Rysunek pomocniczy do prowadzonych obliczeń}
\end{figure}
$$ I=\frac{U_{th}}{R_{th}+R_{x}}$$

\subsubsection{gałąź $I_{R4}$}
$$ I_{R4}=\frac{U_{th}}{R_{th}+R_{4}}=\frac{1.0523V}{121.908\Omega+100\Omega}=4.742mA$$

\subsubsection{gałąź $I_{R5}$}
$$ I_{R5}=\frac{U_{th}}{R_{th}+R_{5}}=\frac{0.533V}{61.63\Omega+510\Omega}=0.932mA$$

\subsubsection{gałąź $I_{R1}$}
$$ I_{R1}=\frac{U_{th}}{R_{th}+R_{1}}=\frac{0.635V}{72.71\Omega+220\Omega}=2.169mA$$

\subsection{Zestawienie dotychczasowych pomiarów oraz obliczeń}
Poniższe zestawienie, widoczne w tablicy 3. podsumowuje dotychczasowe działania, tym samym zamykając część zadania opartą na twierdzeniu Thevenina. W kolejnych punktach postaramy się pokazać, że nasze rozumowanie było prawidłowe i doprowadziło do prawidłowych wyników.
\begin{table}[H]
$$
\begin{array}{|c|c|c|c|}
\hline
 Lp.& U_{th} & R_{th}& I_{R_{x}}\\
\hline 
1 & 1.0523V & 121.908\Omega&4.742mA\\
\hline
 2&0.533V & 61.63\Omega&0.932mA\\
\hline
3 & 0.634V & 72.71\Omega&2.169mA\\
\hline
\end{array}
$$
\caption{Zestawienie wyników pomiarów, oraz opartych na nich obliczeń}
\end{table}
\subsection{}
Analitycznie obliczyć zadane wartości szukanych prądów.

$$I_{1} - \textrm{prąd płynący przez rezystor }R_{1}$$ 
$$I_{4} - \textrm{prąd płynący przez rezystor }R_{4}$$ 
$$I_{5} - \textrm{prąd płynący przez rezystor }R_{5}$$ 
$$R_{z} - \textrm{rezystancja zastępcza }$$ 
\newline
$$R_{z}=R_{2}+R_{3}+\frac{R_{1}*R_{4}*R_{5}}{R_{4}*R_{5}+R_{1}*R_{5}+R_{1}*R_{4}}=670.58\Omega$$
$$I=\frac{U}{R_{z}}=7.456mA$$
Z pierwszego prawa Kirchhoffa 
$$I=I_{1}+I_{4}+I_{5}$$
Z drugiego prawa Kirchhoffa 
$$U_{1}=U_{4}=U_{5}$$
$$I_{1}*R_{1}=I_{4}*R_{4}=I_{5}*R_{5}$$
\newline

$$
\left\{ \begin{array}{l}
I=I_{1}+I_{4}+I_{5}\\
I_{1}*R_{1}=I_{4}*R_{4}\\
I_{1}*R_{1}=I_{5}*R_{5}
\end{array} \right.
$$\newline
$$
\left\{ \begin{array}{l}
I=I_{1}+I_{4}+I_{5}\\
I_{4}=\frac{R_{1}}{R_{4}}*I_{1}\\
I_{5}=\frac{R_{1}}{R_{5}}*I_{1}
\end{array} \right.
$$\newline
$$
\left\{ \begin{array}{l}
7,456=I_{1}+2,2*I_{1}+0,43*I_{1}\\
I_{4}=2,2*I_{1}\\
I_{5}=0,43*I_{1}
\end{array} \right.
$$\newline
$$
\left\{ \begin{array}{l}
I_{1}=2.054mA\\
I_{4}=4.519mA\\
I_{5}=0.883mA
\end{array} \right.
$$

\section{Wnioski}
Tablica 4. jest zestawieniem wyników, jakie uzyskaliśmy, korzystając z dwóch różnych metod obliczeń: z równania Thevenina, oraz II prawa Kirchhoffa. Wyniki są równe w granicy błędu pomiarowego, wynikającego z oporów przewodników czy tolerowanego błędu elementów rezystancyjnych. Wykonane doświadczenie pozwala nam więc sądzić, że użyte metody obliczeń można uznać za równoważne w liczeniu charakterystyki obwodów stałoprądowych.

\begin{table}[H]
\centering
\begin{tabular}{|c|c|c|}
\hline
 Lp.& $I_{R_{x}}$ (z tw. Thevenina) & $I_{Rx}$ (z obliczeń) \\
\hline 
1 & $4.742mA$ & $4.519mA$\\
\hline
 2&$0.932mA$ & $0.883mA$\\
\hline
3 & $2.169mA$ & $2.054mA$\\
\hline
\end{tabular}

\caption{Zestawienie wyników uzyskanych na drodze pomiarów i obliczeń}
\end{table}
\newpage

\begin{thebibliography}{3}
\bibitem{naum} http://mariusznaumowicz.ddns.net/materialy.html - materiały dostępne dla przedmiotu Podstawy Elektroniki
\bibitem{pto}Podstawy teorii obwodów, Osiowski J., Szabatin J., WNT, Warszawa, 1998
\bibitem{pe}"Podstawy Elektrotechniki", R.Kurdziel, wyd II, WNT Warszawa 1972
\end{thebibliography}

\newpage
\tableofcontents{}
\end{document}
