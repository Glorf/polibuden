\documentclass[8pt]{extarticle}
\usepackage{geometry}
 \geometry{
 a4paper,
 left=2mm,
 top=2mm,
 right=2mm,
 bottom=2mm,
 }
\usepackage[english,main=polish]{babel}
\usepackage{polski}
\usepackage[utf8]{inputenc}
\usepackage[T1]{fontenc}
\usepackage{graphicx}
\usepackage{geometry}
\usepackage{float}
\usepackage{enumitem}
\setitemize{noitemsep,topsep=0pt,parsep=0pt,partopsep=0pt,leftmargin=7pt}
\setlength{\itemindent}{0.1in}
\usepackage{multicol}
\setlength{\columnsep}{2mm}
\setlength{\columnseprule}{0.2pt}
\title{Ściąga SK}
\begin{document}
\begin{multicols*}{3}
\textbf{1. Nadawanie IP}
\begin{itemize}
\item ip addr show
\item ip addr add 192.168.1.30/24 dev p4p1
\item ip addr show p4p1
\item ifconfig p4p1 up
\item ip addr del 192.168.1.30/24 dev p4p1
\end{itemize}

\textbf{2. Sprawdzanie połączenia}
\begin{itemize}
\item ping 192.168.1.100
\item traceroute -I 192.168.1.100
\end{itemize}

\textbf{3. Zmiana ustawień medium}
\begin{itemize}
\item ifconfig eth0 down
\item ethtool eth0
\item ethtool -s eth0 speed 10 duplex half autoneg off
\item ethtool eth0 up
\end{itemize}

\textbf{4. Adresacja MAC}
\begin{itemize}
\item ip link show
\item ip link set dev eth0 down
\item ip link set dev eth0 address aa:bb:cc:dd:ee:ff
\item ip link set dev eth0 up
\item ethtool -P eth0 \emph{(permanent address MAC)}
\end{itemize}

\textbf{5. Pomiar prędkości}
\begin{itemize}
\item ping 192.168.1.2
\item ethtool p4p1
\item netserver -D \emph{(jako serwer)}
\item netperf -H 192.168.1.2 \emph{(jako klient)}
\end{itemize}

\textbf{6. ARP}
\begin{itemize}
\item ip neigh show
\item ping 192.168.1.2
\item ip neigh add 192.168.1.100 lladdr \emph{MAC} dev eth0
\item ip link set arp off dev eth0 \emph{włącza/wyłącza ARP}
\item arping -I eth0 192.168.1.103 \emph{ARP REQUEST}
\end{itemize}

\textbf{7. Routing statyczny}
\begin{itemize}
\item ip route show
\item ip route add 172.16.0.0/16 via 192.168.1.101
\item ip route add default via 192.168.1.1
\item ip route get 192.168.1.1 \emph{trasa do adresu}
\item ip route del 172.16.0.0/16
\item ip route flush \emph{wyczyszczenie tablicy routingu}
\end{itemize}

\textbf{8. Forwarding}
\begin{itemize}
\item cat /proc/sys/net/ipv4/ip\_forward
\item echo 1 > /proc/sys/net/ipv4/ip\_forward
\end{itemize}

\textbf{9. Połączenie z konsolą Cisco}
\begin{itemize}
\item Kabel konsolowy (niebieski)
\item picocom [-b 9600] /dev/ttyS0
\item ctrl -a -q \emph{wyjście z programu}
\item Router> \emph{tryb użytkownika}
\item Router> enable
\item Router\# \emph{tryb uprzywilejowany}
\item Router\# configure terminal
\item Router(config)\# \emph{tryb konfiguracyjny}
\item Router(config)\# do ping \emph{komenda niższego trybu}
\item Router(config)\# interface FastEthernet 0/1
\item Router(config-if)\# 
\item Router(config)\# exit
\item Router\# disable
\item Router>
\item Router\# show ip int 
\item Router\# do ping 192.168.0.1 repeat 100000
\end{itemize}
\end{multicols*}
\end{document}

