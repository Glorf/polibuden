\documentclass[10pt,a4paper]{article}
\usepackage[utf8]{inputenc}
\usepackage[polish]{babel}
\usepackage{polski}
\usepackage{minted}
\usepackage{xcolor}
\title{Prolog}
\author{}
\date{}
\begin{document}
\maketitle
\section{Listy}
\subsection{Teoria}
Listy Prologowe określamy np [X,Y,Z], głowę odcinamy [H$|$T]
\subsection{Zadania}
\begin{enumerate}
\item Zwróć ostatni element listy
\item Kopiowanie listy
\item Dodawanie elementu na koniec listy
\item Kopiowanie bez ostatniego elementu
\item Kopiowanie bez przedostatniego elementu
\end{enumerate}
\subsection{Rozwiązania}
\begin{enumerate}
\item
\begin{minted}[bgcolor=lightgray]{prolog}
ostatni([H],H).
ostatni([H|T], X):-ostatni(T, X).
\end{minted}
\item
\begin{minted}[bgcolor=lightgray]{prolog}
kopia([],[]).
kopia([H|T], [H|T2]):-kopia(T,T2).
\end{minted}
\item
\begin{minted}[bgcolor=lightgray]{prolog}
nakoniec([], E, [E]).
nakoniec([H|T], E, [H|T2]):-nakoniec(T, E, T2).
\end{minted}
\item
\begin{minted}[bgcolor=lightgray]{prolog}
bezostat([_], []).
bezostat([H|T], [H|T2]):-bezostat(T,T2).
\end{minted}
\item
\begin{minted}[bgcolor=lightgray]{prolog}
bezpostat([_,T],[T]).
bezpostat([H|T], [H|T2]):-bezpostat(T,T2).
\end{minted}
\end{enumerate}
\section{Konkatenacja append}
\subsection{Teoria}
Predykat append/3 przyjmuje dwa pierwsze argumenty: listy, i zwraca trzeci: połączone listy
np.
\begin{minted}[bgcolor=lightgray]{prolog}
?-append([1,2],[3,4],X).
X=[1,2,3,4]
\end{minted}
\subsection{Zadania}
\begin{enumerate}
\item Dodawanie elementu na koniec listy
\item Kopiowanie listy
\item Kopiowanie bez ostatniego elementu
\item Kopiowanie bez przedostatniego elementu
\item Odwracanie listy
\item Zwracanie środkowego elementu listy
\item Rozbijanie listy na dwie połowy
\end{enumerate}
\subsection{Rozwiązania}
\begin{enumerate}
\item
\item
\item
\item
\item
\item
\item
\end{enumerate}
\section{Operacje na liczbach}
\subsection{Teoria}
\begin{itemize}
\item Przypisanie wartości: X is 1+2
\textbf{Nie mylić z ewaluacją! X = 1+2}
\item Część całkowita z dzielenia: X // 2
\item Reszta całkowita z dzielenia: X mod 2
\end{itemize}
\subsection{Zadania}
\begin{enumerate}
\item Sumowanie elementów tablicy
\end{enumerate}
\subsection{Rozwiązania}
\begin{enumerate}
\item
\end{enumerate}
\section{Operacje boolowskie}
\subsection{Teoria}
Operatory zwracające true lub false. Jeśli są częścią modelu i zwrócą false, to model nie zostanie spełniony.
Można stosować notację prefiksową i infiksową, czyli 1 $<$ 3 lub $<(1,3)$
\begin{itemize}
\item większy, mniejszy: $>, <$
\item większy równy: $>=$
\item mniejszy równy: $=<$ \textbf{Najłatwiej je zapamiętać że mają smutne miny}
\item równy: =:=
\item różny: =\textbackslash=
\item brak unifikacji: \textbackslash=
\end{itemize}
\subsection{Zadania}
\begin{enumerate}
\item Czy lista jest rosnąca
\item Czy lista tworzy ciąg arytmetyczny
\item Podziel listę na listy parzystych i nieparzystych liczb
\item Skopiuj n pierwszych elementów listy
\item Twórz listę z dwóch list przeplatających się co n elementów
\end{enumerate}
\subsection{Rozwiązania}
\begin{enumerate}
\item
\item
\item
\item
\item
\end{enumerate}
\end{document}