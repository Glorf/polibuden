\documentclass[polish,a4paper]{article}
\usepackage{amsmath}
\usepackage{amssymb,amsfonts,amsthm}
\usepackage[english,main=polish]{babel}
\usepackage{polski}
\usepackage[utf8]{inputenc}
\usepackage[T1]{fontenc}
\usepackage{graphicx}
\usepackage{geometry}
\usepackage{tikz}
\usepackage{circuitikz}
\usepackage{float}
\usepackage{etoolbox}
\usepackage{pgfplots}
\usepackage{enumerate}
\patchcmd{\thebibliography}{\section*}{\section}{}{}

\selectlanguage{polish}
\title{Lab4}
\newgeometry{tmargin=3cm, bmargin=3cm, lmargin=2cm, rmargin=2cm}

\newcommand{\PRzFieldDsc}[1]{\sffamily\bfseries\scriptsize #1}
\newcommand{\PRzFieldCnt}[1]{\textit{#1}}
\newcommand{\PRzHeading}[8]{
\begin{center}
\begin{tabular}{ p{0.32\textwidth} p{0.15\textwidth} p{0.15\textwidth} p{0.12\textwidth} p{0.12\textwidth} }

  &   &   &   &   \\
\hline
\multicolumn{5}{|c|}{}\\[-1ex]
\multicolumn{5}{|c|}{{\LARGE #1}}\\
\multicolumn{5}{|c|}{}\\[-1ex]

\hline
\multicolumn{2}{|l|}{\PRzFieldDsc{Kierunek}}		& \multicolumn{1}{|l|}{\PRzFieldDsc{Rok studiów}}	& \multicolumn{2}{|l|}{\PRzFieldDsc{Grupa}} \\
\multicolumn{2}{|c|}{\PRzFieldCnt{#2}}				& \multicolumn{1}{|c|}{\PRzFieldCnt{#4}}		& \multicolumn{2}{|c|}{\PRzFieldCnt{#5}} \\

\hline
\multicolumn{3}{|l|}{\PRzFieldDsc{}}		& \multicolumn{2}{|l|}{\PRzFieldDsc{Data oddania}} \\
\multicolumn{3}{|c|}{\PRzFieldCnt{#6}}				& \multicolumn{2}{|c|}{\PRzFieldCnt{#7}} \\

\hline
\multicolumn{5}{|l|}{\PRzFieldDsc{Imiona, nazwiska, numery indeksu}}\\
\multicolumn{5}{|c|}{\PRzFieldCnt{#8}}\\



\hline
\end{tabular}
\end{center}
}
\pgfplotsset{compat=1.14}
\begin{document}
\PRzHeading{Komunikacja człowiek-komputer}{Informatyka}{--}{III}{I2 piątek 9.45-11.15}{Sygnały}{11.01.2019}{Martyna Maciejewska(132273), Michał Bień (132191), Mikołaj Wolicki (132344)}{}


\section{Wstęp}


\section{Autokorelacja}
Autokorelacja jest to korelacja pomiędzy kolejnymi wartościami tej samej zmiennej.
\newline
Znajdowanie częstotliwości podstawowej za pomocą autokorelacji. Przed jej obliczeniem należy usunąć z sygnału składową stałą, poprzez odjęcie od sygnału jego wartości średniej. Operację tę nazywa się usuwaniem trendu. Po obliczeniu autokorelacji należy wyznaczyć indeks pierwszego "pełnego" maksimum funkcji. Częstotliwość podstawową obliczamy z równania :
$$f = \frac{f_{s}}{m}$$
gdzie m - indeks wyznaczający maksimum funkcji, 
\newline
$f_{s}$ - częstotliwość próbkowania sygnału
\section{HPS}
Celem HPS jest znalezienie podstawowej częstotliwości, która jest pierwszym
występującym pikiem na wykresie FFT poprzez pomiar maksymalnej koincydencji dla każdej ramki. Na podstawie wzoru:
\[Y(\omega )=\prod_{r=1}^{R}\left | X(\omega r) \right |\]
\newline
Następnie ciąg jest następnie przeszukiwany pod kątem maksymalnej wartości możliwych częstotliwości podstawowych.


\section{Podsumowanie}
Nasze algorytmy poddaliśmy sprawdzeniu na udostępnionych ścieżkach dzwiękowych. Do testu wybraliśmy losowo 50 plików dzwiękowych, w celu udoskonalenia parametrów odpowiedzialnych za określenie płci.

\begin{itemize}
  \item HPS:  Zgodność płci na poziomie 82 \%
  \item Autokorelacja: Zgodność płci na poziomie 78 \%
\end{itemize}

Powyższe rezultaty testów uznaliśmy za odpowienie i zaprzestaliśmy dalszego polepszania parametrów.


\end{document}

