\documentclass[polish,a4paper]{article}
\usepackage{amsmath}
\usepackage{amssymb,amsfonts,amsthm}
\usepackage[english,main=polish]{babel}
\usepackage{polski}
\usepackage[utf8]{inputenc}
\usepackage[T1]{fontenc}
\usepackage{float}
\usepackage{etoolbox}
\usepackage{pgfplots}
\usepackage{gensymb}
\usepackage{adjustbox}
\usepackage{graphicx}
\patchcmd{\thebibliography}{\section*}{\section}{}{}

\selectlanguage{polish}
\title{FizLab6}

\newcommand{\PRzFieldDsc}[1]{\sffamily\bfseries\scriptsize #1}
\newcommand{\PRzFieldCnt}[1]{\textit{#1}}
\newcommand{\PRzHeading}[8]{
\begin{center}
\begin{tabular}{ p{0.32\textwidth} p{0.15\textwidth} p{0.15\textwidth} p{0.12\textwidth} p{0.12\textwidth} }

  &   &   &   &   \\
\hline
\multicolumn{5}{|c|}{}\\[-1ex]
\multicolumn{5}{|c|}{{\LARGE #1}}\\
\multicolumn{5}{|c|}{}\\[-1ex]

\hline
\multicolumn{1}{|l|}{\PRzFieldDsc{Kierunek}}	& \multicolumn{1}{|l|}{\PRzFieldDsc{Specjalność}}	& \multicolumn{1}{|l|}{\PRzFieldDsc{Rok studiów}}	& \multicolumn{2}{|l|}{\PRzFieldDsc{Symbol grupy lab.}} \\
\multicolumn{1}{|c|}{\PRzFieldCnt{#2}}		& \multicolumn{1}{|c|}{\PRzFieldCnt{#3}}		& \multicolumn{1}{|c|}{\PRzFieldCnt{#4}}		& \multicolumn{2}{|c|}{\PRzFieldCnt{#5}} \\

\hline
\multicolumn{4}{|l|}{\PRzFieldDsc{Temat Laboratorium}}		& \multicolumn{1}{|l|}{\PRzFieldDsc{Numer lab.}} \\
\multicolumn{4}{|c|}{\PRzFieldCnt{#6}}				& \multicolumn{1}{|c|}{\PRzFieldCnt{#7}} \\

\hline
\multicolumn{5}{|l|}{\PRzFieldDsc{Skład grupy ćwiczeniowej oraz numery indeksów}}\\
\multicolumn{5}{|c|}{\PRzFieldCnt{#8}}\\

\hline
\multicolumn{3}{|l|}{\PRzFieldDsc{Uwagi}}	& \multicolumn{2}{|l|}{\PRzFieldDsc{Ocena}} \\
\multicolumn{3}{|c|}{\PRzFieldCnt{\ }}		& \multicolumn{2}{|c|}{\PRzFieldCnt{\ }} \\

\hline
\end{tabular}
\end{center}
}
\pgfplotsset{compat=1.14}
\begin{document}
\PRzHeading{Laboratorium Fizyczne}{Informatyka}{--}{II}{1}{ Wyznaczanie ogniskowych soczewek}{302}{Michał Bień(132191), Wojciech Taisner(132330)}{}

\section{Wstęp Teoretyczny}
\begin{quotation}\cite{szuba}
\emph{Soczewką} nazywamy ciało przezroczyste ogrnaiczone dwoma powierzchniami sferycznymi. Oś łączącą środki krzywizny obu powierzchni nazywamy \emph{osią optyczną} soczewki. Światło przechodzące przez soczewkę ulega kolejno załamaniu na obu jej powierzchniach.\\
Wiązka promieni biegnąca równolegle do osi optycznej po przejściu przez soczewkę skupia się w jednym punkcie zwanym \emph{ogniskiem}. Odległość ogniska od środka soczewki nazywamy \emph{ogniskową}.\\
Soczewki mają zdolność odwzorowywania punktów polegającą na tym, że promienie wybiegające z punktu P, zwanego przedmiotem, zostają skupione po przejściu przez soczewkę w punkcie O, tworząc obraz przedmiotu. Położenie obrazu zależy od położenia przedmiotu, oraz od ogniskowej soczewki -- jest określone tzw.\emph{równaniem soczewkowym}: $$\frac{1}{p} + \frac{1}{o} = \frac{1}{f}$$gdzie: p jest odległością przedmiotu od soczewki; o jest odległością obrazu od soczewki.\\
Ogniskową układu składającego się z dwóch cienkich soczewek o ogniskowych $f_1$ i $f_2$, znajdujących się we wzajemnej odległości d, wyraża wzór: $$\frac{1}{f} = \frac{1}{f_1} + \frac{1}{f_2} - \frac{d}{f_1f_2}$$ Fizyczną konsekwencją symetrii równania soczewkowego jest możliwość uzyskania ostrego obrazu przy dwóch położeniach soczewki względem przedmiotu. Przy stałej odległości l przedmiotu od ekranu, oraz zmierzonej odległości e pomiędzy dwoma pozycjami soczewki, przy których obraz na ekranie jest ostry, jej ogniskową policzyć można tzw. \emph{metodą Bessela}, korzystając ze wzoru: $$ f = \frac{l^2-e^2}{4l} $$
\end{quotation}


\section{Tabela wyników}
Wszystie wartości podane są w centymetrach
\subsection{Pomiary soczewek skupiających}
\begin{table}[H]
\centering
\begin{tabular}{|c|c|c|c|c|c|c|c|c|}
\hline
Soczewka &  Xo &  Xe &  X1 &  X1przsoc &  X1obrsoc &  X2 &  X2przsoc &  X2obrsoc \\
\hline
C &  0 &  100 &  28,1 &  28,1 &  71,9 &  72 &  71,9 &  28 \\
C &  0 &  95 &  28,9 &  28,9 &  66,1 &  65,7 &  66,1 &  29,3 \\
C &  0 &  90 &  30,5 &  30,5 &  59,5 &  59,3 &  59,5 &  30,7 \\
C &  0 &  85 &  32,4 &  32,4 &  52,6 &  52,3 &  52,6 &  32,7 \\
C &  0 &  82 &  34,8 &  34,8 &  47,2 &  46,4 &  47,2 &  35,6 \\
C &  0 &  80 &  36,1 &  36,1 &  43,9 &  44 &  43,9 &  36 \\
\hline
B &  0 &  100 &  17,5 &  17,5 &  82,5 &  82,7 &  82,5 &  17,3 \\
B &  0 &  95 &  17,4 &  17,4 &  77,6 &  77,5 &  77,6 &  17,5 \\
B &  0 &  90 &  17,6 &  17,6 &  72,4 &  72 &  72,4 &  18 \\
B &  0 &  85 &  18,1 &  18,1 &  66,9 &  66,6 &  66,9 &  18,4 \\
B &  0 &  82 &  18,3 &  18,3 &  63,7 &  63,5 &  63,7 &  18,5 \\
B &  0 &  80 &  18,6 &  18,6 &  61,4 &  61,4 &  61,4 &  18,6 \\
\hline
A &  0 &  100 &  10,9 &  10,9 &  89,1 &  89,1 &  89,1 &  10,9 \\
A &  0 &  95 &  11,2 &  11,2 &  83,8 &  84,1 &  83,8 &  10,9 \\
A &  0 &  90 &  11,2 &  11,2 &  78,8 &  78,7 &  78,8 &  11,3 \\
A &  0 &  85 &  11,3 &  11,3 &  73,7 &  73,7 &  73,7 &  11,3 \\
A &  0 &  82 &  11,3 &  11,3 &  70,7 &  70,7 &  70,7 &  11,3 \\
A &  0 &  80 &  11,5 &  11,5 &  68,5 &  68,6 &  68,5 &  11,4 \\
\hline
\end{tabular}
\caption{Pomiary soczewek skupiających}
\end{table}
\subsection{Obliczanie ogniskowych soczewek skupiających}
\begin{table}[H]
\centering
\begin{tabular}{|c|c|c|c|c|c|c|}
\hline
Soczewka &  f1 socz &  f2 socz &  fsr socz &  delta oe &  delta X &  bessel \\
\hline
C &  20,2039 &  20,15215215 &  20,17802608 &  100 &  43,9 &  20,181975 \\
C &  20,10831579 &  20,30115304 &  20,20473441 &  95 &  36,8 &  20,18621053 \\
C &  20,16388889 &  20,25110865 &  20,20749877 &  90 &  28,8 &  20,196 \\
C &  20,04988235 &  20,16436108 &  20,10712172 &  85 &  19,9 &  20,08526471 \\
C &  20,03121951 &  20,29371981 &  20,16246966 &  82 &  11,6 &  20,0897561 \\
C &  19,809875 &  19,77972466 &  19,79479983 &  80 &  7,9 &  19,80496875 \\
\hline
B &  14,4375 &  14,3011022 &  14,3693011 &  100 &  65,2 &  14,3724 \\
B &  14,21305263 &  14,27970557 &  14,2463791 &  95 &  60,1 &  14,24471053 \\
B &  14,15822222 &  14,4159292 &  14,28707571 &  90 &  54,4 &  14,27955556 \\
B &  14,24576471 &  14,43094959 &  14,33835715 &  85 &  48,5 &  14,33161765 \\
B &  14,21597561 &  14,3363747 &  14,27617515 &  82 &  45,2 &  14,27121951 \\
B &  14,2755 &  14,2755 &  14,2755 &  80 &  42,8 &  14,2755 \\
\hline
A &  9,7119 &  9,7119 &  9,7119 &  100 &  78,2 &  9,7119 \\
A &  9,879578947 &  9,645406547 &  9,762492747 &  95 &  72,9 &  9,764710526 \\
A &  9,806222222 &  9,882796892 &  9,844509557 &  90 &  67,5 &  9,84375 \\
A &  9,797764706 &  9,797764706 &  9,797764706 &  85 &  62,4 &  9,797764706 \\
A &  9,742804878 &  9,742804878 &  9,742804878 &  82 &  59,4 &  9,742804878 \\
A &  9,846875 &  9,773466834 &  9,810170917 &  80 &  57,1 &  9,81121875 \\
\hline
\end{tabular}
\caption{Obliczanie ogniskowych soczewek skupiających}
\end{table}
\subsection{Zestawienie wyników}
\subsubsection{Wyniki zwykłe}
\begin{table}[H]
\centering
\begin{tabular}{|c|c|c|c|c|}
\hline
Soczewka &  f śr (równanie) &  Odch. S. &  f m.Bessela &  Odch. S. \\
\hline
C &  20,10910841 &  0,158250368 &  20,09069585 &  0,148468351 \\
B &  14,29879804 &  0,045771949 &  14,29583387 &  0,046990986 \\
A &  9,778273801 &  0,048367477 &  9,778691477 &  0,048163969 \\
\hline
\end{tabular}
\caption{Zestawienie wyników}
\end{table}
\subsubsection{Wyniki zaokrąglone} 
\begin{table}[H]
\centering
\begin{tabular}{|c|c|c|c|c|}
\hline
Soczewka &  f śr (równanie) &  Odch. S. &  f m.Bessela &  Odch. S. \\
\hline
C &  20,11 &  0,16 &  20,09 &  0,15 \\
B &  14,29 &  0,05 &  14,29 &  0,05 \\
A &  9,78 &  0,05 &  9,78 &  0,05 \\
\hline
\end{tabular}
\caption{Zestawienie wyników}
\end{table}

\subsection{Zestawienie pomiarów układów soczewek}
\begin{table}[H]
\centering
\begin{tabular}{|c|c|c|c|c|c|c|c|}
\hline
S.Skup.&  Ogniskowa &  S.Rozpr. &  odleg od 0 &  Xp &  Xe &  X1 &  X2 \\
\hline
A &  9,778482639 &  1 &  2 &  0 &  100 &  27,5 &  84,5 \\
A &  9,778482639 &  1 &  2 &  0 &  95 &  27,7 &  78,7 \\
A &  9,778482639 &  1 &  2 &  0 &  90 &  28 &  73,5 \\
A &  9,778482639 &  1 &  2 &  0 &  85 &  28,2 &  67 \\
A &  9,778482639 &  1 &  2 &  0 &  82 &  29,1 &  64,1 \\
A &  9,778482639 &  1 &  2 &  0 &  80 &  29,2 &  62,1 \\
\hline
B &  14,29731595 &  2 &  2 &  0 &  100 &  28 &  64 \\
B &  14,29731595 &  2 &  2 &  0 &  95 &  29,6 &  57,2 \\
B &  14,29731595 &  2 &  2 &  0 &  92 &  30,5 &  53 \\
B &  14,29731595 &  2 &  2 &  0 &  90 &  32 &  50 \\
B &  14,29731595 &  2 &  2 &  0 &  87 &  33,4 &  44,2 \\
B &  14,29731595 &  2 &  2 &  0 &  85 &  35 &  41 \\
\hline
B &  14,29731595 &  3 &  5 &  0 &  100 &  14 &  72 \\
B &  14,29731595 &  3 &  5 &  0 &  95 &  14,1 &  66,6 \\
B &  14,29731595 &  3 &  5 &  0 &  90 &  14,5 &  61,1 \\
B &  14,29731595 &  3 &  5 &  0 &  85 &  15 &  55,9 \\
B &  14,29731595 &  3 &  5 &  0 &  82 &  15,7 &  52,4 \\
B &  14,29731595 &  3 &  5 &  0 &  80 &  15,7 &  49,4 \\
\hline
\end{tabular}
\caption{Zestawienie pomiarów układów soczewek}
\end{table}
\subsection{Obliczanie ogniskowych układów soczewek i ogniskowych soczewek rozpraszajacych}
\begin{table}[H]
\centering
\begin{tabular}{|c|c|c|c|c|c|c|}
\hline
Układ &  m.Bessela &  f1 ukł &  f2 ukł &  fśr ukł (równanie) &  f S.Rozpr. (m.Bessela) \\
\hline
A1 &  16,8775 &  19,9375 &  13,0975 &  16,5175 &  -13,73800567 \\
A1 &  16,90526316 &  19,62326316 &  13,50326316 &  16,56326316 &  -13,70699847 \\
A1 &  16,74930556 &  19,28888889 &  13,475 &  16,38194444 &  -13,88438245 \\
A1 &  16,82223529 &  18,84423529 &  14,18823529 &  16,51623529 &  -13,80045543 \\
A1 &  16,7652439 &  18,77304878 &  13,99256098 &  16,38280488 &  -13,8658911 \\
A1 &  16,61746875 &  18,542 &  13,894875 &  16,2184375 &  -14,04064186 \\
\hline
B2 &  21,76 &  20,16 &  23,04 &  21,6 &  -30,0253359 \\
B2 &  21,74536842 &  20,37726316 &  22,75957895 &  21,56842105 &  -30,06409128 \\
B2 &  21,62432065 &  20,38858696 &  22,4673913 &  21,42798913 &  -30,39065365 \\
B2 &  21,6 &  20,62222222 &  22,22222222 &  21,42222222 &  -30,45757194 \\
B2 &  21,41482759 &  20,57747126 &  21,74436782 &  21,16091954 &  -30,98207031 \\
B2 &  21,14411765 &  20,58823529 &  21,22352941 &  20,90588235 &  -31,79990743 \\
\hline
B3 &  16,59 &  12,04 &  20,16 &  16,1 &  -31,09563738 \\
B3 &  16,49671053 &  12,00726316 &  19,90989474 &  15,95857895 &  -32,23231442 \\
B3 &  16,46788889 &  12,16388889 &  19,61988889 &  15,89188889 &  -32,60324523 \\
B3 &  16,32997059 &  12,35294118 &  19,13752941 &  15,74523529 &  -34,52383991 \\
B3 &  16,39362805 &  12,69402439 &  18,91512195 &  15,80457317 &  -33,60596906 \\
B3 &  16,45096875 &  12,618875 &  18,8955 &  15,7571875 &  -32,82563032 \\
\hline
\end{tabular}
\caption{Obliczanie ogniskowych układów soczewek i ogniskowych soczewek rozpraszajacych}
\end{table}
\subsection{Zestawienie wyników}
\subsubsection{Wyniki zwykłe}
\begin{table}[H]
\centering
\begin{tabular}{|c|c|c|c|c|c|c|}
\hline
Układ &  fśr ukł (równ.) &  Odch. S. &  f m.Bessela &  Odch. S. &  f S.R. (m.Bessela) &  Odch. S. \\
\hline
A1 &  16,43003088 &  0,12817741 &  16,78950278 &  0,10393787 &  -13,83939583 &  0,12045840 \\
B2 &  21,34757238 &  0,26625651 &  21,54810572 &  0,23378747 &  -30,61993842 &  0,67304070 \\
B3 &  15,87624397 &  0,13679546 &  16,45486113 &  0,08893356 &  -32,81443939 &  1,17357977 \\
\hline
\end{tabular}
\caption{Zestawienie wyników}
\end{table}
\subsubsection{Wyniki zaokrąglone} 
\begin{table}[H]
\centering
\begin{tabular}{|c|c|c|c|c|c|c|}
\hline
Układ &  fśr ukł (równ.) &  Odch. S. &  f m.Bessela &  Odch. S. &  f S.Rozpr. (m.Bessela) &  Odch. S. \\
\hline
A1 &  16,43 &  0,13 &  16,8 &  0,1 &  -13,84 &  0,12 \\
B2 &  21,35 &  0,27 &  21,55 &  0,23 &  -30,6 &  0,7 \\
B3 &  15,88 &  0,14 &  16,45 &  0,09 &  -32,8 &  1,2 \\
\hline
\end{tabular}
\caption{Zestawienie wyników}
\end{table}

\section{Obliczenia}
\subsection{Obliczanie ogniskowych}
Obliczając ogniskowe soczewek skupiających użyliśmy klasycznego wzoru:
$$\frac{1}{p} + \frac{1}{o} = \frac{1}{f}$$
Obliczając ogniskowe soczewek skupiających i układów soczekek użyliśmy metody Bessela opisanej wzorem:
$$ f = \frac{l^2-e^2}{4l} $$
Wiedząc że:
$$o+p=l$$
$$o-p=e$$
Przy obliczaniu ogniskowej ukłądu soczewek, wiedząc że soczewki są umieszczone symetrycznie względem środka uchwytu, przyjęliśmy za punkt pomiaru środek uchwytu. 
Obliczyliśmy ogniskową soczewki rozpraszajacej z przekształocnego równania:
$$\frac{1}{f} = \frac{1}{f_1} + \frac{1}{f_2} - \frac{d}{f_1f_2}$$
$$f = \frac{f_1f_2}{f_1+f_2+d}$$
$$f_1 = \frac{f(d-f_2)}{f-f_2}$$
\subsection{Rachunek jednostek}
Wszystie wartości podane są w centymetrach:
$$\frac{1}{p} + \frac{1}{o} = \frac{1}{f}$$
$$\frac{1}{[cm]} + \frac{1}{[cm]} = \frac{1}{[cm]}$$
$$ f = \frac{l^2-e^2}{4l} $$
$$\frac{[cm]^2}{[cm]}=[cm]$$
$$\frac{1}{f} = \frac{1}{f_1} + \frac{1}{f_2} - \frac{d}{f_1f_2}$$
$$\frac{1}{[cm]} = \frac{1}{[cm]} + \frac{1}{[cm]} - \frac{[cm]}{[cm]^2}$$


\section{Wnioski}
Obliczanie ogniskowej soczewek metodą Bessela i równania soczewkowego to dwie równorzędne metody, dzięki którym uzyskać możemy długość ogniskowej. Z naszych doświadczeń oraz poziomu skomplikowania obliczeń wywnioskować można, że metoda Bessela jest dużo bardziej intuicyjna i przystępna obliczeniowo w przypadku, gdy znajdujemy w doświadczeniu dwa punkty w których obraz na obrazie jest ostry. Jeżeli dysponujemy tylko jednym takim punktem, równanie soczewkowe jest jedynym rozwiązaniem. W przypadku pojedynczej soczewki wykorzystanie równania soczewkowego nie jest to trudne, w przypadku układu - zaczyna być skomplikowane. Uzyskane obydwoma sposobami wyniki nie różnią się znacząco.

\begin{thebibliography}{1}
\bibitem{szuba} Stanisław Szuba, Ćwiczenia laboratoryjne z fizyki, Wydawnictwo Politechniki Poznańskiej, Poznań 2007
\end{thebibliography}
\end{document}

