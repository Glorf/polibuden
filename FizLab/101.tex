\documentclass[polish,a4paper]{article}
\usepackage{amsmath}
\usepackage{amssymb,amsfonts,amsthm}
\usepackage[english,main=polish]{babel}
\usepackage{polski}
\usepackage[utf8]{inputenc}
\usepackage[T1]{fontenc}
\usepackage{float}
\usepackage{etoolbox}
\usepackage{pgfplots}
\usepackage{gensymb}
\usepackage{adjustbox}
\patchcmd{\thebibliography}{\section*}{\section}{}{}

\selectlanguage{polish}
\title{FizLab1}

\newcommand{\PRzFieldDsc}[1]{\sffamily\bfseries\scriptsize #1}
\newcommand{\PRzFieldCnt}[1]{\textit{#1}}
\newcommand{\PRzHeading}[8]{
\begin{center}
\begin{tabular}{ p{0.32\textwidth} p{0.15\textwidth} p{0.15\textwidth} p{0.12\textwidth} p{0.12\textwidth} }

  &   &   &   &   \\
\hline
\multicolumn{5}{|c|}{}\\[-1ex]
\multicolumn{5}{|c|}{{\LARGE #1}}\\
\multicolumn{5}{|c|}{}\\[-1ex]

\hline
\multicolumn{1}{|l|}{\PRzFieldDsc{Kierunek}}	& \multicolumn{1}{|l|}{\PRzFieldDsc{Specjalność}}	& \multicolumn{1}{|l|}{\PRzFieldDsc{Rok studiów}}	& \multicolumn{2}{|l|}{\PRzFieldDsc{Symbol grupy lab.}} \\
\multicolumn{1}{|c|}{\PRzFieldCnt{#2}}		& \multicolumn{1}{|c|}{\PRzFieldCnt{#3}}		& \multicolumn{1}{|c|}{\PRzFieldCnt{#4}}		& \multicolumn{2}{|c|}{\PRzFieldCnt{#5}} \\

\hline
\multicolumn{4}{|l|}{\PRzFieldDsc{Temat Laboratorium}}		& \multicolumn{1}{|l|}{\PRzFieldDsc{Numer lab.}} \\
\multicolumn{4}{|c|}{\PRzFieldCnt{#6}}				& \multicolumn{1}{|c|}{\PRzFieldCnt{#7}} \\

\hline
\multicolumn{5}{|l|}{\PRzFieldDsc{Skład grupy ćwiczeniowej oraz numery indeksów}}\\
\multicolumn{5}{|c|}{\PRzFieldCnt{#8}}\\

\hline
\multicolumn{3}{|l|}{\PRzFieldDsc{Uwagi}}	& \multicolumn{2}{|l|}{\PRzFieldDsc{Ocena}} \\
\multicolumn{3}{|c|}{\PRzFieldCnt{\ }}		& \multicolumn{2}{|c|}{\PRzFieldCnt{\ }} \\

\hline
\end{tabular}
\end{center}
}
\pgfplotsset{compat=1.14}
\begin{document}
\PRzHeading{Laboratorium Fizyczne}{Informatyka}{--}{II}{1}{Wyznaczanie prędkości dźwięku w powietrzu
metodą przesunięcia fazowego }{101}{Michał Bień(132191), Wojciech Taisner(132330)}{}

\section{Wstęp Teoretyczny}
Celem doświadczenia było wyznaczenie prędkości dźwięku w powietrzu metodą przesunięcia fazowego. Związek ten został opisany równaniem:

$$ v= \lambda f $$

Częstotliwość jest mierzona bezpośrednio miernikiem, a długość fali wyznaczamy przeprowadzając doświadczenie.
\newline
Na jednym końcu ławy ustawiamy głośnik podłączony do generatora fal elektrycznych pełniacego funkcję źródła. Do ich odbioru używamy ruchomego mikrofonu przesuwanego wzdłuż ławy. Podłaćzamy głosnik i mikrofon do oscyloskopu, na którego ekranie otrzymujemy obraz w postaci tzw. figur Lissajous, które są wynikiem nałożenia się na siebie dwóch ruchów harmonicznych. Korzystając z tego, że w tym wypadku częstotliwości obu drgań są równe, o kształcie figur decyduje tylko różnica faz głośnika i mikrofonu. Kształt figury Lissajous jest periodyczną funkcją różnicy faz, stąd będzie on taki sam dla wszystkich położeń mikrofonu różniących się o całkowitą wielokrotność długości fali.
Tę ostatnią właściwość wykorzystujemy do pomiaru długości fali dźwiękowej. W tym celu wyznaczamy odległość między sąsiednimi położeniami mikrofonu, przy których otrzymujemy ten sam kształt figury Lissajous. Wybraliśmy moment, w którym figura Lissajous jest linią prostą przechyloną w lewo.

\subsection{Prędkość fali w powietrzu}
Ogólne wyrażenie określające prędkosć rozchodzenia się fal podłużnych w ośrodku ciągłym ma postać:

$$ v = \sqrt{\frac{E}{\rho}}$$

Po przekształceniach otrzymujemy wzór na prędkość dźwięku w zależności od rodzaju gazu i temperatury:

$$ v = \sqrt{\frac{\kappa RT}{\mu}}$$

\section{Tabela wyników}
\subsection{Częstotliwość: 3.050 kHz (dla linii wychylonej w górę)}

\begin{table}[H]
\centering
\begin{tabular}{|c|}
\hline
$\Delta [cm]$:\\
\hline 
13.1 \\
13.0 \\
12.9 \\
12.5 \\
13.1 \\
\hline
\end{tabular}
\caption{Tabela różnic}
\end{table}

$$\lambda = 12.92 cm$$
$$v(f) = \lambda f = 394.06 \frac{m}{s}$$

\subsection{Częstotliwość: 3.563 kHz (dla linii wychylonej w górę)}

\begin{table}[H]
\centering
\begin{tabular}{|c|}
\hline
$\Delta [cm]$:\\
\hline 
10.2 \\
11.0 \\
10.0 \\
10.8 \\
10.5 \\
10.0 \\
\hline
\end{tabular}
\caption{Tabela różnic}
\end{table}

$$\lambda = 10.42 cm$$
$$v(f) = \lambda f = 371.15 \frac{m}{s}$$

\subsection{Częstotliwość: 4.100 kHz (dla linii wychylonej w górę)}

\begin{table}[H]
\centering
\begin{tabular}{|c|}
\hline
$\Delta [cm]$:\\
\hline 
8.7 \\
9.0 \\
9.5 \\
8.5 \\
7.9 \\
9.7 \\
10.1 \\
\hline
\end{tabular}
\caption{Tabela różnic}
\end{table}

$$\lambda = 9.06 cm$$
$$v(f) = \lambda f = 371.34 \frac{m}{s}$$

\subsection{Częstotliwość: 4.628 kHz (dla linii wychylonej w górę)}

\begin{table}[H]
\centering
\begin{tabular}{|c|}
\hline
$\Delta [cm]$:\\
\hline 
8.0 \\
7.9 \\
7.6 \\
7.3 \\
7.6 \\
8.1 \\
8.1 \\
7.0 \\
8.5 \\
\hline
\end{tabular}
\caption{Tabela różnic}
\end{table}

$$\lambda = 7.79 cm$$
$$v(f) = \lambda f = 360.47 \frac{m}{s}$$

\subsection{Częstotliwość: 5.099 kHz (dla linii wychylonej w górę)}

\begin{table}[H]
\centering
\begin{tabular}{|c|}
\hline
$\Delta [cm]$:\\
\hline 
6.6 \\
7.3 \\
6.6 \\
7.1 \\
7.3 \\
7.3 \\
7.4 \\
7.0 \\
7.7 \\
7.0 \\
\hline
\end{tabular}
\caption{Tabela różnic}
\end{table}

$$\lambda = 7.13 cm$$
$$v(f) = \lambda f = 363.56 \frac{m}{s}$$

\subsection{Częstotliwość: 5.659 kHz (dla linii wychylonej w górę)}

\begin{table}[H]
\centering
\begin{tabular}{|c|}
\hline
$\Delta [cm]$:\\
\hline 
6.4 \\
6.1 \\
6.7 \\
6.3 \\
6.3 \\
6.0 \\
5.9 \\
6.6 \\
6.9 \\
6.7 \\
6.2 \\
\hline
\end{tabular}
\caption{Tabela różnic}
\end{table}

$$\lambda = 6.37 cm$$
$$v(f) = \lambda f = 360.63 \frac{m}{s}$$

\subsection{Częstotliwość: 6.197 kHz (dla linii wychylonej w górę)}

\begin{table}[H]
\centering
\begin{tabular}{|c|}
\hline
$\Delta [cm]$:\\
\hline 
5.7 \\
5.6 \\
5.7 \\
5.7 \\
5.9 \\
6.0 \\
5.4 \\
5.7 \\
6.0 \\
5.3 \\
6.4 \\
6.5 \\
6.2 \\
6.3 \\
\hline
\end{tabular}
\caption{Tabela różnic}
\end{table}

$$\lambda = 5.89 cm$$
$$v(f) = \lambda f = 367.47 \frac{m}{s}$$

\subsection{Częstotliwość: 6.658 kHz (dla linii wychylonej w górę)}

\begin{table}[H]
\centering
\begin{tabular}{|c|}
\hline
$\Delta [cm]$:\\
\hline 
5.5 \\
5.2 \\
5.2 \\
5.1 \\
5.3 \\
5.9 \\
5.1 \\
5.3 \\
5.5 \\
4.8 \\
\hline
\end{tabular}
\caption{Tabela różnic}
\end{table}

$$\lambda = 5.29 cm$$
$$v(f) = \lambda f = 352.21 \frac{m}{s}$$

\subsection{Częstotliwość: 7.000 kHz (dla linii wychylonej w górę)}

\begin{table}[H]
\centering
\begin{tabular}{|c|}
\hline
$\Delta [cm]$:\\
\hline 
5.5 \\
5.0 \\
5.0 \\
5.2 \\
4.7 \\
5.1 \\
5.4 \\
5.7 \\
\hline
\end{tabular}
\caption{Tabela różnic}
\end{table}

$$\lambda = 5.20 cm$$
$$v(f) = \lambda f = 364.00 \frac{m}{s}$$

\subsection{Częstotliwość: 7.499 kHz (dla linii wychylonej w górę)}

\begin{table}[H]
\centering
\begin{tabular}{|c|}
\hline
$\Delta [cm]$:\\
\hline 
4.2 \\
4.6 \\
4.8 \\
4.4 \\
4.9 \\
4.9 \\
4.6 \\
4.9 \\
4.4 \\
5.0 \\
5.5 \\
3.8 \\
5.2 \\
\hline
\end{tabular}
\caption{Tabela różnic}
\end{table}

$$\lambda = 4.71 cm$$
$$v(f) = \lambda f = 353.03 \frac{m}{s}$$


\subsection{Średnia prędkość dźwięku}

$$v_{sr} = 362.407 \frac{m}{s}$$

\newpage
\section{Obliczenia}

\subsection{Obliczanie teoretycznej prędkości dźwięku z temperatury w powietrzu}
Zanotowaliśmy dwie temperatury powietrza
$$t_{początkowa} = 21,4 \degree C$$
$$t_{końcowa} = 23,4 \degree C$$
Za błąd pomairu przyjęliśmy dokłądność termometra tak więc:
$$\Delta t = 0,1 \degree C$$
Ze wzoru:
$$ v = \sqrt{\frac{\kappa RT}{\mu}}$$
Obliczamy wartości prędkości w powietrzu:
$$ v_{1} = 344,02 m/s$$ 
$$\Delta v_{1} = 0,21 m/s$$
$$ v_{2} = 345,19 m/s$$
$$\Delta v_{2} = 0,21 m/s$$
Przyjmujemy wartość średnią:
$$v_{śr} = 344,61  m/s$$
$$\Delta v_{śr} = 0,30 m/s$$
\newpage
 \subsection{Obliczanie prędkości dźwięku z pomiarów przesunięcia fazowego  powietrzu}
 \subsubsection{Obliczanie średniej prędkości dźwięku}
 Z obliczonych dla każdej częstotliwości prędkości średnich, obliczamy średnią prędkość:
$$v_{śr} = 365,53 m/s $$

\subsubsection{Obliczanie wartości błędu}
Na początek obliczamy dla każdej serii błąd prosty pomiaru uwględniając odchylenie standardowe serii, błąd urządzenia i błąd mierzącego.
\newline
Przyjmujemy następujące wartości błędów:
\newline
Błąd urządzenia: $0,1 cm$
\newline
Błąd mierzącego: $0,1 cm$
\begin{table}[H]
\begin{adjustbox}{center}
\begin{tabular}{|c||c|c|c|c|c|c|c|c|c|c|}
\hline
Pomiar:&1&2&3&4&5&6&7&8&9&10\\
\hline 
Średnia [cm] &	12,92	&	10,42	&	9,06	&	7,79	&	7,13	&	6,37	&	5,89	&	5,29	&	5,20	&	4,71\\
Odchylenie standardowe [cm] &	0,22	&	0,38	&	0,70	&	0,43	&	0,33	&	0,30	&	0,35	&	0,28	&	0,30	&	0,43\\
Błąd całkowity [cm] &	0,24	&	0,39	&	0,71	&	0,44	&	0,34	&	0,32	&	0,36	&	0,29	&	0,31	&	0,43\\
\hline
\end{tabular}
\end{adjustbox}
\caption{Tabela średnich długości fal i błędów}
\end{table}
Dla odpowiednich długości fal i częstotliwości obliczamy prędkości i ich błędy złożone

\begin{table}[H]
\begin{adjustbox}{center}
\begin{tabular}{|c||c|c|c|c|c|c|c|c|c|c|}
\hline
Pomiar:&1&2&3&4&5&6&7&8&9&10\\
\hline 
Prędkość średnia [m/s]&	394,06	&	371,15	&	371,34	&	360,47	&	363,56	&	360,63	&	364,74	&	352,21	&	364,00	&	353,03\\
Błąd [m/s]&	7,23	&	14,01	&	29,09	&	20,40	&	17,27	&	17,86	&	22,41	&	19,48	&	21,76	&	32,59\\
\hline
\end{tabular}
\end{adjustbox}
\caption{Tabela średnich prędkości i błędów}
\end{table}

Obliczamy średni błąd:
$$ \Delta v_{śr} = 67,42 m/s $$


\subsubsection{Podsumowanie}
Ostateczny wynik po zaokrągleniu wyniósł:
$$ v_{śr} = 370 \pm 70  m/s$$
\newpage

\section{Wnioski}
 
Prędkość którą otrzymaliśmy w wyniku doświadczenia wynosi $ v_{D} = 370 \pm 70  m/s$, a w wyniku obliczeń teoretycznych $v_{T} = 344,6 \pm 0,3 m/s$ 
\newline
Mimo usunięcia błędów grubych, błąd nadal jest duży, ale wartość teoretyczna mieści się w zakresie błędu pomiaru. Zasadniczy wpływ na wartość błędu miały niedoskonałości układu, w tym urządzeń pomiarowych i mierzących.
\newline
Finalnie udało nam się uzyskać zadowalający wynik, który, mimo tego że jest obarczony błędem, nie odbiega znacząco do wyznaczonej teoretycznej wartości. 


\begin{thebibliography}{3}
\bibitem{szuba} Szuba S. - "Ćwiczenia laboratoryjne z fizyki"
\end{thebibliography}

\newpage
\tableofcontents{}
\end{document}

