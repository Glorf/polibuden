\documentclass[polish,a4paper]{article}
\usepackage{amsmath}
\usepackage{amssymb,amsfonts,amsthm}
\usepackage[english,main=polish]{babel}
\usepackage{polski}
\usepackage[utf8]{inputenc}
\usepackage[T1]{fontenc}
\usepackage{float}
\usepackage{etoolbox}
\usepackage{pgfplots}
\usepackage{gensymb}
\usepackage{adjustbox}
\usepackage{graphicx}
\patchcmd{\thebibliography}{\section*}{\section}{}{}

\selectlanguage{polish}
\title{FizLab2}

\newcommand{\PRzFieldDsc}[1]{\sffamily\bfseries\scriptsize #1}
\newcommand{\PRzFieldCnt}[1]{\textit{#1}}
\newcommand{\PRzHeading}[8]{
\begin{center}
\begin{tabular}{ p{0.32\textwidth} p{0.15\textwidth} p{0.15\textwidth} p{0.12\textwidth} p{0.12\textwidth} }

  &   &   &   &   \\
\hline
\multicolumn{5}{|c|}{}\\[-1ex]
\multicolumn{5}{|c|}{{\LARGE #1}}\\
\multicolumn{5}{|c|}{}\\[-1ex]

\hline
\multicolumn{1}{|l|}{\PRzFieldDsc{Kierunek}}	& \multicolumn{1}{|l|}{\PRzFieldDsc{Specjalność}}	& \multicolumn{1}{|l|}{\PRzFieldDsc{Rok studiów}}	& \multicolumn{2}{|l|}{\PRzFieldDsc{Symbol grupy lab.}} \\
\multicolumn{1}{|c|}{\PRzFieldCnt{#2}}		& \multicolumn{1}{|c|}{\PRzFieldCnt{#3}}		& \multicolumn{1}{|c|}{\PRzFieldCnt{#4}}		& \multicolumn{2}{|c|}{\PRzFieldCnt{#5}} \\

\hline
\multicolumn{4}{|l|}{\PRzFieldDsc{Temat Laboratorium}}		& \multicolumn{1}{|l|}{\PRzFieldDsc{Numer lab.}} \\
\multicolumn{4}{|c|}{\PRzFieldCnt{#6}}				& \multicolumn{1}{|c|}{\PRzFieldCnt{#7}} \\

\hline
\multicolumn{5}{|l|}{\PRzFieldDsc{Skład grupy ćwiczeniowej oraz numery indeksów}}\\
\multicolumn{5}{|c|}{\PRzFieldCnt{#8}}\\

\hline
\multicolumn{3}{|l|}{\PRzFieldDsc{Uwagi}}	& \multicolumn{2}{|l|}{\PRzFieldDsc{Ocena}} \\
\multicolumn{3}{|c|}{\PRzFieldCnt{\ }}		& \multicolumn{2}{|c|}{\PRzFieldCnt{\ }} \\

\hline
\end{tabular}
\end{center}
}
\pgfplotsset{compat=1.14}
\begin{document}
\PRzHeading{Laboratorium Fizyczne}{Informatyka}{--}{II}{1}{Wyznaczanie przyspieszenia ziemskiego
za pomocą wahadła rewersyjnego}{102}{Michał Bień(132191), Wojciech Taisner(132330)}{}

\section{Wstęp Teoretyczny}
\begin{quotation}\cite{szuba}
Wahadła fizyczne i matematyczne wykonują rych drgający pod wpływem siły ciężkości. W zakresie niedużych amplitud ruch ten jest ruchem harmonicznym, jego okres zależy od właściwości danego wahadła i od przyspieszenia ziemskiego.
\emph{Wahadłem fizycznym} jest każde ciało sztywne mogące się wahać wokół osi poziomej. Po wychyleniu z położenia równowagi na ciało działa moment ciężkości $mg\sin \phi$. Po zastosowaniu do tej sytuacji II zasady dynamiki otrzymamy równanie:
$$ -mgL\sin \phi = I \frac{d^2 \phi}{dt^2} $$
\end{quotation}
Znając ogólne równania ruchu harmonicznego, możemy wyprowadzić równanie określające okres drgań wahadła fizycznego:
$$ T_f = 2\pi \sqrt[]{\frac{I}{D}} $$
Gdzie $D=mgL$ nazywa się momentem kierującym

\begin{quotation}\cite{szuba}
Specjalną postacią wahadła fizycznego, ułatwiającą wyznaczenie długości zredukowanej, jest \emph{wahadło rewersyjne}. Punktem zawieszenia i jednocześnie osią wahań może być jedno z ostrzy: A lub B, a położenie środka masy i moment bezwładności zależą od położenia soczewek $S_1$ i $S_2$. Manipulując położeniem soczewki, możemy doprowadzić do równości obu okresów wahań i tylko wtedy odległość między ostrzami jest długością zredukowaną.
\end{quotation}
Jeśli znamy długość zredukowaną wahadła fizycznego, to jego okres drgań możemy wyznaczyć za pomocą równania dla wahadła matematycznego
$$T_f = 2\pi\sqrt[]{\frac{l_r}{g}}$$
Stąd mierząc okresy drgań wahadła fizycznego rewersyjnego, znając jego długość zredukowaną, możemy wyznaczyć przyspieszenie ziemskie, przekształcając wzór do postaci:
$$ g = \frac{4\pi^2l_r}{T_f^2} $$
\section{Tabela wyników}

\subsection{Pomiar Ta}

\begin{table}[H]
\centering
\begin{tabular}{|c|c|c|c|}
\hline
Położenie x [cm] & Pomiar 1 & Pomiar 2 & Średnia dla jednego okresu\\
\hline 
20 &	19,753 &	19,756 &	1,97545 \\
25 &	19,523 &	19,535 &	1,9529 \\
30 &	19,339 &	19,349 &	1,9344 \\
35 &	19,186 &	19,193 &	1,91895 \\
40 &	19,079 &	19,059 &	1,9069 \\
45 &	19,01 &	19,009 &	1,90095 \\
50 &	18,968 &	18,966 &	1,8967 \\
55 &	18,948 &	18,947 &	1,89475 \\
60 &	18,959 &	18,96 &	1,89595 \\
65 &	19,006 &	19,009 &	1,90075 \\
70 &	19,067 &	19,068 &	1,90675 \\
75 &	19,159 &	19,155 &	1,9157 \\
80 &	19,267 &	19,257 &	1,9262 \\
85 &	19,379 &	19,359 &	1,9369 \\
90 &	19,519 &	19,516 &	1,95175 \\
95 &	19,677 &	19,679 &	1,9678 \\

\hline
\end{tabular}
\end{table}

\subsection{Pomiar Tb}

\begin{table}[H]
\centering
\begin{tabular}{|c|c|c|c|}
\hline
Położenie x [cm] & Pomiar 1 & Pomiar 2 & Średnia dla jednego okresu\\
\hline 
90 &	22,153 &	22,179 &	2,2166\\
85 &	20,4 &	20,389	 & 2,03945\\
80 &	19,317 &	19,335 &	1,9326\\
75 &	18,672 &	18,669 &	1,86705\\
70 &	18,283 &	18,286 &	1,82845\\
65 &	18,115 &	18,119 &	1,8117\\
60 &	18,102 &	18,107 &	1,81045\\
55 &	18,185 &	18,186 &	1,81855\\
50 &	18,326 &	18,32 &	1,8323\\
45 &	18,538 &	18,548 &	1,8543\\
40 &	18,809 &	18,819 &	1,8814\\
35 &	19,096 &	19,083 &	1,90895\\
30 &	19,387 &	19,408 &	1,93975\\
25 &	19,745 &	19,749 &	1,9747\\
20 &	20,083 &	20,088 &	2,00855\\
15 &	20,449 &	20,449 &	2,0449\\

\hline
\end{tabular}
\end{table}
\newpage

\section{Obliczenia}
Wyniki z pomiarów umieściliśmy na wykresie
\newline
\includegraphics[width=\textwidth]{fiz2_wykres.PNG}
Nastepnie odczytaliśmy punkty przecięcia, określiliśmy błąd i określiliśmy okresy:
\newline
$$P_{1} = 1,930  \Delta P_{1} = 0,005 [s] $$
$$P_{2} = 1,925  \Delta P_{2} = 0,005 [s] $$
\newline
Do obliczeń wykorzystaliśmy wzór:
$$ g = \frac{4\pi^2l_r}{T_f^2} $$
Przykładowe obliczenia:
$$l_r = 92 [cm] = 0,92 [m]$$
$$T_1 = 1,930 [s]$$
$$ g_1 = \frac{4\pi^2*0,92}{1,93^2} $$
$$g_{1} = 9,750636 [m/s^2]$$
$$g = T_f^{-2} * 4\pi^2l_r$$
$$\Delta g_1 = |-2 * \frac{\Delta T_1}{T_1} |*g_1 $$
$$\Delta g_1 = 0,050521 [m/s^{2}]$$
Rachunek jednostek:
$$ g = \frac{4\pi^2l_r}{T_f^2} = \frac{m}{s^2}$$


Dokonalismy obliczeń przyspieszenia ziemskiego i wartości błędów:
$$g_{1} = 9,750636  \Delta g_{1} = 0,050521 [m/s^{2}]$$
$$g_{2} = 9,801354  \Delta g_{2} = 0,050921 [m/s^{2}]$$
$$g_{śr} = 9,775995  \Delta g_{śr} = 0,050719 [m/s^{2}]$$
Zaokrąglając:
$$g_{śr} = 9,776 \pm 0,051 [m/s^{2}]$$

\section{Wnioski}
Wyznaczona w wyniku doświadczenia wartość przyspieszenia ziemskiego jest zbliżona do wartości tabelarycznej. 
\newline
Wpływ na wynik doświadczenia miały przede wszsytkim niepewności pomiaru wynikające z niedoskonałości urządzenia pomiarowego i czynników niezlaeżnych - oporu powietrza i tarcia na łożysku wahadła.
\newline
Finalnie udało nam się uzyskać zadowalający wynik, który mieści się w granicy błędu pomiarowego.

\begin{thebibliography}{1}
\bibitem{szuba} Szuba S. - "Ćwiczenia laboratoryjne z fizyki"
\end{thebibliography}
\end{document}

