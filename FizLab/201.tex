\documentclass[polish,a4paper]{article}
\usepackage{amsmath}
\usepackage{amssymb,amsfonts,amsthm}
\usepackage[english,main=polish]{babel}
\usepackage{polski}
\usepackage[utf8]{inputenc}
\usepackage[T1]{fontenc}
\usepackage{float}
\usepackage{etoolbox}
\usepackage{pgfplots}
\usepackage{gensymb}
\usepackage{adjustbox}
\usepackage{graphicx}
\patchcmd{\thebibliography}{\section*}{\section}{}{}

\selectlanguage{polish}
\title{FizLab3}

\newcommand{\PRzFieldDsc}[1]{\sffamily\bfseries\scriptsize #1}
\newcommand{\PRzFieldCnt}[1]{\textit{#1}}
\newcommand{\PRzHeading}[8]{
\begin{center}
\begin{tabular}{ p{0.32\textwidth} p{0.15\textwidth} p{0.15\textwidth} p{0.12\textwidth} p{0.12\textwidth} }

  &   &   &   &   \\
\hline
\multicolumn{5}{|c|}{}\\[-1ex]
\multicolumn{5}{|c|}{{\LARGE #1}}\\
\multicolumn{5}{|c|}{}\\[-1ex]

\hline
\multicolumn{1}{|l|}{\PRzFieldDsc{Kierunek}}	& \multicolumn{1}{|l|}{\PRzFieldDsc{Specjalność}}	& \multicolumn{1}{|l|}{\PRzFieldDsc{Rok studiów}}	& \multicolumn{2}{|l|}{\PRzFieldDsc{Symbol grupy lab.}} \\
\multicolumn{1}{|c|}{\PRzFieldCnt{#2}}		& \multicolumn{1}{|c|}{\PRzFieldCnt{#3}}		& \multicolumn{1}{|c|}{\PRzFieldCnt{#4}}		& \multicolumn{2}{|c|}{\PRzFieldCnt{#5}} \\

\hline
\multicolumn{4}{|l|}{\PRzFieldDsc{Temat Laboratorium}}		& \multicolumn{1}{|l|}{\PRzFieldDsc{Numer lab.}} \\
\multicolumn{4}{|c|}{\PRzFieldCnt{#6}}				& \multicolumn{1}{|c|}{\PRzFieldCnt{#7}} \\

\hline
\multicolumn{5}{|l|}{\PRzFieldDsc{Skład grupy ćwiczeniowej oraz numery indeksów}}\\
\multicolumn{5}{|c|}{\PRzFieldCnt{#8}}\\

\hline
\multicolumn{3}{|l|}{\PRzFieldDsc{Uwagi}}	& \multicolumn{2}{|l|}{\PRzFieldDsc{Ocena}} \\
\multicolumn{3}{|c|}{\PRzFieldCnt{\ }}		& \multicolumn{2}{|c|}{\PRzFieldCnt{\ }} \\

\hline
\end{tabular}
\end{center}
}
\pgfplotsset{compat=1.14}
\begin{document}
\PRzHeading{Laboratorium Fizyczne}{Informatyka}{--}{II}{1}{ Wyznaczanie pojemności kondensatora za pomocą drgań relaksacyjnych }{201}{Michał Bień(132191), Wojciech Taisner(132330)}{}

\section{Wstęp Teoretyczny}
\begin{quotation}\cite{szuba}
Jeśli do obwodu RC podłączymy neonówkę równolegle do kondensatora, to wystąpią okresowe, niesymetryczne wzrosty i spadki napięcia na kondensatorze nazywane \emph{drganiami relaksacyjnymi}. \\
Jeżeli do neonówki przykładamy małe napięcie, prąd przez nią nie płynie ze względu na małe przewodnictwo gazu. Po przekroczeniu wartości $U_z$ (napięcie zapłonu) następuje jonizacja lawinowa gazu, widoczne jest jego świecenie i przez neonówkę płynie prąd. Rozpoczęta jonizacja lawinowa trwa dalej przy nieco niższych napięciach - ustaje, gdy napięcie spada poniżej wartości $U_g$ (napięcie gaśnięcia). \\
Opisane właściwości neonówki wykorzystujemy do uzyskania drgań relaksacyjnych. Kondensator C ładuje się ze źródła prądu stałego przez opornik R. Napięcie na okładkach kondensatora rośnie wykładniczo, zgodnie z równaniem: $$ U_c = \epsilon (1-e^{-\frac{t}{RC}})$$ Gdy napięcie osiągnie wartość $U_z$, zapala się neonówka N. Ponieważ opór palącej się neonówki jest bardzo mały, następuje szybkie rozładowanie kondensatora do napięcia $U_g$. Po zgaśnięciu neonówki rozpoczyna się kolejne ładowanie kondensatora i następne jego rozładowanie. Opisane procesy powtarzają się cyklicznie.
\end{quotation}
W większości przypadków możemy przyjąć, że okres drgań relaksacyjnych jest równy czasowi ładowania kondensatora od napięcia gaśnięcia $U_g$, do napięcia zapłonu $U_z$.
Po zastosowaniu powyższego równania i przekształceniu go, otrzymujemy zależność: $$T=RC ln\frac{\epsilon-U_z}{\epsilon-U_g}$$ Wyrażenie $ln\frac{\epsilon-U_z}{\epsilon-U_g}$ jest wielkością stałą dla określonego napięcia i określonego typu neonówki. Jeśli oznaczymy je symbolem K, to równanie uzyska pozstać $$T=RCK$$ Jest to podstawowy wzór, z którego skorzystamy, by wpierw zbadać stałą K obwodu, a następnie na tej podstawie oszacować pojemność mierzonych kondensatorów.


\section{Tabela wyników}
\subsection{Pomiary i obliczanie wartości K}
\begin{table}[H]
\centering
\begin{tabular}{|c|c|c|c|c|c|}
\hline
R [$M\Omega$] & C [$\mu F$] & $T_1$ [s/10] & $T_2$ [s/10] & $T_{sr}$ [s] & K\\
\hline 
1&	0,7& 2,76&	2,79&	0,2775&  	0,396428571 \\
1&	0,8& 2,95&	2,81&	0,288&  	0,36\\
1&	0,9& 3,15&	3,27&	0,321&  	0,356666667 \\
1&	1	&3,49&	3,34&	0,3415& 	0,3415\\
1&	1,1	&3,57&	3,49&	0,353&  	0,320909091 \\
2&	0,7	&4,06&	4,43&	0,4245& 	0,303214286 \\
2&	0,8	&4,93&	5,16&	0,5045& 	0,3153125 \\
2&	0,9	&5,43&	5,63&	0,553&  	0,307222222 \\
2&	1	&6,21&	6,41&	0,631&  	0,3155 \\
2&	1,1	&6,69&	6,81&	0,675& 	    0,306818182 \\
3&	0,7	&5,94&	6,49&	0,6215& 	0,295952381 \\
3&	0,8	&7,48&	8,08&	0,778&  	0,324166667 \\
3&	0,9	&8,31&	8,4&	0,8355& 	0,309444444 \\
3&	1	&9,24&	10& 	0,962&  	0,320666667 \\
3&	1,1	&10,22&	10,69& 	1,0455& 	0,316818182 \\
4&	0,7	&9,15&	8,53& 	0,884&  	0,315714286 \\
4&	0,8	&9,63&	10,69& 	1,016&  	0,3175 \\
4&	0,9	&11,21&	10,92& 	1,1065& 	0,307361111 \\
4&	1	&12,13&	12,35& 	1,224&  	0,306 \\
4&	1,1	&12,76&	13,15& 	1,2955& 	0,294431818 \\
5&	0,7	&11,54&	11,55& 	1,1545& 	0,329857143 \\
5&	0,8	&12,96&	13,11& 	1,3035& 	0,325875 \\
5&	0,9	&14,68&	14,69& 	1,4685& 	0,326333333 \\
5&	1	&16,17&	16,32& 	1,6245& 	0,3249 \\
5&	1,1	&17,07&	17,42& 	1,7245& 	0,313545455 \\
\hline
\end{tabular}
\end{table}

Z wyników pomiarów obliczamy wartość średnią i odchylenie standardowe
$$K_{sr} = 0,322086 $$
$$\Delta K = 0,021717 $$

\subsection{Pomiary wartości $C_1$}
\begin{table}[H]
\centering
\begin{tabular}{|c|c|c|c|c|c|c|}
\hline
R [$M\Omega$] & $T_1$ [s/10] & $T_2$ [s/10] & $T_3$ [s/10] & $T_4$ [s/10] & $T_{sr}$ [s] & C [$\mu F$] \\
\hline
1&	6,85&	7,11&	7,61&	7,09&	0,7165&	2,224564456 \\
2&	13,5&	14,11&	13,9&	13,94&	1,38625&	2,151990563 \\
3&	20,54&	20,74&	20,26&	20,54&	2,052&	2,123659578 \\
4&	27,03&	27,27&	26,76&	26,76&	2,6955&	2,092223828 \\
5&	33,91&	34,03&	33,91&	34,22&	3,40175&	2,112327184 \\

\hline
\end{tabular}
\end{table}
Wartość pojemności obliczamy korzystając z $K_{sr}$
Wartość średnia:
$$C_1 = 2,140953[\mu F]$$

\subsection{Pomiary wartości $C_2$}
\begin{table}[H]
\centering
\begin{tabular}{|c|c|c|c|c|c|c|}
\hline
R [$M\Omega$] & $T_1$ [s/10] & $T_2$ [s/10] & $T_3$ [s/10] & $T_4$ [s/10] & $T_{sr}$ [s] & C [$\mu F$] \\
\hline
1&	3,87&	3,78&	4,38&	4,6& 	0,41575&	1,290806242 \\
2&	8,58&	8,59&	8,39&	8,66&	0,8555&	1,328063428 \\
3&	12,71&	13,08&	12,71&	12,96&	1,2865&	1,331426924 \\
4&	16,74&	17& 	16,88&	16,98&	1,69&	1,311763409 \\
5&	21& 	20,77&	21,35&	20,95&	2,10175&	1,305088163 \\

\hline
\end{tabular}
\end{table}
Wartość pojemności obliczamy korzystając z $K_{sr}$
Wartość średnia:
$$C_2 = 1,31343[\mu F]$$

\subsection{Pomiary wartości $C_3$}

\begin{table}[H]
\centering
\begin{tabular}{|c|c|c|c|c|c|c|}
\hline
R [$M\Omega$] & $T_1$ [s/10] & $T_2$ [s/10] & $T_3$ [s/10] & $T_4$ [s/10] & $T_{sr}$ [s] & C [$\mu F$] \\
\hline
1&	3,78&	3,6&	3,16&	3,52&	0,3515&	1,091325061 \\
2&	6,11&	6,53&	6,5&	6,02&	0,629&	0,976448739 \\
3&	9,68&	9,76&	9,83&	9,78&	0,97625&	1,010342429 \\
4&	12,84&	12,92&	12,55&	12,05&	1,259&	0,97722493 \\
5&	16,11&	16,52&	15,87&	16,05&	1,61375&	1,002063054 \\

\hline
\end{tabular}
\end{table}
Wartość pojemności obliczamy korzystając z $K_{sr}$
Wartość średnia:
$$C_3 = 1,011481[\mu F]$$

\subsection{Pomiary wartości $C_4$}

\begin{table}[H]
\centering
\begin{tabular}{|c|c|c|c|c|c|c|}
\hline
R [$M\Omega$] & $T_1$ [s/10] & $T_2$ [s/10] & $T_3$ [s/10] & $T_4$ [s/10] & $T_{sr}$ [s] & C [$\mu F$] \\
\hline
1&	2,47&	2,65&	2,61&	2,78&	0,26275	&0,815777126 \\
2&	4,04&	4,51&	4,51&	4,51&	0,43925	&0,681884115 \\
3&	6,66&	6,75&	6,67&	6,67&	0,66875&	0,692103968 \\
4&	8,73&	8,93&	8,66&	8,66&	0,8745&	0,67877935 \\
5&	10,68&	11,1&	10,76&	10,76&	1,0825&	0,672181723 \\

\hline
\end{tabular}
\end{table}
Wartość pojemności obliczamy korzystając z $K_{sr}$
Wartość średnia:
$$C_4 = 0,708145[\mu F]$$



\section{Obliczenia}
W obliczeniach wykorzystaliśmy następujące wzór:
$$T=RCK$$
Przekształcając powyższy wzór możemy obliczyć pojemność kondensatora:
$$C=\frac{T}{RK} = \frac{T}{R}*K^{-1}$$
Wyznaczenie błędu pojemności - metodą różniczki logarytmicznej
$$\Delta C = |-1*\frac{\Delta k}{k}|*C$$
Przykładowe obliczenia:
$$C_{1.1} = \frac{T_{sr}}{R*K_{sr}} = \frac{0,7165}{1*10^6*0,322086} = 2,144354 [\mu F]$$
$$\Delta C_1 = |-1*\frac{\Delta K}{K_{sr}}| * C_1 = \frac{0,021717}{0,322086}*2,140953 = 0,144354 [\mu F]$$
Przykładowe zaokrąglanie:
$$\Delta C_1 = 0,144354 [\mu F]$$
$$\Delta C_1 = 0,14 [\mu F]$$
$$(0,2 - 0,14)/0,14 = 0,06/0,14 = 0,428571... > 0,1$$
$$C_1 = 2,14[\mu F]$$
$$C_1 = 2,14 \pm 0,14[\mu F]$$
Rachunek jednostek:
$$C=\frac{T}{RK}$$
$$C=\frac{[s]}{[\Omega]} = \frac{[A]*[s]}{[V]} = \frac{[C]}{[V]} = [F]$$
Obliczone wartości pojemności i ich błędy:
$$C_1 = 2,140953[\mu F]$$
$$\Delta C_1 = 0,144354 [\mu F]$$
$$C_2 = 1,31343[\mu F]$$
$$\Delta C_2 = 0,088558 [\mu F]$$
$$C_3 = 1,011481[\mu F]$$
$$\Delta C_3 = 0,068199 [\mu F]$$
$$C_4 = 0,708145[\mu F]$$
$$\Delta C_4 = 0,047747 [\mu F]$$
Po zaokrągleniu:
$$C_1 = 2,14 \pm 0,14[\mu F]$$
$$C_2 = 1,31 \pm 0,09[\mu F]$$
$$C_3 = 1,01 \pm 0,07[\mu F]$$
$$C_4 = 0,71 \pm 0,05[\mu F]$$
\section{Wnioski}
Przeprowadzając doświadczenie dowiedliśmy, iż jest możliwe doświadczalne wyznaczenie przybliżonej pojemności nieznanego nam kondensatora, jeżeli dysponujemy układem RC zaopatrzonym w neonówkę, której okresy mrugnięć możemy mierzyć. Niezbędne jest jednak wcześniejsze wyznaczenie stałej K dla tego układu.

\begin{thebibliography}{1}
\bibitem{szuba} Szuba S. - "Ćwiczenia laboratoryjne z fizyki"
\end{thebibliography}
\end{document}

