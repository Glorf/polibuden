\documentclass[a4paper]{article}
\usepackage[polish]{babel}
\usepackage[utf8x]{inputenc}
\usepackage[T1]{fontenc}

\usepackage[a4paper,top=3cm,bottom=2cm,left=3cm,right=3cm,marginparwidth=1.75cm]{geometry}
\usepackage{amsmath}
\usepackage{graphicx}
\usepackage{minted}

\author{Michał Bień, Krzysztof Bończyk}
\title{PW Specyfikacja komunikatora IPC}
\date{}
\begin{document}
\maketitle
\section{Architektura i porty}
Komunikacja między serwerem a klientami odbywa się przy pomocy dwóch kolejek IPC tworzonych przez serwer, o z góry zdefiniowanych wartościach: kolejka wejścia do serwera o kluczu 2137; kolejka wyjścia z serwera o numerze 2138.
\section{Struktury danych}
\subsection{Wiadomość}
Każda wiadomość ma rozmiar 128 bajtów i składa się z:
\begin{itemize}
\item \textbf{typu} będącego identyfikatorem unikalnym dla połączenia - wielkości 8 bajtów
\item \textbf{kodu} będącego enumerowanym typem - kodem wiadomości opisującym jej zawartość lub treść żądania - wielkości 4 bajty
\item \textbf{payloadu} będącego właściwą zawartością wiadomości dowolnego typu - wielkości 116 bajtów
\end{itemize}
\begin{minted}[bgcolor=lightgray]{C}
struct message {
	long mtype;
	int code;
	char payload[116];
}
\end{minted}
\subsection{Payloady}
\begin{itemize}
\item \textbf{Login} wysyłany jest w celu zalogowania się
\begin{minted}[bgcolor=lightgray]{C}
struct login {
    char login[56];
    char password[56];
}
\end{minted}

\item \textbf{Show list} służy do przechowywania listy integerów
\begin{minted}[bgcolor=lightgray]{C}
struct show_list {
	int list[29];
}
\end{minted}

\item \textbf{Addressed text} przechowuje integer i tablicę charów
\begin{minted}[bgcolor=lightgray]{C}
struct addressed_text {
    int address;
    char text[112];
};
\end{minted}
\\
\item \textbf{Group addressed text} przechowuje dwa integery i tablicę charów
\begin{minted}[bgcolor=lightgray]{C}
struct group_addressed_text {
    int user;
    int group;
    char text[108];
};
\end{minted}

\item \textbf{String} będący domyślnym typem który przekazuje tablica charów
\end{itemize}
\section{Kody informacji}
\subsection{Żądania klienta:} FREE, LOGIN, LOGOUT, USERID\_NAME, SHOW\_LOGGED, SHOW\_GROUPUSERS, GROUPID\_NAME, SIGNIN\_GROUP, SIGNOUT\_GROUP, SHOW\_GROUPS, SEND\_GROUP, SEND\_PRIV
\subsection{Odpowiedzi serwera:} STATUS\_OK, STATUS\_ERROR
\subsection{Żądania serwera:} RCV\_GROUP, RCV\_PRIV
\begin{minted}[bgcolor=lightgray]{C}
enum TypWiadomosci {
    FREE,
    LOGIN,
    LOGOUT,
    USERID_NAME,
    NAME_USERID,
    SHOW_LOGGED,
    SHOW_GROUPUSERS,
    GROUPID_NAME,
    SIGNIN_GROUP,
    SIGNOUT_GROUP,
    SHOW_GROUPS,
    SEND_GROUP,
    SEND_PRIV,
    RCV_GROUP,
    RCV_PRIV,
    STATUS_OK,
    STATUS_ERROR
};
\end{minted}

\section{Struktura zapytań}
connectionPort -- seria portów wybrana przez klienta
odpowiednio x, 1x i 2x, gdzie x to liczba od 1 do 9.
Port x odpowiada na zapytania dotyczące swojej zajętości (FREE)
Port 1x odpowiada na zapytania kontrolne oraz logowania klienta
Port 2x wysyła wiadomości z serwera do klienta (RCV\_*)

\subsection{Sprawdzenie portu}
Zapytanie klienta: message(connectionPort, FREE, NULL) \\
Odpowiedź serwera: message(connectionPort, STATUS\_*, string(status))\\
Sprawdzenie zajętości otwartego portu skutkuje jego natychmiastowym zamknięciem.
Zajęte porty zwracają STATUS\_ERROR, otwarte STATUS\_OK.
\subsection{Logowanie}
Zapytanie klienta: message(connectionPort, LOGIN, login(login, password,connectionPort))\\
Odpowiedź serwera: message(connectionPort, STATUS\_*, string(status))
\subsection{Wylogowywanie}
Zapytanie klienta: message(connectionPort, LOGOUT, NULL) \\
Odpowiedź serwera: message(connectionPort, STATUS\_*, string(status)) \\
Wylogowywanie zwalnia zajęty przez połączenie port
\subsection{Lista zalogowanych}
Zapytanie klienta: message(connectionPort, SHOW\_LOGGED, NULL) \\
Odpowiedź serwera: message(connectionPort, STATUS\_OK, show\_list(id\_userów)) \\
Zwraca listę podłączonych w danej chwili użytkowników
\subsection{Nazwa użytkownika o podanym id}
Zapytanie klienta: message(connectionPort, USERID\_NAME, addressed\_text(id, NULL))\\
Odpowiedź serwera: message(connectionPort, STATUS\_OK, string(login))
\subsection{Id użytkownika o podanej nazwie}
Zapytanie klienta: message(connectionPort, NAME\_USERID, string(name))\\
Odpowiedź serwera: message(connectionPort, STATUS\_OK, addressed\_text(id, NULL))
\subsection{Lista aktywnych grup}
Zapytanie klienta: message(connectionPort, SHOW\_GROUPS, NULL) \\
Odpowiedź serwera: message(connectionPort, STATUS\_OK, show\_list(id\_grup))
\subsection{Nazwa grupy o podanym id}
Zapytanie klienta: message(connectionPort, GROUPID\_NAME, addressed\_text(id, NULL))\\
Odpowiedź serwera: message(connectionPort, STATUS\_OK, string(nazwa))
\subsection{Lista członków grupy o podanym id}
Zapytanie klienta: message(connectionPort, SHOW\_GROUPUSERS, addressed\_text(id, NULL))\\
Odpowiedź serwera: message(connectionPort, STATUS\_OK, show\_list(id\_userów))
\subsection{Zapisz się do grupy o podanym id}
Zapytanie klienta: message(connectionPort, SIGNIN\_GROUP, addressed\_text(id, NULL))\\
Odpowiedź serwera: message(connectionPort, STATUS\_*, string(status))
\subsection{Wypisz się z grupy o podanym id}
Zapytanie klienta: message(connectionPort, SIGNOUT\_GROUP, addressed\_text(id, NULL))\\
Odpowiedź serwera: message(connectionPort, STATUS\_*, string(status))
\subsection{Wyślij do grupy o podanym id}
Zapytanie klienta: message(connectionPort, SEND\_GROUP, addressed\_text(idgrupy, tresc)\\
Odpowiedź serwera: message(connectionPort, STATUS\_OK, string(status))
\subsection{Wyślij do użytkownika o podanym id}
Zapytanie klienta: message(connectionPort, SEND\_PRIV, addressed\_text(idusera, tresc)\\
Odpowiedź serwera: message(connectionPort, STATUS\_OK, string(status))
\subsection{Odbierz wiadomość od grupy}
Żądanie serwera: message(cP, RCV\_GROUP, group\_addressed\_text(idusera, idgrupy, tresc)\\
\subsection{Odbierz wiadomość prywatną}
Żądanie serwera: message(cP, RCV\_PRIV, addressed\_text(idusera, tresc)\\


\section{Typowy flow}
\begin{enumerate}
\item Klient odpala się, szuka najniższego możliwego portu w serwerze: 1 zajęte, 2 zajęte... 3 wolne!
\item Serwer odpowiadając twierdząco na niezajęcie trójki, pekuje ją dla klienta
\item Klient będzie teraz wysyłał wiadomości do serwera na porcie 13, oraz odbierał na porcie 23.
\item Użytkownik loguje się i wykonuje operacje. W międzyczasie asynchronicznie otrzymuje wiadomości.
\end{enumerate}

\end{document}

