\documentclass[a4paper]{article}
\usepackage[polish]{babel}
\usepackage[utf8x]{inputenc}
\usepackage[T1]{fontenc}

\usepackage{amsmath}

\author{}
\title{Skrypt ASK}
\date{}
\begin{document}
\maketitle
\section{Procesor}
\subsection{Cykle pracy procesora}
\textbf{Cykl zegarowy} - najmniejsza jednostka pracy procesora, oznaczająca jeden okres zegara taktującego procesor\\
\textbf{Cykl maszynowy} - jeden lub wiele cykli zegarowych, wspólnie składających się na proces pobrania danych z pamięci operacyjnej do rejestrów procesora\\
\textbf{Cykl rozkazowy} - jeden lub wiele cykli maszynowych, realizujących jeden rozkaz procesora.

\subsection{Rejestry procesora}
\textbf{Rejestry ogólnego przeznaczenia} - np. AX, BX, CX...\\
\textbf{Rejestry segmentowe:}\\
CS - Code segment - przechowuje adres segmentu kodu do wykonania\\
DS - Data segment - przechowuje adres segmentu danych\\
SS - Stack segment - przechowuje adres stosu\\
\textbf{Rejestry stanu:}\\
Rejestr FLAGS przechowuje flagi: Sign, Zero, Carry, Overflow\\
Rejestr IP przechowuje offset aktualnie wykonywanej instrukcji w segmencie kodu\\

\subsection{Fazy cyklu rozkazowego}
\begin{enumerate}
\item Faza pobrania rozkazu - pobranie rozkazu z CS+IP do pamięci podręcznej
\item Faza dekodowania rozkazy - określenie operacji do wykonania i adresów fizycznych
\item Faza wykonania
\item Faza zapisania wyników - zapisanie danych, ustawienie nowego IP
\end{enumerate}
\newpage
\section{Pamięć operacyjna}
\subsection{Linie sygnałowe: }
Adresowa, Danych, CE, WE, OE, RAS, CAS
\subsection{Tryby pracy}
\textbf{Page mode}\\
odczyt: RAS -> adres wiersza -> CAS -> adres kolumny -> na 'CAS i 'RAS pojawia się dana\\
zapis: RAS -> adres wiersza -> CAS -> adres kolumny -> WE i dane na data\\
\textbf{FPM - fast page mode}\\
Wybieramy wiersz i probujemy kolejne adresy kolumn\\
\textbf{Burst mode}\\
Odczyt blokowy - podajemy adres wiersza i kolumny, każdy kolejny takt strobujący CAS odnosi się do adresu n+1 względem poprzedniego.\\
\subsection{Odświeżanie pamięci}
\textbf{RAS-only: } wysyłamy tylko sygnał RAS z konkretnym adresem, cały wiersz jest odświeżany.
\textbf{Inne nieopisane metody: } CBR (CAS before RAS), ukryte, autoodświeżanie

\newpage
\section{Pamięć dyskowa}
\subsection{Adresowanie danych}
adres CHS:\\
C - numer cylindra - czyli logicznego walca przechodzącego przez ścieżki wszystkich talerzy dysku na tej samej wysokości - adres synchronizacyjny dla głowic sprzężonych\\
H - numer głowicy - czyli numer i strona talerza, do którego przylega konkretna głowica\\
S - numer sektora - logiczny fragment ścieżki zawierający interesujące nas dane\\
\subsection{Macierze RAID}
\begin{itemize}
\item RAID 0 (striping) - łączenie wielu dysków w logicznie jeden - dane "paski"
\item RAID 1 (mirroring) - na dwóch dyskach kopia dokładnie tych samych danych
\item RAID 2 - dane zapisane pojednyczo na n dyskach. Na log(n) dyskach dane korekcyjne
\item RAID 3 - dane zapisane pojedynczo na dyskach, jeden dysk na dane korekcyjne.
\item RAID 4 - dane zapisane pojedynczo na niezależnych od siebie dyskach działąjących równolegle. Jeden dysk na dane korekcyjne
\item RAID 5 - dane zapisane pojedynczo na niezależnych od siebie dyskach działających równolegle. Dane korekcyjne rozproszone w macierzy
\end{itemize}

\newpage
\section{Interfejsy:}
\subsection{IDE}
\textbf{Tryby pracy standardowe Power Management: }\\
gotowości, jałowy, czuwania, uśpienia\\
\textbf{APM} - 254 możliwe poziomy poboru energii\\
\textbf{Mechanizmy ATAPI: }\\
security mode - hasła\\
S.M.A.R.T - wykrywanie awarii dysku\\
AAM - wycisza pracę dysku

\subsection{SATA}
Symbole - informacje krótkie\\
Ramki FIS - informacje długie

\subsection{SCSI}
\textbf{Fazy pracy: }\\
Faza wolnej magistrali\\
Faza arbitrażu - wybór urządzenia przejmującego kontrolę\\
Faza selekcji - wybrane urządzenie wybiera cel transmisji\\
Faza rozkazowa - wysyłanie rozkazów do celu\\
Faza reselekcji - cel przejmuje magistrale i nawiązuje połączenie z nadawcą\\
Faza przesyłania danych - w odpowiedzi na poprzednie rozkazy\\
Faza przesyłania raportu - raport z przebiegu transmisji\\
Faza przesyłania wiadomości - informacje pomocnicze\\
Faza zgłaszania przesyłania wiadomości - żądanie wysłania wiadomości\\
Faza zerowania - reset całej magistrali

\subsection{IEEE-1284}
\textbf{Tryby pracy: }\\
Tryby SPP (Standard Parallel Port): \\
Kompatybilny - stary, wolny, standardowy\\
Półbajtowy - dwukierunkowa transmisja danych przejmując bity stanu (BSY, ACK, PERR, SEL)\\
Bajtowy - najszybszy - dwukierunkowy port danych (?)\\
Tryby SPP wymagają: rejestru stanu, rejestru danych, rejestru sterującego\\
Tryb EPP (Enchanced Parallel Port): \\
Wymaga rejestrów SPP + dwukierunkowego rejestru danych + dwukierunkowego rejestru adresowego\\
Tryb ECP (Extended Capabilities Port): \\
Wymaga rejestrów SPP +rejestr adresowy FIFO + rejestr danych FIFO + rejestr sterujący ECR \\
\subsection{RS-232C}
Stary dobry UART/USART\\
Linie sygnałowe: \\
Tx, Rx,\\
RTS (Request to send)\\
CTS (Clear to send)\\
DSR (Data set ready)\\
DTR (Data terminal ready)\\
DCD (Data carrier detect)\\
RI (Ring indicator)\\

\subsection{USB}
\textbf{Tryby: }\\
asynchroniczny - bulk transfer - obsługa peryferiów wyjścia\\
synchroniczny - isochronous transfer - transfer danych\\
natychmiastowy - interrupt transfer - obsługa peryferiów wejścia\\
sterujący - control transfer - konfiguracja nowego urządzenia\\
\textbf{Rodzaje pakietów: }\\
żeton - token - pakiet kontrolny określający adres urządzenia i kierunek przepływu danych\\
pakiet danych - data - przesył danych zabezpieczonych CRC\\
potwierdzenie - handshake - raport o pomyślnym przesłaniu danych\\
pakiet specjalny - special - do sterowania przepływem danych przy urządzeniach o różnych prędkościach\\
\subsection{Firewire}
Do aparatów i sprzętu audiowizualnego
\subsection{Thunderbolt}
Zastąpić wszystkie magistrale
\newpage

\section{Przetwarzanie potokowe}
Kolejne etapy przetwarzania kolejnych poleceń mogą odbywać się potokowo. Tzn. kolejna instrukcja zaczyna być wykonywana gdy poprzednia wyjdzie z pierwszej fazy i tak lecą potokiem. \\
Problemy: np. przerwania, pętle warunkowe, powodują unieważnienie potoku \\
\textbf{Postępowanie w przypadku rozgałęzień: }\\
\begin{itemize}
\item pobieranie rozkazów po rozgałęzieniach z wyprzedzeniem
\item potokowe przetwarzanie kilku możliwych odgałęzień równocześnie
\item bufor pętli - rozkazy z pętli zapisywane w L1 i sprowadzane szybko w kolejnej iteracji
\item opóźnianie rozgałęzienia - agresywna zmiana kolejności wykonywania rozkazów
\item zaawansowane przewidywanie wyników rozgałęzień
\end{itemize}
Najczęściej aktualnie stosowane jest to ostatnie - w oparciu o bufor decyzji (BTB) przewidywane są kolejne decyzje\\

\section{Przetwarzanie superskalarne}
Wiele potoków wykonawczych w rdzeniu - jeśli dwa, oznaczamy je U, V\\
U - pozwala wykonać dowolną instrukcję x86\\
V - pozwala wykonać tylko instrukcje proste\\
\subsection{Łączenie instrukcji w pary}
UV - połączone w pary mogą być wykonywane współbieżnie\\
PU - instrukcje z potoku U czasami razem z instrukcjami z potoku V\\
PV - instrukcje, które po połączeniu z potokiem PU albo PV wykonywane w V\\
\subsection{Konflikty}
\begin{itemize}
\item RAR (Read After Read) - dwie instrukcje chcą odczytać wartość rejestru równocześnie - dublujemy port rejestru
\item RAW (Read After Write) - jedna instrukcja zapisuje,a druga odczytuje dane z rejestru - należy określić kolejność, a następnie pozwolić instrukcjom wykonać się we właściwej kolejności
\item WAR (Write After Read), WAW (Write After Write) - odpowiednio - zapis po odczycie i zapis po zapisie. Uwspółbieżniamy przydzielając potokowi wirtualny rejestr - realokując rejestr docelowy w tablicy RAT (Register Allocation Table). Po wykonaniu potoków dane należy złożyć.
\end{itemize}

\subsection{Dynamiczne wykonywanie instrukcji}
Wykonywanie poza kolejnością w celu dobrania par UV, albo ominięcia rozgałęzień.\\
\textbf{Systemy:}\\
Zaawansowana predykcja rozgałęzień\\
Dynamiczna analiza przepływu danych - analiza w czasie rzeczywistym uszeregowania instrukcji w jednostkach wykonawczych\\
Spekulatywne wykonywanie rozkazów - wykonywanie instrukcji za nierozstrzygniętą jeszcze pętlą warunkową.\\
\textbf{Typowa sekwencja: }\\
\begin{itemize}
\item MIS (Microcode Instruction Sequencer) tworzy zaoptymalizowaną sekwencje instrukcji
\item MIS ładuje oryginalną sekwencję instrukcji
\item operacje z konfliktów typu WAR i WAW dostają rejestry tymczasowe
\item RS (Reservation Station) pobiera zoptymalizowaną sekwencję żądań przydziałów zasobów
\item RU (Retirement Unit) scala wyniki operacji i porządkuje je w oryginalnym układzie
\end{itemize}

\section{Pamięć podręczna}
\subsection{Schematy architektury}
\begin{itemize}
\item Look-Aside - procesor odwołuje się do kontrolera pamięci - jeśli ten znajdzie daną w cache (cache hit) to sprowadza z cache, jeśli nie znajdzie (cache miss), to sprowadza z pamięci.
\item Look-Through - procesor odwołuje się do cache - jeśli dana istnieje to jest uzyskiwany dostęp, jeśli nie, \textbf{kontroler pamięci cache} sprowadza daną do cache i jest uzyskiwany dostęp
\item Backside - Procesor dostaje dostęp do magistrali BSB niezależnej od FSB i robi z pamięcią podręczną co mu się żywnie podoba - sam nią zarządza.
\end{itemize}
\subsection{Strategie zapisu}
\begin{itemize}
\item Write-through - każdy zapis do cache powoduje natychmiastowy zapis do pamięci
\item Buffered write-through - każdy zapis do cache powoduje zapis do bufora, który w wolnej chwili zapisuje do pamięci
\item Write-back - dane nadpisane sa oznaczane jako przeterminowane i uaktualniane przy próbie dostępu
\end{itemize}
Spójność pamięci - MESI - każdy wiersz może być Modified, Exclusive, Shared, Invalid
\subsection{Odwzorowania pamięci}
\begin{itemize}
\item bezpośrednie - n-ty wiersz każdej strony może być odwzorowany tylko w n-tym wierszu pamięci operacyjnej
\item skojarzeniowe - dowolne odwzorowanie wierszy pamięci na cache, zapisane w katalogu translacji TAG-RAM
\item sekwencyjno-skojarzeniowe - podział cache na 2/4/8 części - odwzorowanie bezpośrednie w ramach jednej częsci i skojarzeniowe odwzorowanie części w ramach całej pamięci - metoda łączy zalety dwóch powyższych niwelując ich wady (ciągła wymiana pamięci, każdorazowe przeszukiwanie TAG-RAM).
\end{itemize}

\section{MMX i SSE}
MMX (MultiMedia eXtension) - przetwarzanie równoległe SIMD (Single Instruction Multiple Data) - udostępnia 8 64-bitowych rejestrów (MM0-MM7) które pozwalają na przechowywanie nowych typów danych - paczek. Pozwala to na lepszy rozkład danych do jednostek obliczeniowych np. przy mnożeniu macierzy, wektorów.\\
SSE (Streaming SIMD Extensions) - rozszerzenie MMX na zmiennoprzecinkowe typy danych - udostępnia 8 128-bitowych rejestrów (XMM0-XMM7)
\end{document}

