\documentclass[a4paper]{article}
\usepackage[polish]{babel}
\usepackage[utf8x]{inputenc}
\usepackage[T1]{fontenc}

\usepackage[a4paper,top=3cm,bottom=2cm,left=3cm,right=3cm,marginparwidth=1.75cm]{geometry}
\usepackage{amsmath}
\usepackage{graphicx}

\author{}
\title{PA - 1. kolokwium}
\date{}
\begin{document}
\maketitle
\section{Definicje charakterystyk}
\begin{itemize}
\item \textbf{Transmitancją operatorową} nazywamy stosunek transformaty Laplace’a odpowiedzi do transformaty Laplace’a wymuszenia przy zerowych warunkach początkowych. Jest ona najczęściej wykorzystywana w analizie i syntezie układów sterowania; pozwala uzyskać niezbędne dane o obiekcie i jego zachowaniu się w przypadku różnych wymuszeń
\item \textbf{Charakterystyką czasową} nazywamy przebieg czasowy wielkości wyjściowej wywołany danym wymuszeniem.
\item \textbf{Charakterystyką (odpowiedzią) impulsową g(t) układu} nazywamy odpowiedź tego układu na wymuszenie w postaci impulsu Diraca $\delta(t)$ przy zerowych warunkach początkowych.\\ Z definicji transmitancji operatorowej mamy $Y(s) = G(s)X(s)$. Przyjąwszy wymuszenie $x(t)$ w postaci impulsu Diraca:
$$
\delta (t) = 
\begin{cases}
\infty,& t = 0\\
0,& t\neq 0 \\ 
\end{cases}
$$
otrzymujemy transformatę Laplace'a wymuszenia $X(s)=1$, stąd $Y(s)=G(s)$.\\
Odpowiedź czasową uzyskamy jako transformatę odwrotną Laplace'a z wyrażenia $Y(s)$, czyli:
$$y(t)=g(t)=\mathcal{L}^{-1}(Y(s))=\mathcal{L}^{-1}(G(s))$$
Odpowiedź impulsowa $g(t)$ sygnału jest zatem oryginałem jego transmitancji operatorowej $G(s)$
\item \textbf{Charakterystyką (odpowiedzią) skokową h(t) układu} nazywamy odpowiedź tego układu na wymuszenie w postaci skoku jednostkowego 1(t) przy zerowych warunkach początkowych. Wymuszenie w postaci skoku jednostkowego zapisujemy następująco: 
$$
1(t) = 
\begin{cases}
0,& t<0\\
1,& t\geq 0\\
\end{cases}
$$
Zatem transformata Laplace'a funkcji traktowanej jako wymuszenie $x(t) = 1(t)$ wynosi:
$$ X(s) = \frac{1}{s} $$
Z definicji transmitancji operatorowej uzyskamy:
$$ Y(s) = G(s)X(s) = G(s)\frac{1}{s} $$
Charakterystyka skokowa przyjmie zatem postać:
$$ h(t) = \mathcal{L}^{-1} \bigg[ \frac{G(s)}{s} \bigg] = \int^t_0 g(\tau)d\tau
\textbf{\ \ lub inaczej \ \ } g(t) = \frac{dh(t)}{dt}$$
Pochodna odpowiedzi skokowej jest zatem oryginałem transmitancji operatorowej. \pagebreak
\item \textbf{Transmitancją widmową układu} nazywamy stosunek wartości zespolonej składowej wymuszonej odpowiedzi YW tego układu wywołanej wymuszeniem sinusoidalnym do wartości zespolonej tego wymuszenia:
$$ G(j\omega)=\frac{Y_W}{X} $$
Sinusoidalny sygnał wejściowy możemy zapisać w postaci: $$X = A_x e^{j\omega t}$$ Odpowiedź tego układu liniowego o parametrach stałych i skupionych zapiszemy jako $$Y_W = A_{YW}e^{j(\omega t + \phi)} $$
Innymi słowy, przejście funkcji harmonicznej przez układ, które opisuje transmitancja widmowa, odbywa się bez zmiany jego charakteru, lecz przy zmianie amplitudy $A_YW$ i przesunięcia fazowego $\phi$.\\
Transmitancję widmową możemy także zapisać równaniem postaci:
$$G(j \omega = \frac{A_{YW}(\omega) e^{j \phi (\omega)}}{A_X(\omega)}$$
uwzględniając w sposób wyraźny fakt, że amplituda i faza są zależne od pulsacji $\omega$. \\
Stosunek amplitudy składowej wymuszonej sygnału wyjściowego do amplitudy sygnału wejściowego określa moduł transmitancji widmowej:
$$|G(j\omega)|=\frac{A_{YW}(\omega)}{A_X(\omega)}$$
Podobnie określimy argument transmitancji widmowej:
$$\phi(\omega) = \text{arg}(G(j\omega))$$
Transmitancję widmową możemy również zapisać w postaci algebraicznej, wyróżniając część rzeczywistą i urojoną: 
$$G(j\omega) = P(\omega) + jQ(\omega) $$
\end{itemize}
\section{Podstawowe charakterystyki układów sterowanie}
\begin{itemize}
\item \textbf{Charakterystyką amplitudowo-fazową układu} nazywamy wykres transmitancji widmowej tego układu na płaszczyźnie zmiennej zespolonej.
\item \textbf{Charakterystyką amplitudową układu} nazywamy zależność modułu transmitancji widmowej $G(j\omega)$ w funkcji pulsacji $\omega$.
\item \textbf{Charakterystyką fazową układu} nazywamy zależność argumentu transmitancji widmowej $\phi(\omega)$ od pulsacji $\omega$.
\item \textbf{Logarytmiczną charakterystyką amplitudową} nazywamy zależność $20 \log G(j\omega)$ w funkcji $\log \omega$.
\item \textbf{Logarytmiczną charakterystyką fazową} nazywamy zależność $\phi(\omega)$ w funkcji $\log \omega$. \pagebreak
\end{itemize}

\section{Charakterystyka podstawowych członów układów sterowania}
\subsection{Człon bezinercyjny}
\begin{itemize}
\item \emph{Transmitancja: } $G(s) = k$, k -- współczynnik wzmocnienia
\item \emph{Przykład: } dźwignia jednostronna i dwustronna, wzmacniacz, zawór
\item \emph {Oznaczenie: } \\\\\\\\\\
\item \emph{Charakterystyka skokowa: } \\\\\\\\\\
\item \emph{Charakterystyka amplitudowo-fazowa: } \\\\\\\\\\
\end{itemize}
\subsection{Człon inercyjny pierwszego rzędu}
\begin{itemize}
\item \emph{Transmitancja: } $G(s) = \frac{k}{1+sT}$, k -- współczynnik wzmocnienia; T -- stała czasowa inercji
\item \emph{Przykład: } wzmacniacz rzeczywisty, maszyny proste, zawór
\item \emph {Oznaczenie: } \\\\\\\\\\
\item \emph{Charakterystyka skokowa: } \\\\\\\\\\
\item \emph{Charakterystyka amplitudowo-fazowa: } \\\\\\\\\\
\end{itemize} \pagebreak
\subsection{Człon inercyjny drugiego rzędu}
\begin{itemize}
\item \emph{Transmitancja: } $G(s) = \frac{k}{(1+sT_1)(1+sT_2)}$, k -- współczynnik wzmocnienia; T1, T2 -- stałe czasowe inercji
\item \emph{Przykład: } wielokrążek, zawory z uwzględnieniem zjawisk niekorzystnych
\item \emph {Oznaczenie: } \\\\\\\\\\
\item \emph{Charakterystyka skokowa: } \\\\\\\\\\
\item \emph{Charakterystyka amplitudowo-fazowa: } \\\\\\\\\\
\end{itemize}
\subsection{Obiekt różniczkujący idealny}
\begin{itemize}
\item \emph{Transmitancja: } $G(s) = ks$, k -- współczynnik wzmocnienia
\item \emph{Przykład: } \textbf{nierealizowalny fizycznie}
\item \emph {Oznaczenie: } \\\\\\\\\\
\item \emph{Charakterystyka skokowa: } \\\\\\\\\\
\item \emph{Charakterystyka amplitudowo-fazowa: } \\\\\\\\\\
\item \emph{Charakterystyka fazowa: } \\\\\\\\\\
\end{itemize}\pagebreak
\subsection{Obiekt różniczkujący rzeczywisty}
\begin{itemize}
\item \emph{Transmitancja: } $G(s) = \frac{ks}{1+sT}$, k -- współczynnik wzmocnienia; T -- stała czasowa
\item \emph{Przykład: } cewka indukcyjna, tłumik hydrauliczny, tarcie mechaniczne
\item \emph {Oznaczenie: } \\\\\\\\\\
\item \emph{Charakterystyka skokowa: } \\\\\\\\\\
\item \emph{Charakterystyka amplitudowo-fazowa: } \\\\\\\\\\
\end{itemize}
\subsection{Obiekt całkujący idealny}
\begin{itemize}
\item \emph{Transmitancja: } $G(s) = \frac{k}{s} = \frac{1}{T_is}, T_i = \frac{1}{k}$, k -- współczynnik wzmocnienia; T -- stała czasowa
\item \emph{Przykład: } kondensator idealny
\item \emph {Oznaczenie: } \\\\\\\\\\
\item \emph{Charakterystyka skokowa: } \\\\\\\\\\
\item \emph{Charakterystyka amplitudowo-fazowa: } \\\\\\\\\\
\item \emph{Logarytmiczna charakterystyka fazowa: } \\\\\\\\\\
\end{itemize} \pagebreak
\subsection{Obiekt całkujący rzeczywisty}
\begin{itemize}
\item \emph{Transmitancja: } $G(s) = \frac{k}{s(1+Ts)}$, k -- współczynnik wzmocnienia; T -- stała czasowa
\item \emph{Przykład: } kondensator, zbiornik cieczy
\item \emph {Oznaczenie: } \\\\\\\\\\
\item \emph{Charakterystyka skokowa: } \\\\\\\\\\
\item \emph{Charakterystyka amplitudowo-fazowa: } \\\\\\\\\\
\item \emph{Logarytmiczna charakterystyka fazowa: } \\\\\\\\\\
\end{itemize}
\subsection{Człon oscylacyjny}
\begin{itemize}
\item \emph{Transmitancja: } $G(s) = \frac{k\omega_0^2}{s^2+2\zeta\omega_0s+\omega_0^2}$, $\omega_0$ -- pulsacja oscylacji własnych; $\zeta$ -- względny współczynnik tłumienia $(0 < \zeta < 1)$; k -- wzmocnienie
\item \emph{Przykład: }  układy mechaniczne oscylujące (masa + sprężyna), elektryczny układ drgający, wahadło
\item \emph {Oznaczenie: } \\\\\\\\\\
\item \emph{Charakterystyka skokowa: } \\\\\\\\\\
\item \emph{Charakterystyka amplitudowo-fazowa: } \\\\\\\\\\
\end{itemize}
\subsection{Obiekt opóźniający}
\begin{itemize}
\item \emph{Transmitancja: } $G(s) = e^{-sT}$, T -- opóźnienie
\item \emph{Przykład: }  transporter taśmowy
\item \emph {Oznaczenie: } \\\\\\\\\\
\item \emph{Charakterystyka skokowa: } \\\\\\\\\\
\item \emph{Charakterystyka amplitudowo-fazowa: } \\\\\\\\\\
\item \emph{Charakterystyka fazowa: } \\\\\\\\\\
\end{itemize}
\iffalse
\section{Jakość regulacji}
\begin{itemize}
\item \textbf{Uchyb regulacji (uchyb statyczny)} to granica, do której dąży składowa wymuszona $e_w(t)$ sygnału uchybu $e(t)$ dla $ t \rightarrow \infty$
$$ e_u = \lim_{t \rightarrow \infty} e_w(t) = \lim_{t \rightarrow \infty } e(t) $$
Z twierdzenia granicznego wynika że:
$$ \lim_{t \rightarrow \infty} e_w(t) = \lim_{s \rightarrow 0}sE(s) \textbf{  gdzie  } E(s) = X(s)G_e(s) \textbf{  stąd  } e_u = \lim_{s \rightarrow 0}sX(s)G_e(s)$$
gdzie $G_e(s)$ to transmitancja uchybowa, $X(s)$ transformata Laplace'a sygnału wymuszenia, $E(s)$ transformata Laplace'a uchybu
\item \textbf{Układ regulacji statycznej} to układ automatycznej regulacji, którego uchyb w stanie ustalonym, przy wymuszeniu skokowym, jest różny od zera i proporcjonalny do amplitusy wymuszenia
\item \textbf{Układ astatyczny l-tego rzędu} to układ automatycznej regulacji, którego uchyb w stanie ustalonym jest równy zeru dla wszystkich sygnałów wejściowych, których pochodne, począwszy od l-tej, są równe zeru dla $t \rightarrow \infty $
\item \textbf{Układ z pełnym sprzężeniem zwrotnym} jest:
\begin{itemize}
\item \textbf{Układem statycznym} jeśli jego transmitancja w stanie otwartym nie ma biegunów zerowych (w układzie nie występują człony całkujące) 
\item \textbf{Układem astatycznym l-tego rzędu} jeśli jego transmitancja w stanie otwartym ma l-krotny biegun zerowy
\end{itemize}
\pagebreak
\item \textbf{Przeregulowanie} to wyrażony w procentach stosunek drugiej amplitudy uchybu do pierwszej:
$$ \kappa \frac{e_{p_2}}{e_{p_1}}100\% = \frac{y_{max}-y_{ust}}{y_{ust}} $$
jako że pierwsza amplituda odpowiedzi wynosi $y_{ust}$
\item \textbf{Czas regulacji} to czas jaki upłynął od momentu wystąpienia skokowej zmiany wartości zadanej (lub zakłócenia) do ustalenia się wahań uchybu $e(t)$ od $2$ do $5\%$ pierwszej amplitudy $e_{p_0}$ wokół wartości uchybu ustalonego.
\item \textbf{Zapasem stabilności amplitudy L [dB]} nazywamy wartość $\Delta k$, o jaką musi wzrosnąć wzmocnienie układu otwartego przy niezmienionej fazie, aby układ zamknięty znalazł się na granicy stabilności.
$$ L = 20 \log \Delta k$$
\item \textbf{Zapasem stabilności fazy} nazywamy wartość, o jaką musi wzrosnąć faza układu otwartego przy niezmienionym wzmocnieniu, aby układ znalazł się na granicy stabilności.
\section{Zadanie}
Zmierz stabilność układu regulacji o transmitancji układu otwartego równej:
$$ G(s) = \frac{2s+1}{s^3+3s^2+2s+1}$$
\end{itemize}
\fi
\end{document}

