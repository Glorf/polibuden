\documentclass[a4paper]{article}
\usepackage[polish]{babel}
\usepackage[utf8x]{inputenc}
\usepackage[T1]{fontenc}

\usepackage[a4paper,top=3cm,bottom=2cm,left=3cm,right=3cm,marginparwidth=1.75cm]{geometry}
\usepackage{amsmath}
\usepackage{graphicx}

\author{}
\title{PA - 1. kolos in blanco}
\date{}
\begin{document}
\maketitle
\section{Definicje charakterystyk}
\begin{itemize}
\item \textbf{Transmitancją operatorową} \\\\\\\\
\item \textbf{Charakterystyką czasową} \\\\
\item \textbf{Charakterystyką (odpowiedzią) impulsową g(t) układu} \\\\\\\\\\
\item \textbf{Charakterystyką (odpowiedzią) skokową h(t) układu} \\\\\\\\\\\\\\
\item \textbf{Transmitancją widmową układu} \\\\\\\\\\\\\\\\\\\\
\end{itemize}
\section{Podstawowe charakterystyki układów sterowanie}
\begin{itemize}
\item \textbf{Charakterystyką amplitudowo-fazową układu} \\\\\\\\\pagebreak
\item \textbf{Charakterystyką amplitudową układu} \\\\\\\\
\item \textbf{Charakterystyką fazową układu} \\\\\\\\
\item \textbf{Logarytmiczną charakterystyką amplitudową} \\\\\\\\
\item \textbf{Logarytmiczną charakterystyką fazową} \\\\\\\\
\end{itemize}

\section{Charakterystyka podstawowych członów układów sterowania}
\subsection{Człon bezinercyjny}
\begin{itemize}
\item \emph{Transmitancja: }
\item \emph{Przykład: }
\item \emph {Oznaczenie: } \\\\\\\\\\
\item \emph{Charakterystyka skokowa: } \\\\\\\\\\
\item \emph{Charakterystyka amplitudowo-fazowa: } \\\\\\\\\\
\end{itemize}
\subsection{Człon inercyjny pierwszego rzędu}
\begin{itemize}
\item \emph{Transmitancja: }
\item \emph{Przykład: }
\item \emph {Oznaczenie: } \\\\\\\\\\
\item \emph{Charakterystyka skokowa: } \\\\\\\\\\
\item \emph{Charakterystyka amplitudowo-fazowa: } \\\\\\\\\\
\end{itemize}
\subsection{Człon inercyjny drugiego rzędu}
\begin{itemize}
\item \emph{Transmitancja: }
\item \emph{Przykład: }
\item \emph {Oznaczenie: } \\\\\\\\\\
\item \emph{Charakterystyka skokowa: } \\\\\\\\\\
\item \emph{Charakterystyka amplitudowo-fazowa: } \\\\\\\\\\
\end{itemize}
\subsection{Obiekt różniczkujący idealny}
\begin{itemize}
\item \emph{Transmitancja: }
\item \emph{Przykład: }
\item \emph {Oznaczenie: } \\\\\\\\\\
\item \emph{Charakterystyka skokowa: } \\\\\\\\\\
\item \emph{Charakterystyka amplitudowo-fazowa: } \\\\\\\\\\
\item \emph{Charakterystyka fazowa: } \\\\\\\\\\
\end{itemize}
\subsection{Obiekt różniczkujący rzeczywisty}
\begin{itemize}
\item \emph{Transmitancja: }
\item \emph{Przykład: }
\item \emph {Oznaczenie: } \\\\\\\\\\
\item \emph{Charakterystyka skokowa: } \\\\\\\\\\
\item \emph{Charakterystyka amplitudowo-fazowa: } \\\\\\\\\\
\end{itemize}
\subsection{Obiekt całkujący idealny}
\begin{itemize}
\item \emph{Transmitancja: }
\item \emph{Przykład: }
\item \emph {Oznaczenie: } \\\\\\\\\\
\item \emph{Charakterystyka skokowa: } \\\\\\\\\\
\item \emph{Charakterystyka amplitudowo-fazowa: } \\\\\\\\\\
\item \emph{Logarytmiczna charakterystyka fazowa: } \\\\\\\\\\
\end{itemize}
\subsection{Obiekt całkujący rzeczywisty}
\begin{itemize}
\item \emph{Transmitancja: }
\item \emph{Przykład: }
\item \emph {Oznaczenie: } \\\\\\\\\\
\item \emph{Charakterystyka skokowa: } \\\\\\\\\\
\item \emph{Charakterystyka amplitudowo-fazowa: } \\\\\\\\\\
\item \emph{Logarytmiczna charakterystyka fazowa: } \\\\\\\\\\
\end{itemize}
\subsection{Człon oscylacyjny}
\begin{itemize}
\item \emph{Transmitancja: } 
\item \emph{Przykład: }  
\item \emph {Oznaczenie: } \\\\\\\\\\
\item \emph{Charakterystyka skokowa: } \\\\\\\\\\
\item \emph{Charakterystyka amplitudowo-fazowa: } \\\\\\\\\\
\end{itemize}
\subsection{Obiekt opóźniający}
\begin{itemize}
\item \emph{Transmitancja: } 
\item \emph{Przykład: } 
\item \emph {Oznaczenie: } \\\\\\\\\\
\item \emph{Charakterystyka skokowa: } \\\\\\\\\\
\item \emph{Charakterystyka amplitudowo-fazowa: } \\\\\\\\\\
\item \emph{Charakterystyka fazowa: } \\\\\\\\\\
\end{itemize}
\iffalse
\section{Jakość regulacji}
\begin{itemize}
\item \textbf{Uchyb regulacji (uchyb statyczny)} to granica, do której dąży składowa wymuszona $e_w(t)$ sygnału uchybu $e(t)$ dla $ t \rightarrow \infty$
$$ e_u = \lim_{t \rightarrow \infty} e_w(t) = \lim_{t \rightarrow \infty } e(t) $$
Z twierdzenia granicznego wynika że:
$$ \lim_{t \rightarrow \infty} e_w(t) = \lim_{s \rightarrow 0}sE(s) \textbf{  gdzie  } E(s) = X(s)G_e(s) \textbf{  stąd  } e_u = \lim_{s \rightarrow 0}sX(s)G_e(s)$$
gdzie $G_e(s)$ to transmitancja uchybowa, $X(s)$ transformata Laplace'a sygnału wymuszenia, $E(s)$ transformata Laplace'a uchybu
\item \textbf{Układ regulacji statycznej} to układ automatycznej regulacji, którego uchyb w stanie ustalonym, przy wymuszeniu skokowym, jest różny od zera i proporcjonalny do amplitusy wymuszenia
\item \textbf{Układ astatyczny l-tego rzędu} to układ automatycznej regulacji, którego uchyb w stanie ustalonym jest równy zeru dla wszystkich sygnałów wejściowych, których pochodne, począwszy od l-tej, są równe zeru dla $t \rightarrow \infty $
\item \textbf{Układ z pełnym sprzężeniem zwrotnym} jest:
\begin{itemize}
\item \textbf{Układem statycznym} jeśli jego transmitancja w stanie otwartym nie ma biegunów zerowych (w układzie nie występują człony całkujące) 
\item \textbf{Układem astatycznym l-tego rzędu} jeśli jego transmitancja w stanie otwartym ma l-krotny biegun zerowy
\end{itemize}
\pagebreak
\item \textbf{Przeregulowanie} to wyrażony w procentach stosunek drugiej amplitudy uchybu do pierwszej:
$$ \kappa \frac{e_{p_2}}{e_{p_1}}100\% = \frac{y_{max}-y_{ust}}{y_{ust}} $$
jako że pierwsza amplituda odpowiedzi wynosi $y_{ust}$
\item \textbf{Czas regulacji} to czas jaki upłynął od momentu wystąpienia skokowej zmiany wartości zadanej (lub zakłócenia) do ustalenia się wahań uchybu $e(t)$ od $2$ do $5\%$ pierwszej amplitudy $e_{p_0}$ wokół wartości uchybu ustalonego.
\item \textbf{Zapasem stabilności amplitudy L [dB]} nazywamy wartość $\Delta k$, o jaką musi wzrosnąć wzmocnienie układu otwartego przy niezmienionej fazie, aby układ zamknięty znalazł się na granicy stabilności.
$$ L = 20 \log \Delta k$$
\item \textbf{Zapasem stabilności fazy} nazywamy wartość, o jaką musi wzrosnąć faza układu otwartego przy niezmienionym wzmocnieniu, aby układ znalazł się na granicy stabilności.
\section{Zadanie}
Zmierz stabilność układu regulacji o transmitancji układu otwartego równej:
$$ G(s) = \frac{2s+1}{s^3+3s^2+2s+1}$$
\end{itemize}
\fi
\end{document}

