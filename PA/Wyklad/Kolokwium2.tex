\documentclass[a4paper]{article}
\usepackage[polish]{babel}
\usepackage[utf8x]{inputenc}
\usepackage[T1]{fontenc}

\usepackage[a4paper,top=3cm,bottom=2cm,left=3cm,right=3cm,marginparwidth=1.75cm]{geometry}
\usepackage{amsmath}
\usepackage{graphicx}

\author{}
\title{PA - 2. kolokwium}
\date{}
\begin{document}
\maketitle
\section{Regulatory}
\begin{itemize}
\item \textbf{Regulator PI} realizuje działanie proporcjonalne i całkujące \\
$$G_r(s) = \frac{k_p}{1+Ts}(1+\frac{1}{T_is}) $$
Gdzie $T_i$ to stała czasowa zdwojenia\\
Charakterystyka skokowa: \\\\\\\\\\\\
\item \textbf{Regulator PD} realizuje działanie proporcjonalne i różniczkujące \\
$$G_r(s) = \frac{k_p}{1+Ts}(1+T_ds) $$
Gdzie $T_d$ to stała czasowa wyprzedzenia\\
Charakterystyka skokowa: \\\\\\\\\\\\
\item \textbf{Regulator PID} realizuje działanie proporcjonalne, całkujące i różniczkujące\\
$$G_r(s) = \frac{k_p}{1+Ts}(1+\frac{1}{T_is}+T_ds)$$
Gdzie $T_d$ i $T_i$ to odpowiednio stałe czasowe wyprzedzenia i zdwojenia
Charakterystyka skokowa: \\\\\\\\\\\\
\end{itemize}
\section{Pomiary}

\subsection{Pirometr}
Termometr optyczny badający spektrum fali czerwonej emitowanej przez ciało

\subsection{Termopara}
Czujnik temperatury składający się z dwóch przewodów wykonanych z innych stopów metali - wykorzystuje zjawisko różnego przewodnictwa elektrycznego metali pod działaniem temperatury

\section{Dobór nastaw}
Dla regulatora P: $$k_p = 0,5k_{kr}$$
Dla regulatora PI: $$k_p = 0,5k_{kr} \text{    } T_i=0,85T_{kr}$$
Dla regulatora PID: $$k_p = 0,5k_{kr} \text{     } T_i=0,85T_{kr} \text{     } T_d = 0,12T_{kr}$$


\section{Kryterium Hurwitza}
\subsection{Warunek konieczny}
Wszystkie współczynniki wielomianu muszą być dodatnie
\subsection{Macierz Hurwitza}
$$ H = 
\begin{bmatrix}
a_{m-1} & a_m & 0 & 0 & \ldots \\
a_{m-3} & a_{m-2} & a_{m-1} & a_m & \ldots \\
a_{m-5} & a_{m-4} & a_{m-3} & a_{m-2} & \ldots \\
\vdots & \vdots & \vdots & \vdots & \ddots \\
0&0&0&0&a_0
\end{bmatrix}$$
\subsection{Warunek dostateczny}
Wielomian jest wielomianem stabilnym wtedy i tylko wtedy gdy wszystkie jego minory główne (dla ludzi: wyznaczniki wyciętych niezerowych kwadratów) są dodatnie. Liczymy wyznaczniki sprawdzamy czy > 0 i bingo\\
W układzie zamkniętym interesuje nas tylko mianownik\\
W układzie otwartym liczymy jako sumę licznik+mianownik
\end{document}

